
\begin{figure}[htbp]
  \centering
  \begin{tikzpicture}
    \DRAWTORIC{3}
    \DRAWPLAQ{1}{1}
    \DRAWPLAQ{0}{1}
    \DRAWPLAQ{0}{2}
    \DRAWERROR{1}{1}{0}{z}
    \DRAWERROR{1}{0}{0}{z}
    \DRAWERROR{2}{1}{1}{z}
    \DRAWERROR{0}{0}{0}{z}
    \DRAWERROR{0}{1}{1}{z}
    \DRAWERROR{0}{2}{1}{z}
    \DRAWERROR{0}{2}{0}{z}
    \DRAWERROR{1}{2}{1}{z}
    \node[below of=Bx-1] {\emph{(a)}};
  \end{tikzpicture}
  \hspace{1cm}
  \begin{tikzpicture}
    \DRAWTORIC{3}
    \DRAWSTAR{1}{1}{3}
    \DRAWSTAR{2}{0}{3}
    \DRAWSTAR{2}{1}{3}
    \DRAWERROR{1}{2}{1}{x}
    \DRAWERROR{0}{1}{0}{x}
    \DRAWERROR{2}{2}{1}{x}
    \DRAWERROR{1}{1}{1}{x}
    \DRAWERROR{1}{0}{0}{x}
    \DRAWERROR{2}{0}{1}{x}
    \DRAWERROR{2}{0}{0}{x}
    \DRAWERROR{2}{1}{0}{x}
    \node[below of=Bx-1] {\emph{(b)}};
  \end{tikzpicture}
  \caption{Multiplication of (a) plaquette and (b) star operators will result in an operator that consists of the Pauli operators that reside on the overall boundary of the joint plaquettes or stars. This figure is inspired by others \cite{browne}.}\label{sf:fig_multistab}
\end{figure}