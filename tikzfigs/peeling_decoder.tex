\tikzset{
  anyon/.style={circle, fill=OrangeRed, minimum size=.2cm, inner sep=0},
  erasure/.style={NavyBlue, very thick},
  correction/.style={Green, very thick},
  description/.style={align=#1, text width=4cm},
  description/.default={left},
  error/.style={text=black, pos=0.5}
  }

\begin{figure}
  \centering
  \begin{tikzpicture}[on grid, scale=0.8]
    \node at (-.5,4) {\emph{(a)}};
    \draw[step=1cm,gray,thin] (0,0) grid (4,4);
    \draw[erasure] (1,1) -- (2,1) node[error]{$X$} -- (3,1) -- (3,2) -- (2,2) -- (1,2) -- cycle node[error]{$X$};
    \draw[erasure] (1,2) -- (1,3) -- (2,3) node[error]{$X$} -- (2,2);
    \node[description={center}] at (2, -.5) {initial state};

    \begin{scope}[shift={(6,0)}]
      \node at (-.5,4) {\emph{(b)}};
      \draw[step=1cm,gray,thin] (0,0) grid (4,4);
      \draw[erasure] (1,1) -- (2,1) node[anyon]{} -- (3,1) -- (3,2) -- (2,2) -- (1,2) node[anyon] (a) {} -- cycle;
      \draw[erasure] (a) -- (1,3) node[anyon]{} -- (2,3) node[anyon]{}-- (2,2);
      \node[description={center}] at (2, -.5) {identify syndrome};
    \end{scope}

    \begin{scope}[shift={(12,0)}]
      \draw[thin] (-0,4) -- ++(.5,0) ++(.5,0) node[anchor=west]{normal edge};
      \draw[thin] (0,3) -- ++(.5,0) node[error]{$X$} ++(.5,0) node[anchor=west]{Pauli error};
      \draw[erasure] (0,2) -- ++(.5,0) ++(.5,0) node[anchor=west, text=black]{erased edge};
      \draw[thin] (0,1) -- ++(.5,0) node[anyon,pos=.5]{} ++(.5,0) node[anchor=west]{syndrome};
      \draw[correction] (0,0)   -- ++(.5,0) ++(.5,0) node[anchor=west,text=black]{correction edge};
    \end{scope}

    \begin{scope}[shift={(0,-6)}]
      \node at (-.5,4) {\emph{(c)}};
      \draw[step=1cm,gray,thin] (0,0) grid (4,4);
      \draw[erasure] (1,3) node[anyon]{} -- (2,3) node[anyon]{} -- (2,2) -- (1,2) node[anyon]{} -- (1,1) -- (2,1) node[anyon]{} -- (3,1) -- (3,2);
      \node[description={center}] at (2, -.5) {construct $\m{T}_\m{R}$};
    \end{scope}

    \begin{scope}[shift={(6,-6)}]
      \node at (-.5,4) {\emph{(d)}};
      \draw[step=1cm,gray,thin] (0,0) grid (4,4);
      \draw[erasure] (1,3) node[anyon]{} -- (2,3) node[anyon]{} -- (2,2) -- (1,2) node[anyon]{} -- (1,1) -- (2,1) node[anyon](a){};
      \draw[erasure, dashed] (a) -- (3,1) node[pos=0, below, text=black]{$v$} node[pos=0.5, above]{$e$} node[pos=1, below, text=black]{$u$};
      \node[description={center}] at (2, -.5) {peel $e=(u,v), u \notin \sigma$};
    \end{scope}

    \begin{scope}[shift={(12,-6)}]
      \node at (-.5,4) {\emph{(e)}};
      \draw[step=1cm,gray,thin] (0,0) grid (4,4);
      \draw[erasure] (1,3) node[anyon]{} -- (2,3) node[anyon]{} -- (2,2) -- (1,2) node[anyon]{} -- (1,1);
      \draw[erasure, dashed] (1,1) node[below, text=black]{$v$} -- (2,1) node[anyon]{} node[pos=0.5, above]{$e$} node[below, text=black]{$u$};
      \node[description={center}] at (2, -.5) {peel $e=(u,v), u \in \sigma$};
    \end{scope}

    \begin{scope}[shift={(0,-12)}]
      \node at (-.5,4) {\emph{(f)}};
      \draw[step=1cm,gray,thin] (0,0) grid (4,4);
      \draw[erasure] (1,3) node[anyon]{} -- (2,3) node[anyon]{} --  (2,2) -- (1,2) node[anyon]{} -- (1,1) node[anyon](a){};
      \draw[correction] (a) node[below, text=black]{$v$} -- (2,1) node[pos=0.5, above]{$e$} node[below, text=black]{$u$};
      \node[description={center}] at (2, -.5) {flip $u,v$, add $e$ to $\m{C}$};
    \end{scope}

    \begin{scope}[shift={(6,-12)}]
      \node at (-.5,4) {\emph{(g)}};
      \draw[step=1cm,gray,thin] (0,0) grid (4,4);
      \draw[correction] (1,3) -- (2,3) node[error]{$X$} (2,1) -- (1,1) node[error]{$X$} -- (1,2) node[error]{$X$};
      \node[description={center}] at (2, -.5) {apply correction $\n{P}(\m{C})$};
    \end{scope}

    \begin{scope}[shift={(12,-12)}]
      \node at (-.5,4) {\emph{(h)}};
      \draw[step=1cm,gray,thin] (0,0) grid (4,4);
      \node[description={center}] at (2, -.5) {end state};
    \end{scope}
  \end{tikzpicture}
  \caption{Schematic representation of the Peeling decoder. On an erasure $\m{R}\subset \m{E}$ (a), there may be some Pauli errors $\m{E}_\m{R}\subset \m{R}$ that anticommutes with some stabilizer measurements (b) that is identified as the syndrome $\sigma$. The first step is to construct a spanning forest $\m{T}_\m{R}\subset \m{R}$, a fully connected acyclic graph. Next the decoder sequentially removes leaf edges $e=(u,v)$ from the forest that connect to the forest via only one vertex $v$. If $u\in\sigma$ (e), remove $u$ from $\sigma$, flip $v$ in $\sigma$ and the edge $e$ is added to the correction set $\m{C}$ (f). If $u \notin \sigma$, move on the the next leaf. After applying the correction $\n{P}(\m{C})$ (g), all errors on the lattice commutes with the stabilizers, potentially solving the error (h).}
\end{figure}