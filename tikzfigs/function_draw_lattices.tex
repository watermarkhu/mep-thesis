
\def\QS{15}
\def\s{1.5}
\def\lw{0.4}

\tikzset{
  qubit/.style={line width=\lw,circle,draw,fill=white,minimum size=\QS},
  bound/.style={line width=\lw,circle, fill=gray!50, minimum size=\QS},
  line/.style={line width=\lw},
  xer/.style={line width=\lw,circle,draw,fill=red!50,inner sep=0,text width=\QS,align=center},
  zer/.style={line width=\lw,circle,draw,fill=cyan!50,inner sep=0,text width=\QS,align=center},
  yer/.style={line width=\lw,circle,draw,fill=orange!50,inner sep=0,text width=\QS,align=center},
  plaq/.style={fill=cyan!20},
  vert/.style={red!50, line width=4*\lw},
  synz/.style={cyan!50, line width=4*\lw, line cap=round},
  synx/.style={purple!25, line width=4*\lw, line cap=round},
  empty/.style={inner sep=0},
  arrow/.style={->, line width=2*\lw},
  script/.style={rectangle, fill=white, opacity=.5, text opacity=1, inner sep=0}
  }

\pgfdeclarelayer{bg}
\pgfdeclarelayer{edges}
\pgfdeclarelayer{qubits}
\pgfsetlayers{bg,edges,qubits,main}

\newcommand{\MINUS}[1]{%
  \number\numexpr#1-1\relax%
}


\newcommand\DRAWTORIC[1]{
  % Draws a toric lattice
  % var 0:    lattice size

  \def\LS{\MINUS{#1}}
  \begin{pgfonlayer}{edges}
    \foreach \x in {0,...,\LS}{
      \draw (0,\s*\x) -- (\s*\LS+\s,\s*\x) (\s*\x, -\s) -- (\s*\x, \s*\LS);
      \draw[dotted] (0,\s*\x - \s/2) -- (\s*\LS+\s,\s*\x - \s/2) (\s*\x + \s/2, -\s) -- (\s*\x + \s/2, \s*\LS);
      }
    \draw[dashed] (0, -\s) -- (\s*\LS+\s, -\s) -- (\s*\LS+\s, \s*\LS);
  \end{pgfonlayer}
  \begin{pgfonlayer}{qubits}
    \foreach \x in {0,...,\LS}
      \foreach \y in {0,...,\LS}{
         \node [qubit] (N-\x-\y-0) at (\s*\x + \s/2, \s*\y) {};
         \node [qubit] (N-\x-\y-1) at (\s*\x, \s*\y - \s/2) {};
         \coordinate (P-\x-\y) at (\s*\x + \s/2, \s*\y - \s/2);
         \coordinate (S-\x-\y) at (\s*\x, \s*\y);
         }
    \foreach \x in {0,...,\LS}{
      \node [bound] (By-\x) at (\s*\LS+\s, \s*\x - \s/2) {};
      \node [bound] (Bx-\x) at (\s*\x + \s/2, -\s) {};
      \coordinate (S-\x-#1) at (\s*\x, -\s);
      \coordinate (S-#1-\x) at (\s*#1, \s*\x);
    }

  \end{pgfonlayer}
}

\newcommand\DRAWPLANAR[1]{
  % Draws a planar lattice#1
  % var 0:    lattice size

  \def\LS{\MINUS{#1}}
  \begin{pgfonlayer}{edges}
    \foreach \x in {1,...,\LS}{
      \draw (\s/2,\s*\x) -- (\s*\LS+\s/2,\s*\x) (\s*\x, 0) -- (\s*\x, \s*\LS);
      \draw[dotted] (\s*\x + \s/2, 0) -- (\s*\x + \s/2, \s*\LS) (\s/2,\s*\x - \s/2) -- (\s*\LS+\s/2,\s*\x - \s/2);
      }
    \draw (\s/2,0) -- (\s*\LS+\s/2,0);
    \draw[dotted] (\s/2, 0) -- (\s/2, \s*\LS);
  \end{pgfonlayer}
  \begin{pgfonlayer}{qubits}
    \foreach \x in {1,...,\LS}
      \foreach \y in {1,...,\LS}{
         \node [qubit] (N-\x-\y-0) at (\s*\x + \s/2, \s*\y) {};
         \node [qubit] (N-\x-\y-1) at (\s*\x, \s*\y - \s/2) {};
         \coordinate (P-\x-\y) at (\s*\x + \s/2, \s*\y - \s/2);
         \coordinate (S-\x-\y) at (\s*\x, \s*\y);
         }
    \foreach \x in {0,...,\LS}{
      \node [qubit] (N-\x-0-0) at (\s*\x + \s/2, 0) {};
      \node [qubit] (N-0-\x-0) at (\s/2, \s*\x) {};
      \coordinate (P-0-\x) at (\s/2, \s*\x - \s/2);
      \coordinate (S-\x-0) at (\s*\x, 0);
    }
  \end{pgfonlayer}
}


\newcommand\DRAWERROR[4]{
  % Draws a toric lattice
  % var 0:    y coordinate
  % var 1:    x coordinate
  % var 2:    td coordinate
  % var 3:    error type x,y,z
  \def\x{#1}
  \def\y{#2}
  \begin{pgfonlayer}{qubits}    % select the background layer
    \ifstrequal{#4}{x}%
      {\ifnumequal{#3}{0}
        {\node [xer] (\y,\x,0) at (\s*\x + \s/2, \s*\y) {\tiny X};}
        {\node [xer] (\y,\x,0) at (\s*\x, \s*\y - \s/2) {\tiny X};}}
      {}
    \ifstrequal{#4}{z}%
      {\ifnumequal{#3}{0}
        {\node [zer] (\y,\x,0) at (\s*\x + \s/2, \s*\y) {\tiny Z};}
        {\node [zer] (\y,\x,0) at (\s*\x, \s*\y - \s/2) {\tiny Z};}}
      {}
    \ifstrequal{#4}{y}%
      {\ifnumequal{#3}{0}
        {\node [yer] (\y,\x,0) at (\s*\x + \s/2, \s*\y) {\tiny Y};}
        {\node [yer] (\y,\x,0) at (\s*\x, \s*\y - \s/2) {\tiny Y};}}
      {}
  \end{pgfonlayer}
}

\newcommand\DRAWPLAQ[2]{
  % Draws a plaquette
  % var 0:    y coordinate
  % var 1:    x coordinate
  \def\x{#1}
  \def\y{#2}
  \begin{pgfonlayer}{bg}
    \fill[plaq] (\x*\s,\y*\s) rectangle (\x*\s+\s,\y*\s-\s);
  \end{pgfonlayer}
}

\newcommand\DRAWEPLAQ[2]{
  % Draws a plaquette
  % var 1:    x coordinate
  % var 2:    y coordinate
  \def\x{#1}
  \def\y{#2}
  \begin{pgfonlayer}{bg}
    \ifnumequal{\x}{0}
      {\fill[plaq] (.5*\s,\y*\s) rectangle (\s,\y*\s-\s);}
      {\fill[plaq] (\x*\s,\y*\s) rectangle (\x*\s + .5*\s,\y*\s-\s);}
  \end{pgfonlayer}
}

\newcommand\DRAWSTAR[3]{
  % Draws a star
  % var 0:    y coordinate
  % var 1:    x coordinate
  % var 2:    lattice size
  \def\x{#1}
  \def\y{#2}
  \begin{pgfonlayer}{edges}
    \ifnumequal{\x}{0}
      {\draw[vert] (#3*\s-\s/2, \y*\s) -- (#3*\s, \y*\s);}
      {\draw[vert] (\x*\s-\s/2, \y*\s) -- (\x*\s, \y*\s);}
    \ifnumequal{\y}{\LS}
      {\draw[vert] (\x*\s, -\s) -- (\x*\s, -\s/2);}
      {\draw[vert] (\x*\s, \y*\s) -- (\x*\s, \y*\s +\s/2);}
    \draw[vert] (\x*\s, \y*\s - \s/2) -- (\x*\s, \y*\s) -- (\x*\s + \s/2, \y*\s);
  \end{pgfonlayer}
}


\newcommand\DRAWESTAR[2]{
  % Draws a star
  % var 1:    x coordinate
  % var 2:    y coordinate
  \def\x{#1}
  \def\y{#2}
  \begin{pgfonlayer}{edges}
    \ifnumequal{\y}{0}
      {\draw[vert] (\x*\s, 0) -- (\x*\s, \s/2);}
      {\draw[vert] (\x*\s, \y*\s) -- (\x*\s, \y*\s-\s/2);}
    \draw[vert] (\x*\s - \s/2, \y*\s) -- (\x*\s + \s/2, \y*\s);
  \end{pgfonlayer}
}



\newcommand\DSPECTRUM[3]{
  \tikzmath{
    \size = #1;
    \X = #2;
    \Y = #3;
    \y = #3-0.1;
  }
  \ifnumequal{\X}{0}{}{
    \path[fill=white!50!black, rounded corners=1pt] (0, 0.1) rectangle (\X,\y);
  }
  \draw[line width =0.5] (0,\Y) -- (0,0) -- (\size,0) -- (\size,\Y);

  \foreach \xx in {0,...,\size}{
    \draw[line width=0.5, font=\footnotesize] (\xx,0) -- +(0,-0.1) node[below] {\xx};
  }
}

\newcommand\DSPECTRA[6]{
  \tikzmath{
    \M = #1;
    \I = #2;
    \x = #3;
    \y = #4;
    \h = #5;
    \w = #6;
    \X = \x + \M;
  }
  \begin{scope}[shift={(\x, 0)}]
  \begin{scope}[xscale=\w]
  
  \ifnumequal{\I}{0}{}{
    \draw[line width=0, fill=white!50!black, rounded corners=2pt] (0,\y) rectangle +(\I,\h);
  }
  \draw[thin, rounded corners = 2pt] (0,\y) rectangle +(\M,\h);
  \foreach [count=\i] \xx in {0,...,\M}{
    \draw[font=\tiny] (\xx,\y) node[below] {\xx};
  }
  \end{scope}
  \end{scope}
}