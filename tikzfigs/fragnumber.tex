\newcommand\Square[1]{+(-#1,-#1) rectangle +(#1,#1)}
\begin{figure}[htbp]
  \centering

  \begin{subfigure}[b]{\textwidth}
    \centering
    \begin{tikzpicture}[scale=0.45, on grid]
      \foreach \x in {0,...,2}{\foreach \y in {0,...,8}{
       \path[fill=white!80!black] (\x,\y) \Square{0.46cm};
      }}
      \draw[l1] (0,0) ++(-.5,-.5) rectangle +(3,9);

      \begin{scope}[shift={(5,0)}]
      \foreach \x in {0,...,2}{
        \foreach \y in {0,...,5}{
          \path[fill=white!50!black] (\x,\y) \Square{0.46cm};}
        \foreach \y in {6,...,8}{
          \path[fill=white!80!black] (\x,\y) \Square{0.46cm};}}

      \draw[l1,dashed] (-.5,-.5) ++(0,6) -- ++(0,-6) -- ++(3,0) -- +(0,6);
      \draw[l1] (0,6) ++(-.5,-.5) rectangle +(3,3);
      \end{scope}

      \begin{scope}[shift={(10,0)}]
        \foreach \x in {0,...,2}{\foreach \y in {0,...,8}{
       \path[fill=white!80!black] (\x,\y) \Square{0.46cm};
      }}
      \draw[l1] (0,0) ++(-.5,-.5) rectangle +(3,3);
      \draw[l1] (0,3) ++(-.5,-.5) rectangle +(3,3);
      \draw[l1] (0,6) ++(-.5,-.5) rectangle +(3,3);
      \end{scope}

      \begin{scope}[shift={(15,0)}]
      \foreach \x in {0,...,2}{
       \foreach \y in {0,1,3,4,6,7}{
       \path[fill=white!50!black] (\x,\y) \Square{0.46cm};}
       \foreach \y in {2,5,8}{
       \path[fill=white!80!black] (\x,\y) \Square{0.46cm};}}

      \foreach \y in {0,3,6}{
       \draw[l1] (0,\y) ++(-.5,1.5) rectangle + (3,1);
       \draw[l1, dashed] (0,\y) ++(-.5,-.5) -- +(0,2) (3,\y) ++(-.5,-.5) -- +(0,2);}
      \draw[l1, dashed] (0,0) ++(-.5,-.5) -- +(3,0);
      \end{scope}

      \begin{scope}[shift={(20,0)}]
        \foreach \x in {0,...,2}{\foreach \y in {0,...,8}{
       \path[fill=white!80!black] (\x,\y) \Square{0.46cm};
      }}
      \foreach \y in {0,...,8}{ \draw[l1] (0,\y) ++(-.5,-.5) rectangle +(3,1);}
      \end{scope}

      \begin{scope}[shift={(25,0)}]
       \foreach \y in {0,...,8}{\foreach \x in {1,2}{
       \path[fill=white!50!black] (\x,\y) \Square{0.46cm};}
       \path[fill=white!80!black] (0,\y) \Square{0.46cm};
       \draw[l1] (0,\y) ++(-.5,-.5) rectangle +(1,1);}

       \foreach \y in {0,...,9}{
         \draw[l1,dashed] (3,\y) ++(-.5,-.5) -- +(-2,0);}
          \draw[l1,dashed] (3,0) ++(-.5,-.5) -- +(0,9);
      \end{scope}

      \begin{scope}[shift={(30,0)}]
        \foreach \x in {0,...,2}{\foreach \y in {0,...,8}{
       \path[fill=white!80!black] (\x,\y) \Square{0.46cm};
       \draw[l1] (\x,\y) ++(-.5,-.5) rectangle +(1,1);
      }}
      \end{scope}

      \begin{scope}[shift={(25,11)}]
        \path[fill=white!80!black] (0,2) \Square{0.46cm};
        \draw[l1] (0,2) ++(-.5,-.5) rectangle +(1,1);
        \path[fill=white!50!black] (0,0) \Square{0.46cm};
        \draw[l1,dashed] (0,0) ++(-.5,-.5) rectangle +(1,1);
        \node[anchor=west] at (1,2) {odd node-tree};
        \node[anchor=west] at (1,0) {even node-tree};
      \end{scope}


      \foreach \x in {3,13,23}{ \draw[l1, ->] (\x,4) -- +(1,0) node[midway, above] {$f_o$};}
      \foreach \x in {8,18,28}{ \draw[l1, ->] (\x,4) -- +(1,0) node[midway, above] {$f_e$};}
      \node at (1,4) {$\nset^o$};
    \end{tikzpicture}
    \caption{A full fragmentation of $\nset^o$ with with $k_f = 2$ and $R_0 = R_1 = R_2 = \frac{1}{3}$. Note that after every $f_o$, the sum of the even node-trees or dark shaded squares is equal.}
    \label{fig:fragcorrect}
  \end{subfigure}

  \vspace{2em}
  
  \begin{subfigure}[b]{\textwidth}
    \centering    
    \begin{tikzpicture}[scale=0.45, on grid]
      \foreach \x in {0,...,2}{\foreach \y in {0,...,8}{
       \path[fill=white!80!black] (\x,\y) \Square{0.46cm};
      }}
      \draw[l1] (0,0) ++(-.5,-.5) rectangle +(3,9);

      \begin{scope}[shift={(5,0)}]
      \foreach \x in {0,...,2}{
        \foreach \y in {0,...,5}{
          \path[fill=white!50!black] (\x,\y) \Square{0.46cm};}
        \foreach \y in {6,...,8}{
          \path[fill=white!80!black] (\x,\y) \Square{0.46cm};}}

      \draw[l1,dashed] (-.5,-.5) ++(0,6) -- ++(0,-6) -- ++(3,0) -- +(0,6);
      \draw[l1] (0,6) ++(-.5,-.5) rectangle +(3,3);
      \end{scope}

      \begin{scope}[shift={(10,0)}]
        \foreach \x in {0,...,2}{\foreach \y in {0,...,8}{
       \path[fill=white!80!black] (\x,\y) \Square{0.46cm};
      }}
      \foreach \y in {0,...,5}{\draw[l1] (0,\y) ++(-.5,-.5) rectangle +(3,1);}
      \draw[l1] (0,6) ++(-.5,-.5) rectangle +(3,3);
      \end{scope}

      \begin{scope}[shift={(15,0)}]
      \foreach \y in {0,...,5}{
       \path[fill=white!80!black] (0,\y) \Square{0.46cm};
       \foreach \x in {1,2}{\path[fill=white!50!black] (\x,\y) \Square{0.46cm};}}
      \foreach \x in {0,1,2}{
       \path[fill=white!50!black] (\x,6) \Square{0.46cm};
       \path[fill=white!50!black] (\x,7) \Square{0.46cm};
       \path[fill=white!80!black] (\x,8) \Square{0.46cm};}
      \foreach \y in {0,...,5}{\draw[l1] (0,\y) ++(-.5,-.5) rectangle +(1,1);}
      \draw[l1] (0,8) ++(-.5,-.5) rectangle +(3,1);
      \foreach \y in {0,...,6}{\draw[l1,dashed] (3,\y) ++(-.5,-.5) -- +(-2,0);}
      \draw[l1,dashed] (3,0) ++(-.5,-.5) -- +(0,8) (0,6) ++(-.5,-.5) -- +(0,2);
      \end{scope}

      \begin{scope}[shift={(20,0)}]
        \foreach \x in {0,...,2}{\foreach \y in {0,...,8}{
       \path[fill=white!80!black] (\x,\y) \Square{0.46cm};
      }}
      \foreach \y in {6,7,8}{ \draw[l1] (0,\y) ++(-.5,-.5) rectangle +(3,1);}
      \foreach \y in {0,...,5}{
       \draw[l1,dashed] (3,\y) ++(-.5,-.5) -- +(-2,0);
       \foreach \x in {0,1,2}{\draw[l1] (\x,\y) ++(-.5,-.5) rectangle +(1,1);}}
      \end{scope}

      \begin{scope}[shift={(25,0)}]
       \foreach \y in {6,7,8}{\foreach \x in {1,2}{
       \path[fill=white!50!black] (\x,\y) \Square{0.46cm};}
       \path[fill=white!80!black] (0,\y) \Square{0.46cm};
       \draw[l1] (0,\y) ++(-.5,-.5) rectangle +(1,1);}
       \foreach \y in {7,8,9}{\draw[l1,dashed] (3,\y) ++(-.5,-.5) -- +(-2,0);}
       \draw[l1,dashed] (3,6) ++(-.5,-.5) -- +(0,3);

       \foreach \y in {0,...,5}{\foreach \x in {0,1,2}{
        \path[fill=white!80!black] (\x,\y) \Square{0.46cm};
        \draw[l1] (\x,\y) ++(-.5,-.5) rectangle +(1,1);}}
      \end{scope}

      \begin{scope}[shift={(30,0)}]
        \foreach \x in {0,...,2}{\foreach \y in {0,...,8}{
       \path[fill=white!80!black] (\x,\y) \Square{0.46cm};
       \draw[l1] (\x,\y) ++(-.5,-.5) rectangle +(1,1);
      }}
      \end{scope}

      \foreach \x in {3,13,23}{ \draw[l1, ->] (\x,4) -- +(1,0) node[midway, above] {$f_o$};}
      \foreach \x in {8,18,28}{ \draw[l1, ->] (\x,4) -- +(1,0) node[midway, above] {$f_e$};}
      \node at (1,4) {$\nset^o$};
    \end{tikzpicture}
    \caption{A full fragmentation of $\nset^o$ where in the first $f_o$, the fragmentation number is increased to $k_f = 6$. The number of dark shaded squares or $N_{PDC}$ has decreased from the fragmentation with optimal settings in (a). }\label{fig:fragfnumber}
  \end{subfigure}
  \caption{The full fragmentation of $\nset^o$ per equation \eqref{eq:fullfrag}. every odd node-tree in the fragmentation is a rectangle with continuous lines, and even node-tree has dashed lines. Every square is equivalent to a node, where the sum of all dark shaded squares is $N_{PDC}$}\label{fig:fragexamples}
\end{figure}