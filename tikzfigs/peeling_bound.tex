
\begin{figure}
    \centering
    \begin{tikzpicture}[on grid, scale=0.8]
      \node at (-.5,4) {a)};
      \draw[step=1cm,gray,thin] (0.1,0) grid (3.9,4);
      \draw[erasure] (0.1,1) -- (1,1)  -- (2,1)  node[error]{$X$} -- (2,2) -- (2,3) -- (1,3)  -- (1,2) node[error]{$X$}  -- (0.1,2)  node[error]{$X$} (1,1) -- (1,2) -- (2,2);
      \draw[erasure] (3.9,4) -- (3,4)  node[error]{$X$} -- (3,3)  -- (3,2)  node[error]{$X$}  -- (3.9,2) (2,3) -- (3,3) -- (3.9,3);
      \node[description={center}] at (2, -.5) {initial state};
  
      \begin{scope}[shift={(6,0)}]
        \node at (-.5,4) {b)};
        \draw[step=1cm,gray,thin] (0.1,0) grid (3.9,4);
        \draw[erasure] (0.1,1) -- (1,1) node[anyon]{} -- (2,1) node[anyon]{} -- (2,2) -- (2,3) -- (1,3) node[anyon]{} -- (1,2)  -- (0.1,2) (1,1) -- (1,2) -- (2,2);
        \draw[erasure] (3.9,4) -- (3,4) node[anyon]{} -- (3,3) node[anyon]{} -- (3,2) node[anyon]{}  -- (3.9,2) (2,3) -- (3,3) -- (3.9,3);
        \node[description={center}] at (2, -.5) {identify syndrome};
      \end{scope}
  
      \begin{scope}[shift={(12,-.5)}]
        \draw[thin] (0,4) -- ++(.5,0) ++(.5,0) node[anchor=west]{normal edge};
        \draw[thin] (0,3) -- ++(.5,0) node[error]{$X$} ++(.5,0) node[anchor=west]{Pauli error};
        \draw[erasure] (0,2) -- ++(.5,0) ++(.5,0) node[anchor=west, text=black]{erased edge};
        \draw[thin] (0,1) -- ++(.5,0) node[anyon,pos=.5]{} ++(.5,0) node[anchor=west]{syndrome};
        \draw[correction] (0,0)   -- ++(.5,0) ++(.5,0) node[anchor=west,text=black]{correction edge};
      \end{scope}

      \begin{scope}[shift={(0,-6)}]
        \node at (-.5,4) {c)};
        \draw[step=1cm,gray,thin] (0.1,0) grid (3.9,4);
        \draw[erasure] (0.1, 2) -- (1,2) -- (1,1) node[anyon]{} -- (2,1) node[anyon]{} -- (2,2) -- (2,3) -- (1,3) node[anyon]{};
        \draw[erasure] (3.9,4) -- (3,4) node[anyon]{} -- (3,3) node[anyon]{} -- (3,2) node[anyon]{};
        \node[description={center}] at (2, -.5) {construct $F_{\varepsilon}$};
      \end{scope}

      \begin{scope}[shift={(6,-6)}]
        \node at (-.5,4) {d)};
        \draw[step=1cm,gray,thin] (0.1,0) grid (3.9,4);
        \draw[correction] (0.1, 2) -- (1,2) node[error]{$X$} -- (1,1) node[error]{$X$} (2,1)  -- (2,2) node[error]{$X$} -- (2,3) node[error]{$X$} -- (1,3) node[error]{$X$};
        \draw[correction] (3.9,4) -- (3,4) node[error]{$X$} (3,3) -- (3,2) node[error]{$X$};
        \node[description={center}] at (2, -.5) {apply correction set $C$};
      \end{scope}

      \begin{scope}[shift={(12,-6)}]
        \node at (-.5,4) {e)};
        \draw[step=1cm,gray,thin] (0.1,0) grid (3.9,4);
        \path (1,1) -- (2,1) node[error]{$X$} -- (2,2) node[error]{$X$} -- (2,3) node[error]{$X$} -- (1,3) node[error]{$X$} -- (1,2) node[error]{$X$} -- (1,1) node[error]{$X$};
        \node[description={center}] at (2, -.5) {end state};
      \end{scope}

    \end{tikzpicture}
    \caption{Schematic visualization of the Peeling decoder on a surface with boundaries. On an erasure $\m{E}\subset E$ (a), there may be some Pauli errors $P\subset \m{E}$ that anticommutes with some stabilizer measurements (b) that is identified as the syndrome $\sigma$. (c) The forest $F_\m{E}$ now has the extra constriction that it can only support single element of $V_\delta$, the open vertices, and peeling is only allowed on pendant vertices $v\notin V_\delta$. (d) After peeling, the correction $C$ is outputted and can be applied to correct error. (e) The end state is now a cycle of errors, which commutes with the stabilizer. This is not a feature of the Peeling decoder, but is just an example.}
  \end{figure}