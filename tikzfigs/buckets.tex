\tikzstyle{cluster}=[circle, line width=1, fill=white, text=black, inner sep=1, minimum size=12]
\tikzstyle{ocluster}=[cluster, draw=Peach, minimum size=12, font=\small]
\tikzstyle{ec}=[line width=1, fill=white, text=black, inner sep=3, minimum size=12, font=\small, draw=LimeGreen, rounded corners=6pt]
\tikzstyle{oc}=[line width=1, fill=white, text=black, inner sep=3, minimum size=12, font=\small, draw=Peach, rounded corners=6pt]
\tikzstyle{arrow}=[-latex, line width=1]
\tikzstyle{lefttext} = [align=left, anchor=west]

\newcommand{\DRAWBUCKET}[3]{
    \path[line width=3, draw=black!50!white, fill=black!20!white] (#1,#2) ++(-1,1) -- ++(0,-2) -- ++(2,0) -- ++(0,2);
    \path[fill=black!50!white] (#1,#2) ++ (-1,-1) rectangle ++(2,.5);
    \draw (#1,#2) ++(0,-.75) node[align=center, text=white]{$b_#3$};
}

\begin{figure}[htbp]
    \centering
    
    \begin{tikzpicture}
        \DRAWBUCKET{0}{-.25}{0}
        \DRAWBUCKET{2.5}{-.25}{1}

        \node[ocluster] at (0,.25) {$c_1$};
        \node[ocluster] at (-.3,-.25) {$c_2$};
        \node[ocluster] at (.3,-.25) {$c_3$};

        \draw[arrow] (-.75,-1.5) -- +(0,-1.5);
        \draw[arrow] (3.25,-1.5) -- +(0,-1.5);
        \node[lefttext]                 (a1) at (.25,-2)    {Grow};
        \node[right=0pt of a1, oc]                          {$c_1$};
        \node[lefttext]                  at (.25, -2.5)      {Place in $b_1$};
        \node[ocluster]                 (a2) at (4.5,.25)   {};
        \node[lefttext, right=1pt of a2]                    {odd cluster};
        \node[ocluster, draw=LimeGreen] (a3) at (4.5,-.25)  {};
        \node[lefttext, right=1pt of a3]                    {even cluster};

        \begin{scope}[shift={(0,-4)}]
            \DRAWBUCKET{0}{-.25}{0}
            \DRAWBUCKET{2.5}{-.25}{1}
            \node[ocluster]     at (2.5,0)          {$c_1$};
            \node[ocluster]     at (-.3,0)          {$c_2$};
            \node[ocluster]     at (.3,0)           {$c_3$};
            \draw[arrow] (-.75,-1.5) -- +(0,-1.5);
            \node[lefttext]     (b1) at (-.5,-2)    {Grow};
            \node[right=0pt of b1, oc] (b2)         {$c_2$};
            \node[right=0pt of b2]                  {, merge};
            \node[lefttext]     (b3) at (-.5, -2.5) {with};
            \node[right=0pt of b3, oc] (b4)         {$c_1$};
            \node[right=0pt of b4]     (b5)         {to};
            \node[right=0pt of b5, ec]              {$c_{12}$}; 
            \draw[arrow] (-.75,-1.5) -- +(0,-1.5);
            \draw[arrow] (3.25,-1.5) -- +(0,-1.5);
            
        \end{scope}

        \begin{scope}[shift={(0,-8)}]
            \DRAWBUCKET{0}{-.25}{0}
            \DRAWBUCKET{2.5}{-.25}{1}
            \node[ocluster, opacity=.5] at (2.5,0)  {$c_1$};
            \draw[arrow] (-.75,-1.5) -- +(0,-2);
            \draw[arrow] (3.25,-1.5) -- +(0,-2);

            \node[lefttext]             (c1) at (-.5,-2)    {Grow};
            \node[right=0pt of c1, oc]  (c2)                {$c_3$};
            \node[right=0pt of c2]                          {, merge};
            \node[lefttext]             (c3) at (-.5, -2.5) {with};
            \node[right=1.5em of c3, ec]  (c4) at (0, -2.5)   {$c_{12}$}; 
            \node[right=0pt of c4]      (c5)                {to}; 
            \node[right=0pt of c5, oc]                      {$c_{123}$};
            \node[lefttext]             at (-.5, -3)        {Place in $b_k$};
        \end{scope}
        \begin{scope}[shift={(0,-12.5)}]
            \DRAWBUCKET{0}{-.25}{0}
            \DRAWBUCKET{2.5}{-.25}{1}
            \node[ocluster, opacity=.5] at (2.5,0) {$c_1$};
        \end{scope}

    \end{tikzpicture}

    \caption{Faulty entries of clusters can occur in the buckets. If an odd-parity cluster $c_1$ is from grown from $b_0$ and placed into $b_1$, after which cluster $c_2$ grows and merges into $c_1$, resulting in an even-parity cluster $c_{12}$, the entry of $c_1$ in $b_1$ is faulty. A third cluster $c_3$ can merge with $c_{12}$ again to form a new odd-parity cluster that is now placed in a higher bucket than $b_1$. This problem can be solved by additionally storing the bucket number $k$ at the cluster as $c.bucket$ and checking for this number when growing clusters from $b_k$.}\label{3.fig.clustermergeB}
\end{figure}

