\documentclass[11pt, a4paper, twoside, titlepage, dvipsnames]{report}
\usepackage[page, toc]{appendix}
\usepackage[a4paper, total={6in, 8in}]{geometry}
\usepackage[section]{placeins}
\usepackage[utf8]{inputenc}
\usepackage[hidelinks]{hyperref}
\usepackage[mathscr]{euscript}
\usepackage[table]{xcolor}
\usepackage{physics, adjustbox}
\usepackage{amsfonts, amsmath, amssymb, amsthm}
\usepackage{framed,enumitem} 
\usepackage{xparse, todonotes}
\usepackage{titlesec}
\usepackage{cleveref}

% Define colors
\definecolor{mblue}{HTML}{1F77B4}
\definecolor{morange}{HTML}{FF7F0E}
\definecolor{mgreen}{HTML}{2CA02C}
\definecolor{mred}{HTML}{D62728}
\definecolor{mpurple}{HTML}{9467BD}
\definecolor{mbrown}{HTML}{8C564B}
\definecolor{mpink}{HTML}{E377C2}
\definecolor{mgrey}{HTML}{7F7F7F}
\definecolor{mlime}{HTML}{BCBD22}
\definecolor{mcyan}{HTML}{17BECF}


% Set math environments
\setlength{\parindent}{24pt}
\newtheorem{definition}{Definition}[chapter]
\newtheorem{lemma}{Lemma}[chapter]
\newtheorem{theorem}{Theorem}[chapter]
\newtheorem{proposition}{Proposition}[chapter]

% Set new commands
\newcommand{\codeword}[1]{\texttt{\textcolor{MidnightBlue}{#1}}}
%\newcommand{\codefunc}[1]{\texttt{\textcolor{OliveGreen}{#1}}}
\let\oldemptyset\emptyset
\let\emptyset\varnothing
\newcommand{\codefunc}[1]{\texttt{#1}}
\newcommand{\m}[1]{\mathcal{#1}}
\newcommand{\n}[1]{\mathscr{#1}}
\newcommand{\bound}{\mathscr{B}}
\newcommand{\akker}{\mathscr{A}}
\newcommand{\nset}{\mathcal{N}}
\newcommand{\vset}{\mathcal{V}}
\newcommand{\pre}[1]{ {}^{#1} }
\newcommand{\ceil}[1]{{\left \lceil #1 \right \rceil }}
\newcommand{\floor}[1]{{\left \lfloor #1 \right \rfloor }}


% Set graphics folder
\usepackage{float}
\usepackage{graphicx}
\usepackage[format=plain, font=it]{caption}
\usepackage[format=plain, font=it]{subcaption}
\graphicspath{{"/home/watermarkhu/links/onedriveuni/MEP - thesis Mark/Figures/ch_toric_code/"}}

% Set algorithm2e package settings
\usepackage[linesnumbered, vlined, boxruled]{algorithm2e}
\SetAlgoCaptionLayout{centerline}
\setlength{\algoheightrule}{1pt}
\setlength{\algotitleheightrule}{1pt}
\setlength{\interspacetitleboxruled}{.5em}
\SetStartEndCondition{ }{}{}
\SetKwProg{Fn}{def}{\string:}{}
\SetKw{KwTo}{in}
\SetKwFor{For}{for}{\string:}{}
\SetKwIF{If}{ElseIf}{Else}{if}{ then}{else if}{else}{}
\SetKwFor{While}{while}{ do}{}
\SetAlgoNoEnd\DontPrintSemicolon

% Set algorithm title custom box
\usepackage[most]{tcolorbox}
\tcbset{algotitle/.style={title={ \strut Algorithm~\thetcbcounter\ifstrempty{#1}{\ignorespaces}{:~#1}}}}
\newtcolorbox[auto counter]{algo}[1][]{
    colback=white, colframe=black, boxrule=0.5pt,
    titlerule=0pt, sharp corners, colbacktitle=white,enhanced,
    attach boxed title to top center={yshift=-10pt},
    boxed title style={boxrule=-1pt},
    fonttitle=\bfseries, coltitle=black, algotitle={}, #1}


% listings of pseudo-codes
\usepackage{listings}
\lstset{language=C,keywordstyle={\bfseries \color{blue}}}

\usepackage{tikz, pgfplots}
\usepgfplotslibrary{groupplots,dateplot}
\usetikzlibrary{quantikz,shapes,calc,math,patterns,arrows}
\def\checkmark{\tikz\fill[scale=0.4](0,.35) -- (.25,0) -- (1,.7) -- (.25,.15) -- cycle;}
\SetAlFnt{\footnotesize}
\tikzset{bellstar/.style ={draw, star,star points = 8, star point height = 1em, star point ratio = 1.8, minimum size = 1em, inner sep = 0pt}}
\tikzset{selection/.style={draw, signal, signal to =south, minimum size = 1em, inner sep = 0pt}}
\tikzset{sel/.style={label position = below, yshift = 0.5cm}}

\DeclareUnicodeCharacter{2212}{−}  
\pgfplotsset{compat=newest}


% Glossary
\usepackage[symbols,nomain,automake]{glossaries-extra}
\usepackage{glossary-mcols}

\makeglossaries
\glsxtrnewsymbol[description={Pauli group}]{spauligroup}{\ensuremath{\m{P}}}
\glsxtrnewsymbol[description={Pauli X operator}]{oX}{\ensuremath{\hat{X}}}
\glsxtrnewsymbol[description={Pauli Y operator}]{oY}{\ensuremath{\hat{Y}}}
\glsxtrnewsymbol[description={Pauli Z operator}]{oZ}{\ensuremath{\hat{Z}}}
\glsxtrnewsymbol[description={Identity operator}]{oI}{\ensuremath{\hat{I}}}
\glsxtrnewsymbol[description={Error operator}]{oerror}{\ensuremath{\hat{E}}}
\glsxtrnewsymbol[description={Logical X operator}]{oXlogical}{\ensuremath{\hat{X}_L}}
\glsxtrnewsymbol[description={Logical Z operator}]{oZlogical}{\ensuremath{\hat{Z}_L}}
\glsxtrnewsymbol[description={Stabilizer group}]{sstabilizergroup}{\ensuremath{\m{S}}}
\glsxtrnewsymbol[description={Stabilizer, element of $\m{S}$}]{ostabilizer}{\ensuremath{\hat{S}}}
\glsxtrnewsymbol[description={Stabilizer generator, generates $\m{S}$}]{sstabilizergenset}{\ensuremath{\mathfrak{s}}}
\glsxtrnewsymbol[description={Stabilizer generator, generates $\m{S}$}]{ostabilizergenerator}{\ensuremath{\hat{s}}}
\glsxtrnewsymbol[description={Syndrome set}]{ssyndrome}{\ensuremath{\sigma}}
\glsxtrnewsymbol[description={Plaquette operator}]{oplaquette}{\ensuremath{\hat{P}_f}}
\glsxtrnewsymbol[description={Star operator}]{ostar}{\ensuremath{\hat{S}_v}}
\glsxtrnewsymbol[description={Graph of a surface code}]{ngraph}{\ensuremath{G}}
\glsxtrnewsymbol[description={Vertex set}]{svertices}{\ensuremath{\m{V}}}
\glsxtrnewsymbol[description={vertex, element of $V$}]{nvertex}{\ensuremath{v}}
\glsxtrnewsymbol[description={Edge set}]{sedges}{\ensuremath{\m{E}}}
\glsxtrnewsymbol[description={edge, element of $E$}]{nedge}{\ensuremath{e}}
\glsxtrnewsymbol[description={Face set}]{sfaces}{\ensuremath{\m{A}}}
\glsxtrnewsymbol[description={Face, element of $F$}]{nface}{\ensuremath{f}}
\glsxtrnewsymbol[description={Probability of error correction}]{zpcorrect}{\ensuremath{k_C}}
\glsxtrnewsymbol[description={Threshold probability}]{zpthres}{\ensuremath{p_{th}}}
\glsxtrnewsymbol[description={Error operator}]{operror}{\ensuremath{\hat{E}}}
\glsxtrnewsymbol[description={Correction operator}]{ocorrection}{\ensuremath{\hat{C}}}
\glsxtrnewsymbol[description={Erasure, subset of $E$}]{serasure}{\ensuremath{\m{R}}}
\glsxtrnewsymbol[description={Forest of erasure $\m{R}$}]{sforest}{\ensuremath{\m{T}_\m{R}}}
\glsxtrnewsymbol[description={Pauli product of set of edges}]{ppauliproduct}{\ensuremath{\mathscr{P}}}
\glsxtrnewsymbol[description={Correction set, subset of $E$}]{scorrectionset}{\ensuremath{\m{C}}}
\glsxtrnewsymbol[description={Boundary of an edge set}]{pboundary}{\ensuremath{\mathscr{B}}}
\glsxtrnewsymbol[description={Cluster, subgraph of $G$}]{ncluster}{\ensuremath{c}}
\glsxtrnewsymbol[description={Disjoint-set tree of vertices}]{zuftree}{\ensuremath{T}}
\glsxtrnewsymbol[description={List of odd clusters}]{sloddlist}{\ensuremath{\m{L}_o}}
\glsxtrnewsymbol[description={List of merging edges}]{slmerging}{\ensuremath{\m{L}_m}}
\glsxtrnewsymbol[description={List of buckets}]{slbuckets}{\ensuremath{\m{L}_b}}
\glsxtrnewsymbol[description={Bucket with number $k$}]{zbucket}{\ensuremath{b_k}}
\glsxtrnewsymbol[description={List of clusters for placement}]{slplace}{\ensuremath{\m{L}_p}}
\glsxtrnewsymbol[description={Node set}]{snodeset}{\ensuremath{\nset}}
\glsxtrnewsymbol[description={Node, element of $\nset$}]{nn}{\ensuremath{n}}
\glsxtrnewsymbol[description={Node radius}]{nnradius}{\ensuremath{n.r}}
\glsxtrnewsymbol[description={Node parity}]{nnparity}{\ensuremath{n.p}}
\glsxtrnewsymbol[description={Node delay}]{nndelay}{\ensuremath{n.d}}
\glsxtrnewsymbol[description={Node wait}]{nnwait}{\ensuremath{n.w}}
\glsxtrnewsymbol[description={Node absolute delay}]{nndelaya}{\ensuremath{n.D}}
\glsxtrnewsymbol[description={Syndrome-node}]{nsyndromenode}{\ensuremath{s}}
\glsxtrnewsymbol[description={Linking-node}]{nlinkingnode}{\ensuremath{l}}
\glsxtrnewsymbol[description={Boundary-node}]{nboundarynode}{\ensuremath{b}}
\glsxtrnewsymbol[description={Equilibrium factor}]{zkeq}{\ensuremath{k_{eq}}}
\glsxtrnewsymbol[description={Fragmentation set of generation $k$}]{sfragmentation}{\ensuremath{\m{F}_k}}
\glsxtrnewsymbol[description={Fragmentation step}]{zfstep}{\ensuremath{f}}
\glsxtrnewsymbol[description={Partial fragmentation of odd $\nset$}]{zfpe}{\ensuremath{f_e}}
\glsxtrnewsymbol[description={Partial fragmentation of even $\nset$}]{zfpo}{\ensuremath{f_o}}
\glsxtrnewsymbol[description={Fragmentation ratio}]{zfragratio}{\ensuremath{R}}
\glsxtrnewsymbol[description={Fragmentation number}]{zkfragnumber}{\ensuremath{k_f}}


\renewcommand*{\glossarypreamble}{\vspace{-\baselineskip}}
\newcommand*{\Ngroupname}{Objects, elements of sets}
\newcommand*{\Ogroupname}{Operators}
\newcommand*{\Sgroupname}{Sets and groups}
\newcommand*{\Pgroupname}{Set operations}
\newcommand*{\Zgroupname}{Others}

% Referencing
\usepackage[style=numeric-comp, sorting=none]{biblatex}
\addbibresource{cit.bib}

% Tables
\usepackage{tabularx, hhline, multirow} 
\newcolumntype{L}[1]{>{\hsize=#1\hsize\raggedright\arraybackslash}X}%
\newcolumntype{R}[1]{>{\hsize=#1\hsize\raggedleft\arraybackslash}X}%
\newcolumntype{C}[1]{>{\hsize=#1\hsize\centering\arraybackslash}X}%
\newcommand{\gc}{\cellcolor[gray]{0.9}}

\title{mep-thesis}
\author{Shui Hu}
\date{April 2019}


\begin{document}

    
\def\QS{15}
\def\s{1.5}
\def\lw{0.4}

\tikzset{
  qubit/.style={line width=\lw,circle,draw,fill=white,minimum size=\QS},
  bound/.style={line width=\lw,circle, fill=gray!50, minimum size=\QS},
  line/.style={line width=\lw},
  xer/.style={line width=\lw,circle,draw,fill=red!50,inner sep=0,text width=\QS,align=center},
  zer/.style={line width=\lw,circle,draw,fill=cyan!50,inner sep=0,text width=\QS,align=center},
  yer/.style={line width=\lw,circle,draw,fill=orange!50,inner sep=0,text width=\QS,align=center},
  plaq/.style={fill=cyan!20},
  vert/.style={red!50, line width=4*\lw},
  synz/.style={cyan!50, line width=4*\lw, line cap=round},
  synx/.style={purple!25, line width=4*\lw, line cap=round},
  empty/.style={inner sep=0},
  arrow/.style={->, line width=2*\lw},
  script/.style={rectangle, fill=white, opacity=.5, text opacity=1, inner sep=0}
  }

\pgfdeclarelayer{bg}
\pgfdeclarelayer{edges}
\pgfdeclarelayer{qubits}
\pgfsetlayers{bg,edges,qubits,main}

\newcommand{\MINUS}[1]{%
  \number\numexpr#1-1\relax%
}


\newcommand\DRAWTORIC[1]{
  % Draws a toric lattice
  % var 0:    lattice size

  \def\LS{\MINUS{#1}}
  \begin{pgfonlayer}{edges}
    \foreach \x in {0,...,\LS}{
      \draw[line width=\lw] (0,\s*\x) -- (\s*\LS+\s,\s*\x) (\s*\x, -\s) -- (\s*\x, \s*\LS);
      \draw[line width=\lw, dotted] (0,\s*\x - \s/2) -- (\s*\LS+\s,\s*\x - \s/2) (\s*\x + \s/2, -\s) -- (\s*\x + \s/2, \s*\LS);
      }
    \draw[line width=\lw, dashed] (0, -\s) -- (\s*\LS+\s, -\s) -- (\s*\LS+\s, \s*\LS);
  \end{pgfonlayer}
  \begin{pgfonlayer}{qubits}
    \foreach \x in {0,...,\LS}
      \foreach \y in {0,...,\LS}{
         \node [qubit] (N-\x-\y-0) at (\s*\x + \s/2, \s*\y) {};
         \node [qubit] (N-\x-\y-1) at (\s*\x, \s*\y - \s/2) {};
         \coordinate (P-\x-\y) at (\s*\x + \s/2, \s*\y - \s/2);
         \coordinate (S-\x-\y) at (\s*\x, \s*\y);
         }
    \foreach \x in {0,...,\LS}{
      \node [bound] (By-\x) at (\s*\LS+\s, \s*\x - \s/2) {};
      \node [bound] (Bx-\x) at (\s*\x + \s/2, -\s) {};
      \coordinate (S-\x-#1) at (\s*\x, -\s);
      \coordinate (S-#1-\x) at (\s*#1, \s*\x);
    }

  \end{pgfonlayer}
}

\newcommand\DRAWPLANAR[1]{
  % Draws a planar lattice#1
  % var 0:    lattice size

  \def\LS{\MINUS{#1}}
  \begin{pgfonlayer}{edges}
    \foreach \x in {1,...,\LS}{
      \draw (\s/2,\s*\x) -- (\s*\LS+\s/2,\s*\x) (\s*\x, 0) -- (\s*\x, \s*\LS);
      \draw[dotted] (\s*\x + \s/2, 0) -- (\s*\x + \s/2, \s*\LS) (\s/2,\s*\x - \s/2) -- (\s*\LS+\s/2,\s*\x - \s/2);
      }
    \draw (\s/2,0) -- (\s*\LS+\s/2,0);
    \draw[dotted] (\s/2, 0) -- (\s/2, \s*\LS);
  \end{pgfonlayer}
  \begin{pgfonlayer}{qubits}
    \foreach \x in {1,...,\LS}
      \foreach \y in {1,...,\LS}{
         \node [qubit] (N-\x-\y-0) at (\s*\x + \s/2, \s*\y) {};
         \node [qubit] (N-\x-\y-1) at (\s*\x, \s*\y - \s/2) {};
         \coordinate (P-\x-\y) at (\s*\x + \s/2, \s*\y - \s/2);
         \coordinate (S-\x-\y) at (\s*\x, \s*\y);
         }
    \foreach \x in {0,...,\LS}{
      \node [qubit] (N-\x-0-0) at (\s*\x + \s/2, 0) {};
      \node [qubit] (N-0-\x-0) at (\s/2, \s*\x) {};
      \coordinate (P-0-\x) at (\s/2, \s*\x - \s/2);
      \coordinate (S-\x-0) at (\s*\x, 0);
    }
  \end{pgfonlayer}
}


\newcommand\DRAWERROR[4]{
  % Draws a toric lattice
  % var 0:    y coordinate
  % var 1:    x coordinate
  % var 2:    td coordinate
  % var 3:    error type x,y,z
  \def\x{#1}
  \def\y{#2}
  \begin{pgfonlayer}{qubits}    % select the background layer
    \ifstrequal{#4}{x}%
      {\ifnumequal{#3}{0}
        {\node [xer] (\y,\x,0) at (\s*\x + \s/2, \s*\y) {\tiny X};}
        {\node [xer] (\y,\x,0) at (\s*\x, \s*\y - \s/2) {\tiny X};}}
      {}
    \ifstrequal{#4}{z}%
      {\ifnumequal{#3}{0}
        {\node [zer] (\y,\x,0) at (\s*\x + \s/2, \s*\y) {\tiny Z};}
        {\node [zer] (\y,\x,0) at (\s*\x, \s*\y - \s/2) {\tiny Z};}}
      {}
    \ifstrequal{#4}{y}%
      {\ifnumequal{#3}{0}
        {\node [yer] (\y,\x,0) at (\s*\x + \s/2, \s*\y) {\tiny Y};}
        {\node [yer] (\y,\x,0) at (\s*\x, \s*\y - \s/2) {\tiny Y};}}
      {}
  \end{pgfonlayer}
}

\newcommand\DRAWPLAQ[2]{
  % Draws a plaquette
  % var 0:    y coordinate
  % var 1:    x coordinate
  \def\x{#1}
  \def\y{#2}
  \begin{pgfonlayer}{bg}
    \fill[plaq] (\x*\s,\y*\s) rectangle (\x*\s+\s,\y*\s-\s);
  \end{pgfonlayer}
}

\newcommand\DRAWEPLAQ[2]{
  % Draws a plaquette
  % var 1:    x coordinate
  % var 2:    y coordinate
  \def\x{#1}
  \def\y{#2}
  \begin{pgfonlayer}{bg}
    \ifnumequal{\x}{0}
      {\fill[plaq] (.5*\s,\y*\s) rectangle (\s,\y*\s-\s);}
      {\fill[plaq] (\x*\s,\y*\s) rectangle (\x*\s + .5*\s,\y*\s-\s);}
  \end{pgfonlayer}
}

\newcommand\DRAWSTAR[3]{
  % Draws a star
  % var 0:    y coordinate
  % var 1:    x coordinate
  % var 2:    lattice size
  \def\x{#1}
  \def\y{#2}
  \begin{pgfonlayer}{edges}
    \ifnumequal{\x}{0}
      {\draw[vert] (#3*\s-\s/2, \y*\s) -- (#3*\s, \y*\s);}
      {\draw[vert] (\x*\s-\s/2, \y*\s) -- (\x*\s, \y*\s);}
    \ifnumequal{\y}{\LS}
      {\draw[vert] (\x*\s, -\s) -- (\x*\s, -\s/2);}
      {\draw[vert] (\x*\s, \y*\s) -- (\x*\s, \y*\s +\s/2);}
    \draw[vert] (\x*\s, \y*\s - \s/2) -- (\x*\s, \y*\s) -- (\x*\s + \s/2, \y*\s);
  \end{pgfonlayer}
}


\newcommand\DRAWESTAR[2]{
  % Draws a star
  % var 1:    x coordinate
  % var 2:    y coordinate
  \def\x{#1}
  \def\y{#2}
  \begin{pgfonlayer}{edges}
    \ifnumequal{\y}{0}
      {\draw[vert] (\x*\s, 0) -- (\x*\s, \s/2);}
      {\draw[vert] (\x*\s, \y*\s) -- (\x*\s, \y*\s-\s/2);}
    \draw[vert] (\x*\s - \s/2, \y*\s) -- (\x*\s + \s/2, \y*\s);
  \end{pgfonlayer}
}



\newcommand\DSPECTRUM[3]{
  \tikzmath{
    \size = #1;
    \X = #2;
    \Y = #3;
    \y = #3-0.1;
  }
  \ifnumequal{\X}{0}{}{
    \path[fill=white!50!black, rounded corners=1pt] (0, 0.1) rectangle (\X,\y);
  }
  \draw[line width =0.5] (0,\Y) -- (0,0) -- (\size,0) -- (\size,\Y);

  \foreach \xx in {0,...,\size}{
    \draw[line width=0.5, font=\footnotesize] (\xx,0) -- +(0,-0.1) node[below] {\xx};
  }
}

\newcommand\DSPECTRA[6]{
  \tikzmath{
    \M = #1;
    \I = #2;
    \x = #3;
    \y = #4;
    \h = #5;
    \w = #6;
    \X = \x + \M;
  }
  \begin{scope}[shift={(\x, 0)}]
  \begin{scope}[xscale=\w]
  
  \ifnumequal{\I}{0}{}{
    \draw[line width=0, fill=white!50!black, rounded corners=2pt] (0,\y) rectangle +(\I,\h);
  }
  \draw[thin, rounded corners = 2pt] (0,\y) rectangle +(\M,\h);
  \foreach [count=\i] \xx in {0,...,\M}{
    \draw[font=\tiny] (\xx,\y) node[below] {\xx};
  }
  \end{scope}
  \end{scope}
}

    \def\QS{15}
    \def\s{1.5}
    \def\lw{0.5}


    \begin{tikzpicture}
        \DRAWTORIC{3}
    \end{tikzpicture}
    \begin{tikzpicture}
        \DRAWTORIC{4}
        \DRAWERROR{1}{2}{0}{z}
        \DRAWERROR{2}{0}{0}{z}
        \DRAWERROR{1}{0}{0}{z}
        \DRAWSTAR{1}{2}{5}
        \DRAWSTAR{2}{2}{5}
        \DRAWSTAR{1}{0}{5}
        \DRAWSTAR{3}{0}{5}
        \begin{pgfonlayer}{edges}
            \draw[synz] (N-1-0-0) -- (N-2-0-0);
        \end{pgfonlayer}
    \end{tikzpicture}
    \newline
    \begin{tikzpicture}
        \DRAWTORIC{4}
        \DRAWSTAR{1}{2}{5}
        \DRAWSTAR{2}{2}{5}
        \DRAWSTAR{1}{0}{5}
        \DRAWSTAR{3}{0}{5}
    \end{tikzpicture}

    \tikzstyle{rednode}=[circle, fill=red!50, minimum size=4]
    \tikzstyle{bluenode}=[circle, fill=cyan!50, minimum size=4]
    \tikzstyle{redline}=[red!50, line width = 2]
    \tikzstyle{blueline}=[cyan!50, line width = 2]
    \tikzstyle{legend}=[anchor=west, font=\small]

    \newcommand{\drawquasigrid}{
    \draw[step=.4cm, opacity=.25] (0,-.8) grid (4,3.2);
    \draw (0,-.8) rectangle (4,3.2);
    \node[rednode] (N1) at (0.5,0.4) {};
    \node[rednode] (N2) at (2,0.7) {};
    \node[rednode] (N3) at (2.5,1.2) {};
    \node[rednode] (N4) at (3.6,1) {};
    \node[rednode] (N5) at (0.75,2.1) {};
    \node[rednode] (N6) at (1.95,1.8) {};
    \draw[blueline] (N1) to[in=170, out=20] (N2);
    \draw[blueline] (N3) to[in=180, out=-10] (N4);
    \draw[blueline] (N5) to[in=170, out=-20] (N6);
    }
    \begin{figure}[htbp]
        \centering
        \begin{tikzpicture}[scale=0.9]
            \drawquasigrid
        \end{tikzpicture}
        \hspace{.3cm}
        \begin{tikzpicture}[scale=0.9]
            \drawquasigrid
            \draw[dashed, blueline] (N1) to[out=90, in=270] (N5);
            \draw[dashed, blueline] (N2) to[out=80, in=275] (N6);
            \draw[dashed, blueline] (N3) to[out=10, in=160] (N4);
        \end{tikzpicture}
        \hspace{.3cm}
        \begin{tikzpicture}[scale=0.9]
        \drawquasigrid
            \draw[yellow, line width=6, opacity=.3] (N3) to[in=180, out=-10] (N4);
            \draw[yellow, line width=6, opacity=.3] (N5) to[in=170, out=-20] (N6);
            \draw[yellow, line width=6, opacity=.3] (N3) to[out=120, in=-30] (N6);
            \draw[yellow, line width=6, opacity=.3] (N5) to[out=200, in=45] (0, 1.75);
            \draw[yellow, line width=6, opacity=.3] (N4) to[out=80, in=225] (4, 1.75);
            \draw[dashed, blueline] (N1) to[out=0, in=200] (N2);
            \draw[dashed, blueline] (N3) to[out=120, in=-30] (N6);
            \draw[dashed, blueline] (N5) to[out=200, in=45] (0, 1.75);
            \draw[dashed, blueline] (N4) to[out=80, in=225] (4, 1.75);
        \end{tikzpicture}

    \end{figure}


    \def\QS{4}
    \def\s{1.5}
    \def\lw{0.5}
    \begin{tikzpicture}[scale=0.5]
        \DRAWTORIC{8}
    \end{tikzpicture}

    \def\QS{0.15}
    \def\s{1.5}
    \def\lw{0.5}
    \begin{tikzpicture}[scale=0.4]
        \DRAWTORIC{24}
    \end{tikzpicture}

\end{document}