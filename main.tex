\documentclass[11pt, a4paper, twoside, titlepage, dvipsnames]{report}
\usepackage[page, title, titletoc]{appendix}
\usepackage[a4paper, total={6in, 8in}]{geometry}
\usepackage[section]{placeins}
\usepackage[utf8]{inputenc}
\usepackage[hidelinks]{hyperref}
\usepackage{physics}
\usepackage{adjustbox}
\usepackage{amsfonts, amsmath, amssymb, amsthm}
\usepackage[mathscr]{euscript}

\let\oldemptyset\emptyset
\let\emptyset\varnothing

\setlength{\parindent}{24pt}
\newtheorem{definition}{Definition}[chapter]
\newtheorem{lemma}{Lemma}[chapter]
\newtheorem{theorem}{Theorem}[chapter]
\newtheorem{proposition}{Proposition}[chapter]

% Set graphics folder
\usepackage{float}
\usepackage{graphicx}
\usepackage[format=plain, font=it]{caption}
\usepackage[format=plain, font=it]{subcaption}
\graphicspath{{"/home/watermarkhu/links/onedriveuni/MEP - thesis Mark/Figures/ch_toric_code/"}}

% Set algorithm2e package settings
\usepackage[linesnumbered, vlined]{algorithm2e}
\SetStartEndCondition{ }{}{}
\SetKwProg{Fn}{def}{\string:}{}
\SetKw{KwTo}{in}
\SetKwFor{For}{for}{\string:}{}
\SetKwIF{If}{ElseIf}{Else}{if}{ then}{elif}{else:}{}
\SetKwFor{While}{while}{ do}{}
\SetAlgoNoEnd\DontPrintSemicolon

% Set algorithm title custom box
\usepackage[most]{tcolorbox}
\tcbset{algotitle/.style={title={ \strut Algorithm~\thetcbcounter\ifstrempty{#1}{\ignorespaces}{:~#1}}}}
\newtcolorbox[auto counter]{algo}[1][]{
    colback=white, colframe=black, boxrule=0.5pt,
    titlerule=0pt, sharp corners, colbacktitle=white,enhanced,
    attach boxed title to top center={yshift=-10pt},
    boxed title style={boxrule=-1pt},
    fonttitle=\bfseries, coltitle=black, algotitle={}, #1}

\usepackage{xcolor, listings, xparse, todonotes}

\newcommand{\codeword}[1]{\texttt{\textcolor{MidnightBlue}{#1}}}
%\newcommand{\codefunc}[1]{\texttt{\textcolor{OliveGreen}{#1}}}
\newcommand{\codefunc}[1]{\texttt{#1}}
\newcommand{\m}[1]{\mathcal{#1}}
\newcommand{\n}[1]{\mathscr{#1}}
\newcommand{\bound}{\mathscr{B}}
\newcommand{\nset}{\mathcal{N}}
\newcommand{\vset}{\mathcal{V}}
\newcommand{\pre}[1]{ {}^{#1} }
\newcommand{\ceil}[1]{{\left \lceil #1 \right \rceil }}
\newcommand{\floor}[1]{{\left \lfloor #1 \right \rfloor }}

\lstset{language=C,keywordstyle={\bfseries \color{blue}}}
% Set quantikz settings
\usepackage{tikz}
\usetikzlibrary{quantikz,shapes,calc,math,patterns}
\def\checkmark{\tikz\fill[scale=0.4](0,.35) -- (.25,0) -- (1,.7) -- (.25,.15) -- cycle;}
\SetAlFnt{\footnotesize}
\tikzset{bellstar/.style ={draw, star,star points = 8, star point height = 1em, star point ratio = 1.8, minimum size = 1em, inner sep = 0pt}}
\tikzset{selection/.style={draw, signal, signal to =south, minimum size = 1em, inner sep = 0pt}}
\tikzset{sel/.style={label position = below, yshift = 0.5cm}}


\usepackage[symbols,nomain,nogroupskip,automake]{glossaries-extra}
\makeglossaries
\glsxtrnewsymbol[description={Pauli group}]{pauligroup}{\ensuremath{\m{P}}}
\glsxtrnewsymbol[description={Pauli X operator}]{X}{\ensuremath{X}}
\glsxtrnewsymbol[description={Pauli Y operator}]{Y}{\ensuremath{Y}}
\glsxtrnewsymbol[description={Pauli Z operator}]{Z}{\ensuremath{Z}}
\glsxtrnewsymbol[description={Identity operator}]{I}{\ensuremath{I}}
\glsxtrnewsymbol[description={Logical X operator}]{Xlogical}{\ensuremath{\bar{X}}}
\glsxtrnewsymbol[description={Logical Z operator}]{Zlogical}{\ensuremath{\bar{Z}}}
\glsxtrnewsymbol[description={Stabilizer group}]{stabilizergroup}{\ensuremath{\m{S}}}
\glsxtrnewsymbol[description={Stabilizer, element of $\m{S}$}]{stabilizer}{\ensuremath{S}}
\glsxtrnewsymbol[description={Stabilizer generator, generates $\m{S}$}]{stabilizergenerator}{\ensuremath{s}}
\glsxtrnewsymbol[description={Plaquette operator}]{plaquette}{\ensuremath{P_f}}
\glsxtrnewsymbol[description={Star operator}]{star}{\ensuremath{S_v}}
\glsxtrnewsymbol[description={Graph of vertices, edges (and faces)}]{graph}{\ensuremath{G}}
\glsxtrnewsymbol[description={Vertex set}]{vertices}{\ensuremath{V}}
\glsxtrnewsymbol[description={vertex, elemet of $V$}]{vertex}{\ensuremath{v}}
\glsxtrnewsymbol[description={Edge set}]{edges}{\ensuremath{E}}
\glsxtrnewsymbol[description={edge, element of $E$}]{edge}{\ensuremath{e}}
\glsxtrnewsymbol[description={Face set}]{faces}{\ensuremath{F}}
\glsxtrnewsymbol[description={Face, element of $F$}]{face}{\ensuremath{f}}
\glsxtrnewsymbol[description={Probability of succesfull error correction}]{pcorrect}{\ensuremath{p_C}}
\glsxtrnewsymbol[description={Threshold probability}]{pthres}{\ensuremath{p_{th}}}
\glsxtrnewsymbol[description={Error operator, product of Pauli operators}]{perror}{\ensuremath{P}}
\glsxtrnewsymbol[description={Correction operator, product of Pauli operators}]{correction}{\ensuremath{C}}
\glsxtrnewsymbol[description={Erasure, subset of $E$}]{erasure}{\ensuremath{\m{E}}}
\glsxtrnewsymbol[description={Forest of erasure $\m{E}$}]{forest}{\ensuremath{F_\m{E}}}
\glsxtrnewsymbol[description={Pauli product of set of edges}]{pauliproduct}{\ensuremath{\mathscr{P}}}
\glsxtrnewsymbol[description={Correction set, subset of $E$}]{correctionset}{\ensuremath{\m{C}}}
\glsxtrnewsymbol[description={Boundary of an edge set}]{boundary}{\ensuremath{\mathscr{B}}}




\usepackage[style=numeric-comp, sorting=none]{biblatex}
\addbibresource{cit.bib}

\title{mep-thesis}
\author{Shui Hu}
\date{April 2019}


\begin{document}

\begin{titlepage}
	\newcommand{\HRule}{\rule{\linewidth}{0.3mm}}
	\center

	\textsc{\Large Delft University of Technology}\\[1.5cm]

	\textsc{\large Master's Thesis}\\[0.5cm]

    \vfill

	\HRule\\[1cm]

	{\huge\bfseries Quantum Error Correction of Patch-Loss of the Surface Code}\\[0.4cm]

	\HRule\\[1.8cm]

	\begin{minipage}{0.4\textwidth}
		\begin{flushleft}
			\large
			\textit{Author}\\
			S. \textsc{Hu}
		\end{flushleft}
	\end{minipage}
	~
	\begin{minipage}{0.4\textwidth}
		\begin{flushright}
			\large
			\textit{Supervisor}\\
			D. \textsc{Elkouss} % Supervisor's name
		\end{flushright}
	\end{minipage}


	%------------------------------------------------
	%	Date
	%------------------------------------------------

	\vfill\vfill\vfill % Position the date 3/4 down the remaining page

	{\large\today} % Date, change the \today to a set date if you want to
	\vfill % Push the date up 1/4 of the remaining page

\end{titlepage}

\tableofcontents
\printglossary[type=symbols,style=long]


\def\QS{15}
\def\s{1.5}
\def\lw{0.4}

\tikzset{
  qubit/.style={line width=\lw,circle,draw,fill=white,minimum size=\QS},
  bound/.style={line width=\lw,circle, fill=gray!50, minimum size=\QS},
  line/.style={line width=\lw},
  xer/.style={line width=\lw,circle,draw,fill=red!50,inner sep=0,text width=\QS,align=center},
  zer/.style={line width=\lw,circle,draw,fill=cyan!50,inner sep=0,text width=\QS,align=center},
  yer/.style={line width=\lw,circle,draw,fill=orange!50,inner sep=0,text width=\QS,align=center},
  plaq/.style={fill=cyan!20},
  vert/.style={red!50, line width=4*\lw},
  synz/.style={cyan!50, line width=4*\lw, line cap=round},
  synx/.style={purple!25, line width=4*\lw, line cap=round},
  empty/.style={inner sep=0},
  arrow/.style={->, line width=2*\lw},
  script/.style={rectangle, fill=white, opacity=.5, text opacity=1, inner sep=0}
  }

\pgfdeclarelayer{bg}
\pgfdeclarelayer{edges}
\pgfdeclarelayer{qubits}
\pgfsetlayers{bg,edges,qubits,main}

\newcommand{\MINUS}[1]{%
  \number\numexpr#1-1\relax%
}


\newcommand\DRAWTORIC[1]{
  % Draws a toric lattice
  % var 0:    lattice size

  \def\LS{\MINUS{#1}}
  \begin{pgfonlayer}{edges}
    \foreach \x in {0,...,\LS}{
      \draw[line width=\lw] (0,\s*\x) -- (\s*\LS+\s,\s*\x) (\s*\x, -\s) -- (\s*\x, \s*\LS);
      \draw[line width=\lw, dotted] (0,\s*\x - \s/2) -- (\s*\LS+\s,\s*\x - \s/2) (\s*\x + \s/2, -\s) -- (\s*\x + \s/2, \s*\LS);
      }
    \draw[line width=\lw, dashed] (0, -\s) -- (\s*\LS+\s, -\s) -- (\s*\LS+\s, \s*\LS);
  \end{pgfonlayer}
  \begin{pgfonlayer}{qubits}
    \foreach \x in {0,...,\LS}
      \foreach \y in {0,...,\LS}{
         \node [qubit] (N-\x-\y-0) at (\s*\x + \s/2, \s*\y) {};
         \node [qubit] (N-\x-\y-1) at (\s*\x, \s*\y - \s/2) {};
         \coordinate (P-\x-\y) at (\s*\x + \s/2, \s*\y - \s/2);
         \coordinate (S-\x-\y) at (\s*\x, \s*\y);
         }
    \foreach \x in {0,...,\LS}{
      \node [bound] (By-\x) at (\s*\LS+\s, \s*\x - \s/2) {};
      \node [bound] (Bx-\x) at (\s*\x + \s/2, -\s) {};
      \coordinate (S-\x-#1) at (\s*\x, -\s);
      \coordinate (S-#1-\x) at (\s*#1, \s*\x);
    }

  \end{pgfonlayer}
}

\newcommand\DRAWPLANAR[1]{
  % Draws a planar lattice#1
  % var 0:    lattice size

  \def\LS{\MINUS{#1}}
  \begin{pgfonlayer}{edges}
    \foreach \x in {1,...,\LS}{
      \draw (\s/2,\s*\x) -- (\s*\LS+\s/2,\s*\x) (\s*\x, 0) -- (\s*\x, \s*\LS);
      \draw[dotted] (\s*\x + \s/2, 0) -- (\s*\x + \s/2, \s*\LS) (\s/2,\s*\x - \s/2) -- (\s*\LS+\s/2,\s*\x - \s/2);
      }
    \draw (\s/2,0) -- (\s*\LS+\s/2,0);
    \draw[dotted] (\s/2, 0) -- (\s/2, \s*\LS);
  \end{pgfonlayer}
  \begin{pgfonlayer}{qubits}
    \foreach \x in {1,...,\LS}
      \foreach \y in {1,...,\LS}{
         \node [qubit] (N-\x-\y-0) at (\s*\x + \s/2, \s*\y) {};
         \node [qubit] (N-\x-\y-1) at (\s*\x, \s*\y - \s/2) {};
         \coordinate (P-\x-\y) at (\s*\x + \s/2, \s*\y - \s/2);
         \coordinate (S-\x-\y) at (\s*\x, \s*\y);
         }
    \foreach \x in {0,...,\LS}{
      \node [qubit] (N-\x-0-0) at (\s*\x + \s/2, 0) {};
      \node [qubit] (N-0-\x-0) at (\s/2, \s*\x) {};
      \coordinate (P-0-\x) at (\s/2, \s*\x - \s/2);
      \coordinate (S-\x-0) at (\s*\x, 0);
    }
  \end{pgfonlayer}
}


\newcommand\DRAWERROR[4]{
  % Draws a toric lattice
  % var 0:    y coordinate
  % var 1:    x coordinate
  % var 2:    td coordinate
  % var 3:    error type x,y,z
  \def\x{#1}
  \def\y{#2}
  \begin{pgfonlayer}{qubits}    % select the background layer
    \ifstrequal{#4}{x}%
      {\ifnumequal{#3}{0}
        {\node [xer] (\y,\x,0) at (\s*\x + \s/2, \s*\y) {\tiny X};}
        {\node [xer] (\y,\x,0) at (\s*\x, \s*\y - \s/2) {\tiny X};}}
      {}
    \ifstrequal{#4}{z}%
      {\ifnumequal{#3}{0}
        {\node [zer] (\y,\x,0) at (\s*\x + \s/2, \s*\y) {\tiny Z};}
        {\node [zer] (\y,\x,0) at (\s*\x, \s*\y - \s/2) {\tiny Z};}}
      {}
    \ifstrequal{#4}{y}%
      {\ifnumequal{#3}{0}
        {\node [yer] (\y,\x,0) at (\s*\x + \s/2, \s*\y) {\tiny Y};}
        {\node [yer] (\y,\x,0) at (\s*\x, \s*\y - \s/2) {\tiny Y};}}
      {}
  \end{pgfonlayer}
}

\newcommand\DRAWPLAQ[2]{
  % Draws a plaquette
  % var 0:    y coordinate
  % var 1:    x coordinate
  \def\x{#1}
  \def\y{#2}
  \begin{pgfonlayer}{bg}
    \fill[plaq] (\x*\s,\y*\s) rectangle (\x*\s+\s,\y*\s-\s);
  \end{pgfonlayer}
}

\newcommand\DRAWEPLAQ[2]{
  % Draws a plaquette
  % var 1:    x coordinate
  % var 2:    y coordinate
  \def\x{#1}
  \def\y{#2}
  \begin{pgfonlayer}{bg}
    \ifnumequal{\x}{0}
      {\fill[plaq] (.5*\s,\y*\s) rectangle (\s,\y*\s-\s);}
      {\fill[plaq] (\x*\s,\y*\s) rectangle (\x*\s + .5*\s,\y*\s-\s);}
  \end{pgfonlayer}
}

\newcommand\DRAWSTAR[3]{
  % Draws a star
  % var 0:    y coordinate
  % var 1:    x coordinate
  % var 2:    lattice size
  \def\x{#1}
  \def\y{#2}
  \begin{pgfonlayer}{edges}
    \ifnumequal{\x}{0}
      {\draw[vert] (#3*\s-\s/2, \y*\s) -- (#3*\s, \y*\s);}
      {\draw[vert] (\x*\s-\s/2, \y*\s) -- (\x*\s, \y*\s);}
    \ifnumequal{\y}{\LS}
      {\draw[vert] (\x*\s, -\s) -- (\x*\s, -\s/2);}
      {\draw[vert] (\x*\s, \y*\s) -- (\x*\s, \y*\s +\s/2);}
    \draw[vert] (\x*\s, \y*\s - \s/2) -- (\x*\s, \y*\s) -- (\x*\s + \s/2, \y*\s);
  \end{pgfonlayer}
}


\newcommand\DRAWESTAR[2]{
  % Draws a star
  % var 1:    x coordinate
  % var 2:    y coordinate
  \def\x{#1}
  \def\y{#2}
  \begin{pgfonlayer}{edges}
    \ifnumequal{\y}{0}
      {\draw[vert] (\x*\s, 0) -- (\x*\s, \s/2);}
      {\draw[vert] (\x*\s, \y*\s) -- (\x*\s, \y*\s-\s/2);}
    \draw[vert] (\x*\s - \s/2, \y*\s) -- (\x*\s + \s/2, \y*\s);
  \end{pgfonlayer}
}



\newcommand\DSPECTRUM[3]{
  \tikzmath{
    \size = #1;
    \X = #2;
    \Y = #3;
    \y = #3-0.1;
  }
  \ifnumequal{\X}{0}{}{
    \path[fill=white!50!black, rounded corners=1pt] (0, 0.1) rectangle (\X,\y);
  }
  \draw[line width =0.5] (0,\Y) -- (0,0) -- (\size,0) -- (\size,\Y);

  \foreach \xx in {0,...,\size}{
    \draw[line width=0.5, font=\footnotesize] (\xx,0) -- +(0,-0.1) node[below] {\xx};
  }
}

\newcommand\DSPECTRA[6]{
  \tikzmath{
    \M = #1;
    \I = #2;
    \x = #3;
    \y = #4;
    \h = #5;
    \w = #6;
    \X = \x + \M;
  }
  \begin{scope}[shift={(\x, 0)}]
  \begin{scope}[xscale=\w]
  
  \ifnumequal{\I}{0}{}{
    \draw[line width=0, fill=white!50!black, rounded corners=2pt] (0,\y) rectangle +(\I,\h);
  }
  \draw[thin, rounded corners = 2pt] (0,\y) rectangle +(\M,\h);
  \foreach [count=\i] \xx in {0,...,\M}{
    \draw[font=\tiny] (\xx,\y) node[below] {\xx};
  }
  \end{scope}
  \end{scope}
}

% \chapter{Preliminary report}

\subsection*{Introduction}

Quantum computing has the potential to transcend the information technology as we know it. Small scale quantum systems are already possible today and the goal is to scale up these quantum architectures to build practical quantum devices. One approach to do this is to by networking many simple processor cells together through quantum links, avoiding the necessity to build a single complex structure. Processor cells that are located physically close to each other are connected by "short" links and lie in a patch. Patches that are located physically far from each other can in turn be connected by "long" links, such as remote optical connections. The total state of system, which contains the stored information, is shared across these patches, such that it can be accessed in either one of these patches. \\

This is somewhat analogous to the idea of a shared database. Many online services that we use today rely on servers that host the data that we want to view, store or edit. This data is often not stored on a single server, but copied to many others, in a shared database. In case one of these servers goes offline due to file corruption or an electricity outrage, the data is not lost, and can still be accessed on another server in the cluster.\\

In our Quantum network, information cannot be copied across different processor cells due to the no cloning theorem. In stead, it is shared across cells through entanglement. A cell can also go "offline", when a qubit or multiple qubits are lost from the system due to some interaction with the environment. This process is called decoherence, also described with \emph{loss} or \emph{erasure}. Luckily, if the losses are not too much, these cells can be restored through quantum error correction (QEC) such that the quantum state or encoded information can still be extracted from the system. \\

\subsection*{Quantum errors}
Errors that can occur during Quantum computation can generally be classified as 1) noise, in which there is an error are within the computational basis, or as 2) a loss, in which the qubit is
taken out of the computational basis. Losses are both detectable and locatable, which means that a higher rate of loss ($p_{loss}$) can be tolerated than noise or computational errors ($p_{com}$). The process of finding and correcting these errors is called decoding. \\

Kitaev's surfaces codes are defined by a set of stabilizers which act on a set of physical qubits that lie on the edges of a square lattice \cite{kitaev}. The stabilizers commute, and are generated by plaquettes (group of $Z$ operators), or by stars (group of $X$ operators). Logical operates corresponds to a set of stabilizer operators along a homologically nontrivial cycle. Any homologically equivalent set of operators can be used to measure the physical qubit operator. Therefore, in the case of a qubit loss, another set of operators can be used, if there is no \emph{percolated} region of losses that span the entire lattice. \\

To decode for computational errors, one measures the stabilizer generators, which returns eigenvalue -1 on the edges of the error chains or syndromes, and can be corrected by finding a nontrivial closing chain, which either equals a stabilizer measurement that corrects the error, or a logical operator which equals a logical error. This problem is equivalent to the two dimensional random-bond Ising model (RBIM), where the shortest path needs to be found between matching pairs. The closing chain is found using the Edmonds' minimum weight perfect matching (MWPM) algorithm. This algorithm scales quadratic in time as the lattice size increases \cite{stace2009}. More recently, an almost-linear decoding approach has been described by Delfosse et al. \cite{nickerson2017}. \\

There are also multiple methods to decode for losses on the surface code. Stace et. al \cite{stace2009,stace2010} describes the method of so-called superplaquettes and superstars, in which the lost qubit is accounted for by combining neighboring stabilizers. The resulting lattice can be than decoded using the same MWPM algorithm. Delfosse et al \cite{delfosse2017} describes a linear-time maximum likelihood method to decode for losses. Here, the errors are found in a \emph{peeling} algorithm that iteratively peels branches away from a tree of possible error chains until the lost qubits remain.

\subsection*{Patch loss}
In terms of the surface code, a patch is a set of neighboring qubit cells on the lattice that lie physically close to each other. The entire lattice is built by connecting multiple patches. It is possible that cells within the same patch suffer the same decoherence, due to their physical vicinity, such that entire patches may be lost. The Quantum links that connect these patches, in turn, may suffer a larger amount of noise due to the distance it must cross. \\

These patch losses may be investigated using the decoding algorithms described above. There are two main question to be answered here. The first one involves the size of the patches. What are tolerable sizes of patches such that in the event of a patch loss, the encoded Quantum information can still be extracted from the system? The second question concerns the Quantum links between these patches. What effects does the patch model on the surface have on the tolerable noise thresholds, especially on the thresholds of the Quantum links that connect different patches?

\subsection*{Project}
This project will focus on the two problems of the patch model described in the previous section. We will use simulations to calculate for the thresholds on patch sizes and noise. Most probably, we will try to build this simulation on top of the fault-tolerance simulations by N. Nickerson \cite{naomi}, whose thesis will also be used as our foundation on quantum error correction. Furthermore, the introduction on topological codes will be provided by the lecture notes of D. Browne \cite{browne}. We we use the knowledge acquired in the courses Advanced Quantum Mechanics (AP3051G), Computational Physics (AP3082D), as well as the EDX courses The Building Blocks of a Quantum Computer (DelftX - QTM2x, QTM3x) and Quantum Cryptography (CaltechDelftX - QuCryptox), which are similar to the TU Delft courses AP3421 and CS4090. \\

The project will be done under supervision of David Elkouss, who leads the Elkouss group, which is part of the Roadmap Quantum Internet and Networked Computing at QuTech. Weekly meetings are planned for discussions and evaluations. A presentation for the group will be held after a month. Furthermore, we also expect to be working closely together with Sebiastian de Bone, who is a Phd student at the Elkouss Group.\\

\noindent
The project can be divided into 5 successive objectives:
\begin{enumerate}\label{test}
  \item Theory and prediction of model with patch-loss of the surface code.
  \item Calculation of threshold on patch sizes.
  \item Definitions of noise between patches.
  \item Combine patch-loss and noise between patches.
  \item Improve decoder.
\end{enumerate}
The preparation for the master thesis ends on April 12, 2019. From here, the first 3 months of the project, until July 2019, we expect to be working on objective 1, expanding our theoretical knowledge of the Quantum erasure code and formulating definitions of the patch model. Ideally, we will also have some simulations results for objective 2 by the end of July. We will take a short break in August, whereafter we will continue on objectives 3 and 4. Objective 5, improving the decoder, is an optional objective. The mechanism of the decoders are so complicated, that an improvement may be a project on its own. However, due to my particular interest in these decoders, I will be alert on trying to find a point of improvement if possible. The timeline of this project will look as such.
\begin{figure}[h]
  \centering
  \includegraphics[width=\linewidth]{fig/Timeline.png}
\end{figure}

If the simulation of the patch sizes (2) will be so complicated such that after it cannot be completed before the end of July, we will drop objectives 3 and 4 and focus on the first two objectives. The deadline for the thesis is set to the end of November, such that a defense may be held in December.


%In the case that $p_{com} = 0$, the threshold for logical qubit recovery is $p_{loss} < 0.5$, which is equivalent to the \emph{bond percolation threshold}. In the case that $p_{loss} = 0$, computational errors can be found by measuring the stabilizer generators, which returns eigenvalue -1 on the edges of the error chains or syndromes, and can be corrected by finding a nontrivial closing chain, which either equals a stabilizer measurement that corrects the error, or a logical operator which equals a logical error. This problem is equivalent to the two dimensional random-bond Ising model (RBIM), where the shortest path needs to be found between matching pairs. The closing chain is found using the Edmonds' minimum weight perfect matching (MWPM) algorithm. In this case, the threshold is $p_{com} < 0.104$, for which a larger lattice size will decrease the chance of a logical error. These two thresholds corresponds to two end points of a boundary of correctability: $(p_{loss}, p_{comp}) = (0.5,0)$ and $(0,0.104)$.\\

%\subsubsection*{Superstabilizer decoder\cite{stace2009,stace2010}}
%To correct for losses on the surface code, new stabilizer generators are formed by connecting neighboring plaquettes or stars, which are connected to the same lost qubit, forming so-called superplaquettes and superstars respectively. These superstabilizers may share multiple qubits, and a nontrivial error chain arises of there are an odd number of (computational) errors in these shared qubits. This can be avoided by degrading the edges (shared qubits) between the superstabilizers to a single superedge whose error rates depend on the number of physical qubits shared, accounting for this degeneracy. For the sake of computation it is easier to stay on a square lattice. Take each stabilizer as a node and each shared qubit as an edge. The superstabilizer approach can be achieved by setting the weight of the edge of a lost qubit to 0, and the weight of shared edges between superstabilizers to the weight of the superedge. \\

%Using the superstabilizer approach, Edmonds' MWPM algorithm can be applied with altered edge weights. Simulations using the superstabilizers scheme with varying $p_{loss}$ show that the computational error threshold $p_{comp}^{thr}$ stay within the boundary of correctability, following the universal scaling law. At the limit of small $p_{loss}$, superstabilizer mostly consist of only 6 qubits, and increasing $p_{loss}$ only contributes to the number of superstabilizers, resulting in a linear relationship of threshold $p_{comp}^{thr}$ with $p_{loss}$.   As $p_{loss}$ increases further, larger superstabilizers containing more qubits appear, which have an increased chance of syndrome error. At $p_{loss} > 0.425$, the universal scaling law breaks down due to the size of the superstabilizers, as finite lattice effects dominate.\\

%One must also take into account that some error matchings may have a higher path degeneracy than others, indicating that the former is more likely. Therefore, the weights of the edges must additionally account for this degeneracy. However, due to this notion it possible that the algorithm may favor a matching that is not necessarily the shortest path. To balance these factors an additional factor $\tau$ is introduced that specifies the importance of path degeneracy. An optimal value for $\tau$ is found for which the computational error threshold is increased to $p_{comp}^{thr} = 0.1065$. Various implementations of Edmonds' MWPM algorithms are tested not to account for this degeneracy, and therefore a higher threshold may be possible with a computationally efficient algorithms that takes this into account.

%\subsubsection*{Maximum likelihood decoder\cite{delfosse2017}}
%Another way to look at erasure, is that the lost qubit is replaced by a maximally mixed state, which can also be interpreted as a qubit suffering a Pauli error $I,X,Y,Z$ chosen at random. Just as before, the decoding is done by measuring the plaquettes and stars, except now there is the additional knowledge of the erasure pattern, in which the errors must occur.\\

%For a set of errors $\sigma$ in the erasure pattern $\varepsilon$, either $X$ or $Z$ errors ($Y$ is a combination of the two), the algorithm is to make a spanning forest $F_{\varepsilon}$ inside of $\varepsilon$, a maximal subset of edges of $\varepsilon$ that contains no cycles. From this tree, boundary edges called leafs are iteratively stripped from $F_{\varepsilon}$. If the single connected vertex of the leave is in $\sigma$, the leaf or edge is added to the correction chain, and the other vertex is flipped in $\sigma$. What remains after the stripping $F_{\varepsilon}$ is the correction chain. On surface codes with boundaries, additional constrains are applied to the method of how the spanning forest is grown. \\

%\subsubsection*{Comparison}
%The benefit of the maximum likelihood decoder is that it scales linear with the size of the surface code. The spanning forest can be grown in linear-time, and the stripping process also only passes each leaf once, ensuring linear complexity. The superstabilizer decoder applies Edmonds' algorithm, which scales quadratic in time, but does however solve for erasure and computation errors simultaneously, whereas the maximum likelihood decoder only solves for erasure errors. Therefore, the optimal decoder probably depends on the size of the system.

% \chapter{Introduction}


Quantum computing has the potential to transcend the information technology as we know it. Small scale quantum systems are already possible today and the goal is to scale up these quantum architectures to build practical quantum devices. One approach to do this is to by networking many simple processor cells together through quantum links, avoiding the necessity to build a single complex structure. Processor cells that are located physically close to each other are connected by ``short'' links and lie in a patch. Patches that are located physically far from each other can in turn be connected by ``long'' links, such as remote optical connections. The total state of system, which contains the stored information, is shared across these patches, such that it can be accessed in either one of these patches. \\

This is somewhat analogous to the idea of a shared database. Many online services that we use today rely on servers that host the data that we want to view, store or edit. This data is often not stored on a single server, but copied to many others, in a shared database. In case one of these servers goes offline due to file corruption or an electricity outrage, the data is not lost, and can still be accessed on another server in the cluster.\\

In our Quantum network, information cannot be copied across different processor cells due to the no cloning theorem. In stead, it is shared across cells through entanglement. A cell can also go ``offline", when a qubit or multiple qubits are lost from the system due to some interaction with the environment. This process is called decoherence, also described with \emph{loss} or \emph{erasure}. Luckily, if the losses are not too much, these cells can be restored through quantum error correction (QEC) such that the quantum state or encoded information can still be extracted from the system. \\

\subsection{Quantum errors}
Errors that can occur during Quantum computation can generally be classified as 1) noise, in which there is an error are within the computational basis, or as 2) a loss, in which the qubit is
taken out of the computational basis. Losses are both detectable and locatable, which means that a higher rate of loss ($p_{loss}$) can be tolerated than noise or computational errors ($p_{com}$). The process of finding and correcting these errors is called decoding. \\

Kitaev's surfaces codes are defined by a set of stabilizers which act on a set of physical qubits that lie on the edges of a square lattice \cite{dennis2002topological}. The stabilizers commute, and are generated by plaquettes (group of $Z$ operators), or by stars (group of $X$ operators). Logical operates corresponds to a set of stabilizer operators along a homologically nontrivial cycle. Any homologically equivalent set of operators can be used to measure the physical qubit operator. Therefore, in the case of a qubit loss, another set of operators can be used, if there is no \emph{percolated} region of losses that span the entire lattice. \\

To decode for computational errors, one measures the stabilizer generators, which returns eigenvalue -1 on the edges of the error chains or syndromes, and can be corrected by finding a nontrivial closing chain, which either equals a stabilizer measurement that corrects the error, or a logical operator which equals a logical error. This problem is equivalent to the two dimensional random-bond Ising model (RBIM), where the shortest path needs to be found between matching pairs. The closing chain is found using the Edmonds' minimum weight perfect matching (MWPM) algorithm. This algorithm scales quadratic in time as the lattice size increases \cite{stace2009thresholds}. More recently, an almost-linear decoding approach has been described by Delfosse et al. \cite{delfosse2017linear}. \\

There are also multiple methods to decode for losses on the surface code. Stace et. al \cite{stace2009thresholds,stace2010error} describes the method of so-called superplaquettes and superstars, in which the lost qubit is accounted for by combining neighboring stabilizers. The resulting lattice can be than decoded using the same MWPM algorithm. Delfosse et al \cite{delfosse2017linear} describes a linear-time maximum likelihood method to decode for losses. Here, the errors are found in a \emph{peeling} algorithm that iteratively peels branches away from a tree of possible error chains until the lost qubits remain.

The Union-Find decoder is preferred over other types of decoders because it is \emph{simple}. Even though it may not seem so due to the length of its chapter, the concept of the Union-Find decoder is much more straightforward compared with other, more advanced decoders. 


Chapters \ref{ch:qec} up until Chapter \ref{ch:UFdecoder} Section \ref{sec:bucketwg} are descriptions of existing and known material. Section \ref{sec:bucketwg} and onwards includes exclusively our own contributions. 

\chapter{Quantum error correction}

To build a real world quantum computer, or a quantum communications device, one has to deal with the presence of noise, which will inevitably alter the quantum state of a qubit stored or passed through a communications channel. Recent developments have raised the fidelity of single qubit operations to up to one single failure in $10^6$ operations \cite{ballance2016high}. But even this fidelity is not enough, as a full quantum computation may require millions of qubits, and the generation of entangled states over a large number of qubits. With imperfect quantum gates, anything we do in order to perform a computation will add to the error. 

The theory of \emph{quantum error correction} has been developed to counteract this noise, by using a larger number of redundant \emph{physical qubits} to encode for a smaller number of \emph{logical qubits}. By adding extra redundant qubits in our \emph{error correcting code}, we can carefully encode the quantum state which we wish to protect, as long as the rate of errors on the physical errors is low enough \cite{calderbank1996good, steane1996multiple, preskill1998reliable}.

In this chapter, we will introduce the principles of quantum error correction by the example of the \emph{three-bit repetition code}. In section \ref{sec:classical3bit}, we will first cover the classical variant, and the quantum variant in section \ref{sec:quantum3bit}. A more practical language to describe these codes is the \emph{stabilizer formalism}, in section \ref{sec:stabilizerformalism}. The set of tools and principles explained in these sections form the basis for higher level quantum codes that we will come later in this thesis.

\section{Classical three-bit repetition code}\label{sec:classical3bit}
To introduce some of the terms that we are going to use later, let us first start with a classical example the three-bit repetition code. This code encodes bits (a single bit in the example) by repeating them. Let the \emph{codewords} of the single logical bit be:
\begin{align}\label{eq:qb_3bitlogical}
    && 0_L = 000 &&& 1_L=111 &
\end{align}
In order to do computations, a NOT gate may be applied to the codewords to flip the logical value:
\begin{equation}
 000 \leftrightarrow 111
\end{equation}
A \emph{bit-flip} error can occur on any of the three bits in the code, which flips the single bit-value from 0 to 1 and 1 to 0. An error can be \emph{detected} by measuring the bits and comparing whether the bits have equal value. An detected error can be \emph{corrected} by computing the majority-function of the bitstring. Thus the three-bit repetition code will be correctly corrected if less than half of the bits were flipped.
\begin{align}
  0_L \xrightarrow{E_2} 010 \xrightarrow{correction} 000 && 0_L \xrightarrow{E_2, E_3} 011 \xrightarrow{correction} 111
\end{align}
The \emph{distance} $d$ of a classical code is the minimum amount of bit flips to transfer one codeword to another. In the case of the three-bit repetition code, the code distance is 3. The number of bits in de code $n$, the number of encoded bits $k$ and the distance of a code can be used to fully describe a code in the $[n, k, d]$ notation. The three-bit repetition code is a $[3,1,3]$ code.

\section{Quantum three-bit repetition code}\label{sec:quantum3bit}

From here, we can describe how to do computations on a quantum system. We start by considering an example of the \emph{quantum three-bit repetition code}, where the classical bits are now replaced by qubits that can be in the superposition of the two classical 0 or 1 states. The basis states of the encoded qubit is the tensor product of the single qubit states:
\begin{align}\label{eq:qb_3bitlogicalq}
&& \ket{0}_L = \ket{0}\otimes\ket{0}\otimes\ket{0} = \ket{000} && \ket{1}_L = \ket{1}\otimes\ket{1}\otimes\ket{1} = \ket{111}
\end{align}
A pure qubit state can also be a superposition of the bases states and is encoded as:
\begin{equation}\label{eq:qec_3bitstate}
  \ket{\psi}_L = \alpha\ket{0}_L + \beta\ket{1}_L
\end{equation}

\subsection{Pauli operators}\label{subsec:pauli}

The Pauli operations are unitary operations on single qubits, and will be applied very often throughout this thesis. Including the identity operator, the Pauli group on a single qubit, $\mathcal{P}_1$, consists of:
\begin{align}
  X = \begin{bmatrix} 0 & 1 \\ 1 & 0 \end{bmatrix} &&
  Y = \begin{bmatrix} 0 & -i \\ i & 0 \end{bmatrix} &&
  Z = \begin{bmatrix} 1 & 0 \\ 0 & -1 \end{bmatrix} &&
  I = \begin{bmatrix} 1 & 0 \\ 0 & 1 \end{bmatrix}
\end{align}
The Pauli operators represent errors that can occur on a single qubit. The Pauli X operator is analogous to the classical \emph{bit-flip} error and acts on the qubit computational basis states:
\begin{align}\label{eqq:qec_bitflip}
  & X\ket{0} = \ket{1} && X\ket{1} = \ket{0} &
\end{align}
Additionally, the Pauli Z operator introduces phase errors on a quantum bit:
\begin{align}\label{eq:eqc_phaseflip}
  & Z\ket{0} = -\ket{0} && Z\ket{1} = -\ket{1} &
\end{align}
The elements of the Pauli group on $n$ qubits, $\mathcal{P}_n$, consists of tensor products of single qubit Pauli operators, such that  $\mathcal{P}_n = \mathcal{P}_1^{\otimes n}$. We use the index of a Pauli operator to indicated on which qubit it has operated on, while other qubits are acted on by the identify. In cases without indices, the order of the operators indicate the qubit it acts on. For example, element $P=X_1\otimes X_2 = XX$ on the three-bit repetition code means that qubit 1 and 2 have been acted on by the Pauli X operator, while qubit 3 is acted on by the identity. The tensor product symbol is often omitted for clarity, such that the above operation can be also written as  $P=X_1X_2$. The \emph{weight} of an operator is the number of qubits on which it does not act non-trivially. On the pure  three-qubit encoded state, a bit-flip error on the second qubit is applied as:
\begin{equation}\label{eq:qec_3bitflip}
  X_2\ket{\psi}_L = (I\otimes X \otimes I) \ket{\psi}_L = \alpha\ket{010} + \beta\ket{101}
\end{equation}

\subsection{Logical operations}

In order to do computations on the encoded qubit of our three-bit repetition code, we wish to find the Pauli operators in $\mathcal{P}_3$ which flips any basis of the basis states of the encoded qubit to the other. We find that $X\otimes X\otimes X$ transforms $\ket{0}_L$ to $\ket{1}_L$, which is known as the logical bit-flip.

Furthermore, we now have the logical Z operator which must map $\ket{0}_L$ to $-\ket{0}_L$ and $\ket{1}_L$ to $-\ket{1}_L$. We see that for example $Z\otimes I\otimes I$ achieves this, but also any other $\mathcal{P}_3$ operator with two identities and one Pauli Z operator. Thus there are multiple operators that achieves the same. We formalize the logical operators as
\begin{align}\label{eq:qec_3bitlogical}
  \bar{X} = XXX && \bar{Z} = ZII
\end{align}

The distance of a quantum code is the minimal weight of any logical operators on the code. In the above case, the weight of the encoded X operator $\bar{X}$ is 3, hence the code can detect up to 2 X errors, analogous to the classical case. However, the weight of the encoded Z operator $\bar{Z}$ is only 1, which means that Z errors cannot be detected at all.

\subsection{Error detection}

To detect errors in our repetition code, we now cannot measure the states directly, as any measurement would collapse the encoded state, and therefore destroy the encoded information. Instead, we can now detect errors by measuring the \emph{parity} of two or more qubits rather than single qubits. For example, for two qubits, we can measure the parity by adding an \emph{ancillary} or \emph{ancilla} qubit prepared in $\ket{0}$ and measure it in the computational basis after connecting our quantum circuit as:

\begin{figure}
  \centering
    \begin{tikzcd}[row sep={0.65cm,between origins}]
    \lstick{$Q_1$} & \qw & \ctrl{2} & \qw & \qw &\\
    \lstick{$Q_2$} & \qw & \qw & \ctrl{1} & \qw &\\
    \lstick{$\ket{0}$} & \qw & \targ{} & \targ{} & \qw & \meter{}
  \end{tikzcd}
  \caption{The quantum circuit for a parity measurement on two qubits, $Q_1$ and $Q_2$, which is measured on the ancilla qubit prepared in $\ket{0}$. }\label{fig:2qubitparity}
\end{figure}


Note that measuring the ancilla qubit in the computational basis will be equivalent to measuring $Z\otimes Z$ on the first two qubits, as
\begin{equation}
\begin{aligned}
    (ZZ)\ket{00} &= \ket{00} && (ZZ)\ket{01} &= -\ket{01} \\
    (ZZ)\ket{10} &= -\ket{10} && (ZZ)\ket{11} &= \ket{11}
\end{aligned}
\end{equation}
Now for our quantum three-bit repetition code, we need to setup ancilla qubits between each of the 3 qubits, such that the parity between any two qubits can be measured. For the state in equation \ref{eq:qec_3bitstate}, any parity measurement $ZZI$, $ZIZ$ or $IZZ$ will return even parity. If one of the qubits has encountered a bit-flip error such as in the second qubit in equation \ref{eq:qec_3bitflip}, two of the parity measurements will return a -1 eigenvalue, in this case $ZZI$ and $IZZ$.

Furthermore, we see that no configuration of ancilla qubits could be set up for the three-bit repetition code to detect for phase errors, which was already set by the weight of the logical Z operator.

\subsection{Error correction}

As the error has been identified, we can apply correct Pauli operator from $\mathcal{P}_3$ to correct the error. In the above case, we wish to apply $IXI$ to clip the second qubit to correct the code.

In the quantum three-bit repetition code, we can very simply deduct which qubit has encountered an error from the combination of uneven parity measurements. This is called \emph{decoding} the error and is the main function of a \emph{decoder}. More complex error correcting codes involves decoding algorithms which are far more complex than in the three-bit repetition code, which we will see more of later.


\section{Stabilizer Formalism}\label{sec:stabilizerformalism}

As we have seen above, we can use the Pauli group $\mathcal{P}_n$ to easily describe a quantum error correcting code of $n$ qubits, without explicitly looking at the \emph{state} of the qubit. This powerful technique is called the \emph{stabilizer formalism} \cite{gottesman1997stabilizer}, and is the most widely formalism used to describe topological codes.

A quantum error correcting code that can be described using the stabilizer formalism is called a \emph{stabilizer code}. A stabilizer code is defined by two sets of operators, a set of \emph{stabilizer generators} which form the \emph{stabilizer group} $\mathcal{S}$, which is an Abelian subgroup of the Pauli group, and a set of encoded \emph{logical operators}. A stabilizer group is the set of Pauli operators which leave all states $\ket{\psi}_i$ from the \emph{codespace}, a subspace of the Hilbert space of $n$ qubits spanned by the codeword basis states, invariant, such that:
\begin{align}
  & S\ket{\psi}_i = \ket{\psi}_i, && \forall S \in \mathcal{S} &
\end{align}
These elements of the of the stabilizer group are most simply referred to as \emph{stabilizers}. The elements of this Abelian group can be written by a set of independent generators $S_j$ of size $N_S$, where any stabilizer $S$ can be written as
\begin{align}
 & \mathcal{S} = \prod_{j}S_j^{a_j}, && a\_j \in {0, 1} &
\end{align}
The set of generators $S_j$ is \emph{independent} if no generator can be written as a product of other generators. This implies that any stabilizer can be written in terms of the bitstring $a_1, a_2, ...a_{N_S}$ and that the stabilizer group takes up $n_S$ degrees of freedom of the Hilbert space of $N$ qubits. The remaining $N-N_S = N_L$ degrees of freedom which are not specified by the stabilizers make up the \emph{codespace}, the subspace spanned by the logical basis states, or the number of encoded logical qubits. Thus, if $N_L$ logical qubits are to be encoded by $n$ qubits, we require a total of $N_S = N-N_L$ independent stabilizer generators.


\subsection{Encoded logical operators}

Next to the set of stabilizers, we can construct the set of logical operators that will act on the encoded qubits from a set commutation rules. First of all, all logical operators must commute with all elements of the stabilizers, as a logical operator is made up from Pauli operators, and any Pauli operator which anticommutes with a stabilizer cannot leave the codespace invariant. Note that his means that a logical operator is not unique, as it can be multiplied with an element of the stabilizer. Secondly, we can impose commutation rules for the logical operators themselves based on the Pauli operators that they are representing. For example, logical $\bar{X}$ and $\bar{Z}$ operators must anticommute.

The minimum weight of the logical operator determines the distance $d$ of the error correcting code. This is thus the minimal amount of errors that can cause a logical failure. Together with the total number of qubits $n$ and number of logical qubits $k$, they provide a rough measure of the error correcting capabilities of the code.

\subsection{Error detection procedure}

As the stabilizers are a set of Pauli operators, they correspond to blip-flip or phase-flip errors that may have happened on any of our $n$ qubits. Furthermore, as any stabilizer leaves all states  $\ket{\psi}_i$ invariant, measuring the stabilizers does not disrupt the encoded information. If no error has occurred, all stabilizer measurements will return a '+1' eigenvalue, while any '-1' outcome points to the presence of errors, which we will call a \emph{stabilizer violation}.

This outcome is dependent on whether an error caused by the error operator $E$, must either commute or anticommute with the stabilizer generators, since all operators are members of the same Pauli group $\mathcal{P}_n$. If $E$ and generator $S_j$ commute then,
\begin{equation}
  S_jE\ket{\psi} = ES_j\ket{\psi} = E\ket{\psi}
\end{equation}
which means that the post-error state is a +1 eigenstate of $S_j$. If $E$ and generator $S_j$ anticommute then,
\begin{equation}\label{qec:eq:stabmeas}
  S_jE\ket{\psi} = -ES_j\ket{\psi} = -E\ket{\psi}
\end{equation}
and the post-error state is a -1 eigenstate of $S_j$. Errors on stabilizers codes are therefore detected by measuring the stabilizers, which returns a series of eigenvalue outcomes that is called the \emph{syndrome}. However, it is not necessary to measure all operators in the stabilizer group $\mathcal{S}$. Measuring the set of independent stabilizer generators $S_j$ suffices as any other stabilizer is just a combination of already measured states.

\subsection{Error models}\label{qec:sec_errormodels}

Any qubit can be subject to a combination of errors, each can be caused by a difference factor in our quantum system. To generalize these errors, we define certain \emph{error models} that constricts the errors that take place. Here, we list a few models that we will encounter in this thesis. 

\paragraph{Independent noise model}
We were already introduced in the \emph{bit-flip} and \emph{phase-flip} errors in section \ref{subsec:pauli}. But let us know generalized them in the form of the density matrix $\rho$. Let $\Phi$ be a quantum channel that maps $\rho$ to $\Phi(\rho)$, and let the chance of a bit-flip error be $p_X$, the resulting state should be
\begin{equation}\label{qec:eq:bitflip}
  \Phi_X(\rho) = (1-p_X)\rho + p_X(X\rho X).
\end{equation}
Analogously, let the chance of a phase-flip error be $p_Z$, the resulting state should be
\begin{equation}\label{qec:eq:phaseflip}
  \Phi_Z(\rho) = (1-p_Z)\rho + p_Z(Z\rho Z).
\end{equation}

The bit-flip and phase-flip errors can be considered together as the \emph{independent noise model} or the \emph{uncorrelated noise model}. As the two types of errors are independent, they can be studied separately from each other.

\paragraph{The depolarizing noise model}
In the \emph{depolarizing} noise model, the afflicted quantum state is replaced by a complete mixed state with probability $p_D$. Let the completely mixed state be written as
\begin{equation}\label{qec:eq:mixstate}
  \frac{1}{2}I = \frac{1}{2}(\rho + X\rho X + Y\rho Y + Z\rho Z),
\end{equation}
then the depolarizing channel is described as
\begin{align}\label{qec:eq:depolarizing}
  \Phi_D(\rho) &= (1-\frac{3}{4}p_D)\rho + \frac{p_D}{4}(X\rho X + Y\rho Y + Z\rho Z) \\
   &= (1-p^*_D)\rho + \frac{p^*_D}{3}(X\rho X + Y\rho Y + Z\rho Z)
\end{align}
which can be interpreted as the state $\rho$ is left untouched with probability $p^*_D =\frac{3}{4}p_D$, and each Pauli gate is applied to it with probability $1-p^*_D$. Differently from the \emph{independent noise model}, to optimally decode, we need to take into account correlations between X and Z errors.

\paragraph{The erasure noise model}
In the \emph{erasure} noise model, a qubit is completely erased or lost from the system. Such a loss can be detected and the missing qubit is then replaced by a qubit in the totally mixed state (equation \ref{qec:eq:mixstate}). The erasure channel can therefore be seen as the depolarizing channel with the extra property that it can be detected which qubits suffer the error.

\subsection{Error correction or decoding}

As the Pauli operators are self-inverse, any error $E$ can be corrected by applying it again. From the measurement outcomes of a stabilizer measurement, we can deduce which error $E$ must have caused the syndrome. However, this relationship is not always one-to-one, as an error $E$ and its multiplication with a stabilizer $ES$ will lead to an identical syndrome. This is called the \emph{code degeneracy}. The choice of the most appropriate error to correct is not a trivial task, and algorithms that are tasked to automate this process are called \emph{decoders}.

\subsection{Stabilizer codes}

The three-qubit repetition code we already covered can now be described in the stabilizer formalism. We had already found that the logical operators are encoded $\bar{X} = XXX$ and $\bar{Z} = ZII$. Other logical operators also exist up to a stabilizer. The stabilizer generators are needed to complete its description. We had found that two parity measurements, for example $ZZI$ and $IZZ$, will identify the error, as they will either commute or anticommute with the error, as such they are a set of independent stabilizer generators. 

The resulting measurement of '+1' or '-1' eigenvalues make up the syndrome. De decoder algorithm here is quite simple, but fails in the case if there is more than 1 bit-flip error. The code has $n=3$ and $n_S=2$ which results in the expected $n_L = 1$ encoded bits. The distance $d$ of the is 1, which conforms that there are certain errors, in this case phase-flip errors, which this code cannot detect.

The smallest code which can correctly solve for both single bit-flip and phase-flip errors is the \emph{5-qubit repetition code} \cite{laflamme1996perfect}. This code has the following stabilizer generators:
\begin{align}
  XZZXI && IXZZX && XIXZZ && ZXIXZ
\end{align}
with the logical operators up to a stabilizer:
\begin{align}
  & \bar{X} = XXXXX && \bar{Z} = ZIIII &
\end{align}
This codes now has $n=5$ bits with $n_S = 4$ stabilizers, which means it still encodes for a single bit.\\
\\
\\
With the principles of encoding and decoding in the quantum error correction in mind, we are now ready to move on to a more complicated variant of stabilizer codes, the \emph{surface code}. 





\chapter{The surface code}

\begin{figure}[h]
  \centering
  \begin{tikzpicture}
    \DRAWTORIC{3}
    \draw [arrow] (-1,0 |- N-0-2-1) node [align=right, left] {qubit/edge} -- (N-0-2-1);
    \node (plaquette) at ($(N-0-1-0)!0.5!(N-0-0-0)$) {};
    \node (star) at ($(N-1-0-1)!0.5!(N-1-1-1)$) {};
    \draw [arrow] (-1,0 |- plaquette)  node [align=right, left] {face} -- (plaquette);
    \draw [arrow] (-1,0 |- N-1-0-1) node [align=right, left] {vertex} to [out=0, in=225] (star);
    \node [align=left, right] at (3*\s + .5, .5*\s) {periodic boundary};

  \end{tikzpicture}
  \caption{The toric code is defined as a $L\times L$ lattice (here $L=3$) with periodic boundary conditions. The edges on the lattice, which represents the qubits, make up faces and vertices.}\label{sf:fig_toriclattice}
\end{figure}

The variant of the stabilizer codes that we are going to explore in this thesis is Kitaev's \emph{surface code} \cite{kitaev2003fault}, which is of the category of \emph{topological codes}. Among this category, the surface code is preferable as it offers the highest error tolerance under realise noise channels and requires only local stabilizer measurements of physically neighboring qubits. Two variants of the surface code will be considered here, the \emph{toric code} in section \ref{sec:surface_toric} and the \emph{planar code} in section \ref{sec:surface_planar}, and various decoders are detailed in \ref{sec:surface_decoders}.


\section{The toric code}\label{sec:surface_toric}

The \emph{toric code} is defined by arranging qubits on the edges of a square lattice with periodic boundary conditions, as seen in Figure \ref{sf:fig_toriclattice}. The name of the toric code lends itself from the torus, or donut, shape, where any point on the surface of the torus will encounter itself after traversing the torus in either x or y directions. Hence, the top edge of the toric code meets the bottom edge, whereas the left edge meets the right. On a $L\times L$ grid there are $N = 2L^2$ edges and the same amount of physical qubits. This topology of qubit arrangement plays an important part in encoding the logical qubits, which is stored in the non-trivial cycles on the torus. Errors, beneath a certain threshold, will only introduce local effects and does not change these cycles.

\subsection{Stabilizer generators}

To define a stabilizer code, we need to specify the $m$ independent stabilizer generators and the encoded $\bar{X}$ and $\bar{Z}$ operators. On the toric code there are two types of stabilizer generators, \emph{plaquette} and \emph{star} operators, which are associated with the \emph{faces} and \emph{vertices} of the square lattice, respectively.

\begin{figure}
  \centering
  \begin{tikzpicture}
    \DRAWTORIC{3}
    \DRAWPLAQ{1}{1}
    \DRAWERROR{1}{1}{0}{z}
    \DRAWERROR{1}{1}{1}{z}
    \DRAWERROR{1}{0}{0}{z}
    \DRAWERROR{2}{1}{1}{z}
    \node[below of=Bx-1] {(a)};
  \end{tikzpicture}
  \hspace{1cm}
  \begin{tikzpicture}
    \DRAWTORIC{3}
    \DRAWSTAR{1}{1}{3}
    \DRAWERROR{1}{1}{1}{x}
    \DRAWERROR{1}{1}{0}{x}
    \DRAWERROR{1}{2}{1}{x}
    \DRAWERROR{0}{1}{0}{x}
    \node[below of=Bx-1] {(b)};
  \end{tikzpicture}

  \begin{center}
    \hspace{1cm}
    \begin{tikzcd}[row sep={0.5cm,between origins}]
      \lstick{$Q_1$} & \qw & \qw & \qw & \ctrl{5} & \qw \\
      \lstick{$Q_2$}& \qw & \qw & \ctrl{4} & \qw & \qw \\
      \lstick{$Q_3$} & \qw & \ctrl{3} & \qw & \qw & \qw \\
      \lstick{$Q_4$} & \ctrl{2} & \qw & \qw & \qw & \qw \\
      &&&&&&&\\
      \lstick{$P_f$} & \ctrl{} & \ctrl{} & \ctrl{} & \ctrl{} & \meter{}
    \end{tikzcd}
    \begin{tikzcd}[row sep={0.5cm,between origins}]
      \lstick{$Q_1$} & \qw & \qw & \qw & \targ{} & \qw \\
      \lstick{$Q_2$} & \qw & \qw & \targ{} & \qw & \qw \\
      \lstick{$Q_3$} & \qw & \targ{} & \qw & \qw & \qw \\
      \lstick{$Q_4$} & \targ{} & \qw & \qw & \qw & \qw \\
      &&&&&&&\\
      \lstick{$S_v$} & \ctrl{-2} & \ctrl{-3} & \ctrl{-4} & \ctrl{-5} & \meter{}
    \end{tikzcd}
  \end{center}

  \caption{Each face (a) and vertex (b) on the lattice represents a plaquette and star operator, respectively. The non-identity single qubit operators on which they act are indicated. The set of all (but one) plaquettes and vertices make up the stabilizers of the code. }\label{sf:fig_stabilizers}
\end{figure}

\paragraph{Plaquette operators}
For every face $f$ on our lattice, we define a plaquette operator $P_f$, consisting of tensor product of Pauli Z operators on qubits on these edges (see Figure \ref{sf:fig_toriclattice}a),
\begin{equation}\label{eq:sf_plaquette}
  P_f = \bigotimes_{i\in Q(f)} Z_i
\end{equation}
where $Q(f)$ is the set of qubits touching face $f$. On a $L\times L$ grid there are $L^2$ plaquettes.

\paragraph{Star operators}

Similarly, for every vertex $v$ on our lattice, we define a star operator $S_v$, consisting of tensor product of Pauli X operators on qubits neighboring the vertex (see Figure \ref{sf:fig_toriclattice}b),
\begin{equation}\label{eq:sf_star}
  S_v = \bigotimes_{i\in Q(v)} X_i
\end{equation}
where $Q(v)$ is the set of qubits neighboring vertex $v$. On a $L\times L$ grid there are $L^2$ plaquettes.

As each plaquette and star operator needs to be measured, an ancilla qubit is needed at the physical locations of each of these operators. The structure of the full lattice is now clear, as it just a simple square arrangement of alternating data and ancilla qubits in both x and y directions.

The full stabilizer of the code $\m{S}$ can be generated by multiplying elements of the generator operators. Consider two plaquette operators. These two operators will either share one boundary consisting of a qubit, or none. This means that the Pauli Z operator on the boundary qubit will add up to identity as they commute. The result is that the product of the plaquette operators will consists of the overall boundary Pauli operators of the joint plaquette (see Figure \ref{sf:fig_multistab}a).

However, if all plaquettes are applied to the lattice, no boundary will be left. Thus the product of all plaquettes is the identity, which means that the full set of plaquettes are not independent. The full set of plaquette generators can therefore be completed by simply removing a single plaquette from all available plaquettes. There are therefore $L^2 - 1$ independent plaquette operators.

The multiplication of star operators follow the same properties as the plaquette operators described above (see Figure \ref{sf:fig_multistab}b). Thus there are also $L^2 - 1$ independent star operators, which are the star generators. This sums up to $N_S = 2L^2 - 2$ independent stabilizer generators.

\begin{figure}
  \centering
  \begin{tikzpicture}
    \DRAWTORIC{3}
    \DRAWPLAQ{1}{1}
    \DRAWPLAQ{0}{1}
    \DRAWPLAQ{0}{2}
    \DRAWERROR{1}{1}{0}{z}
    \DRAWERROR{1}{0}{0}{z}
    \DRAWERROR{2}{1}{1}{z}
    \DRAWERROR{0}{0}{0}{z}
    \DRAWERROR{0}{1}{1}{z}
    \DRAWERROR{0}{2}{1}{z}
    \DRAWERROR{0}{2}{0}{z}
    \DRAWERROR{1}{2}{1}{z}
    \node[below of=Bx-1] {(a)};
  \end{tikzpicture}
  \hspace{1cm}
  \begin{tikzpicture}
    \DRAWTORIC{3}
    \DRAWSTAR{1}{1}{3}
    \DRAWSTAR{2}{0}{3}
    \DRAWSTAR{2}{1}{3}
    \DRAWERROR{1}{2}{1}{x}
    \DRAWERROR{0}{1}{0}{x}
    \DRAWERROR{2}{2}{1}{x}
    \DRAWERROR{1}{1}{1}{x}
    \DRAWERROR{1}{0}{0}{x}
    \DRAWERROR{2}{0}{1}{x}
    \DRAWERROR{2}{0}{0}{x}
    \DRAWERROR{2}{1}{0}{x}
    \node[below of=Bx-1] {(b)};
  \end{tikzpicture}
  \caption{Multiplication of (a) plaquette and (b) star operators will result in a operator that consists of the Pauli operators that reside on the overall boundary of the joint plaquettes or stars. }\label{sf:fig_multistab}
\end{figure}

\subsection{Dual lattice}
Note that if we shift our lattice half a cell down, and half a cell to the right, we can create a \emph{dual} lattice. This dual lattice has the same size and same boundary conditions as the \emph{primal} lattice, but every plaquette in the primal lattice is a star in the dual lattice, and every star in the primal lattice is a plaquette in the dual lattice. The edges of the dual lattice are plotted with dotted lines in the figures.

This interesting property of \emph{lattice duality} leads to the fact that plaquette and star operators are in fact the same, and we can choose from either that is best suited for the calculation. The multiplication of operators is best pictured in the plaquette picture, for example. For the square lattice in the toric code, the dual lattice is coincidentally also square. For other types of topological codes with non-square lattices, the dual lattice has a different lattice structure than the primal lattice. We will not explore these kind of lattices in this thesis.

\subsection{Encoded qubits}
Since there are $N = L^2$ qubits and $N_S = 2L^2 - 2$ independent stabilizers, we must have $N_L = N - N_S = 2$ encoded qubits and therefore 4 logical operators $\bar{X}_1, \bar{X}_2, \bar{Z}_1$ and $\bar{Z}_2$.

Recall the logical operators consists of the Pauli operators, and must commute with all stabilizer generators, but cannot be part of the stabilizer itself. We can construct the logical operators by starting with, for example, a single Pauli Z operator. It commutes with all plaquette operators trivially. In terms of the star operators, this single Pauli Z operator commutes with all but the two neighboring qubits, as all others apply to different qubits. Adding another Pauli Z operator will shift will of the anticommuting neighboring star operators. We know see that a closed loop of Z operators around the torus does not have neighboring star operators, and therefore commute with all stabilizers. As the torus has two directions we can loop over, these are the logical $\bar{Z}$ operators (see Figure \ref{sf:fig_logical}a-b). Analogously, we can construct the logical $\bar{X}$ operators in the same way (Figure \ref{sf:fig_logical}c-d).

Note that these logical operators are not unique. As the logical operators commute with the stabilizers, these $\bar{X}$ and $\bar{Z}$ operators can be multiplied with e.g. a plaquette or star operator, respectively, which create a diversion from its original path. But as the path still loops around the torus, this is still a valid logical operator.

The logical operators have a minimum length of $L$ qubits, which is also the distance of the toric code. The toric code is therefore a $[L^2,2,L]$ in the [n,k,d] notation. This implies that the toric code might be more robust against errors if the size of the lattice is increased. Later we will see that this is also very much dependent on the type of decoder that is used, and that different decoders will lead to different regimes of error for which this reasoning is true.

\def\QS{10}
\def\s{1}

\begin{figure}
  \centering
  \begin{tikzpicture}
    \DRAWTORIC{5}
    \DRAWERROR{0}{2}{0}{z}
    \DRAWERROR{1}{2}{0}{z}
    \DRAWERROR{2}{2}{0}{z}
    \DRAWERROR{3}{2}{0}{z}
    \DRAWERROR{4}{2}{0}{z}
    \DRAWERROR{3}{0}{1}{z}
    \DRAWERROR{3}{1}{1}{z}
    \DRAWERROR{3}{2}{1}{z}
    \DRAWERROR{3}{3}{1}{z}
    \DRAWERROR{3}{4}{1}{z}
    \begin{pgfonlayer}{edges}
      \draw[synz] (S-0-2) -- (S-5-2);
      \draw[synz] (S-3-4) -- (S-3-5);
    \end{pgfonlayer}
    \node[above=.25cm of S-3-4] {(a)};
    \node[left=.25cm of S-0-2] {(b)};
  \end{tikzpicture}
  \hspace{1cm}
  \begin{tikzpicture}
    \DRAWTORIC{5}
    \DRAWERROR{0}{2}{1}{x}
    \DRAWERROR{1}{2}{1}{x}
    \DRAWERROR{2}{2}{1}{x}
    \DRAWERROR{3}{2}{1}{x}
    \DRAWERROR{4}{2}{1}{x}
    \DRAWERROR{3}{0}{0}{x}
    \DRAWERROR{3}{1}{0}{x}
    \DRAWERROR{3}{2}{0}{x}
    \DRAWERROR{3}{3}{0}{x}
    \DRAWERROR{3}{4}{0}{x}
    \begin{pgfonlayer}{edges}
      \draw[synx] (N-0-2-1) -- (By-2);
      \draw[synx] (Bx-3) -- (N-3-4-0);
    \end{pgfonlayer}
    \node[above=.25cm of N-3-4-0] {(c)};
    \node[right=.25cm of By-2] {(d)};
  \end{tikzpicture}
  \caption{The logical (a) $\bar{X}_1$, (b) $\bar{X}_2$, (c) $\bar{Z}_1$ and (d) $\bar{Z}_2$ operators are the closed loop of $X$ and $Z$ operators, respectively, that go around the two boundaries of the torus. }\label{sf:fig_logical}
\end{figure}

\subsection{Error detection}
As discussed in the previous chapter, errors are detected by measuring the set of stabilizer generators. As we have seen in the previous section, this consists of all but one plaquette operators $P_f$ and all but one star operators $S_v$. Let us first consider to measure all of them.

In the case of a single $Z$ error (Fig \ref{sf:fig_degenerate}a.i), the neighboring plaquette operators will commute with this error, as it consists of Pauli Z operators itself. But the neighboring star operators anticommutes with this error according to equation \ref{qec:eq:stabmeas}. Similarly, a single $X$ error (Fig \ref{sf:fig_degenerate}a.ii) commutes with neighboring star operators but anticommutes with neighboring plaquette operators. A $Y$ error is a combination of $X$ and $Z$ operators and therefore anticommutes with all neighboring generator operators (Fig \ref{sf:fig_degenerate}a.iii).

In the case of two $Z$ errors (Fig \ref{sf:fig_degenerate}a.iv), the star operators between the two errors now commute with the errors, creating a virtual path between them. This is a general property: given any string of errors, the generator operators at the end of the string will anticommute with the errors and measure -1. For $Z$ errors, star operators at the end of strings on the primal lattice will measure -1. The detection of $X$ errors occur in the same way, albeit now the strings of errors is defined on the \emph{dual} lattice, and plaquette errors will measure -1 at the end of these strings.

Since $Z$ and $X$ errors independently affect different types of stabilizer measurements (stars and plaquettes, respectively), these two types of errors can be considered independently in two error correction processes. The two processes are analogous, up to the duality of the lattice. Therefore, for the remainder of the section, only $Z$ errors, which leave a string of errors on the primal lattice, will be considered.

\begin{figure}
  \centering
  \begin{tikzpicture}
    \DRAWTORIC{5}
    \DRAWERROR{1}{2}{0}{z}
    \DRAWERROR{2}{4}{1}{x}
    \DRAWERROR{3}{1}{0}{y}
    \DRAWERROR{1}{0}{0}{z}
    \DRAWERROR{0}{0}{0}{z}
    \DRAWPLAQ{1}{4}
    \DRAWPLAQ{2}{4}
    \DRAWSTAR{1}{2}{5}
    \DRAWSTAR{2}{2}{5}
    \DRAWSTAR{0}{0}{5}
    \DRAWSTAR{2}{0}{5}
    \DRAWPLAQ{3}{1}
    \DRAWPLAQ{3}{2}
    \DRAWSTAR{3}{1}{5}
    \DRAWSTAR{4}{1}{5}
    \begin{pgfonlayer}{edges}
      \draw[synz] (N-0-0-0) -- (N-1-0-0);
    \end{pgfonlayer}
    \node[below=.25cm of Bx-2] {(a)};
    \node[script] at (P-1-2) {\textit{(i)}};
    \node[script] at (P-1-4) {\textit{(ii)}};
    \node[script] at (P-3-1) {\textit{(iii)}};
    \node[script] at (P-1-0) {\textit{(iv)}};

  \end{tikzpicture}
  \hspace{1cm}
  \begin{tikzpicture}
    \DRAWTORIC{5}
    \DRAWERROR{0}{2}{0}{z}
    \DRAWERROR{2}{2}{0}{z}
    \DRAWERROR{3}{2}{0}{z}
    \DRAWERROR{4}{2}{0}{z}
    \DRAWSTAR{1}{2}{5}
    \DRAWSTAR{2}{2}{5}
    \begin{pgfonlayer}{edges}
      \draw[synz] (N-0-2-0) -- (S-0-2) (N-2-2-0) -- (S-5-2);
    \end{pgfonlayer}
    \node[below=.25cm of Bx-2] {(b)};
  \end{tikzpicture}

  \hspace{1cm}
  \begin{tikzpicture}
    \DRAWTORIC{5}
    \DRAWERROR{1}{2}{1}{z}
    \DRAWERROR{1}{1}{1}{z}
    \DRAWERROR{1}{0}{0}{z}
    \DRAWERROR{2}{0}{1}{z}
    \DRAWERROR{2}{4}{1}{z}
    \DRAWERROR{2}{3}{1}{z}
    \DRAWSTAR{1}{2}{5}
    \DRAWSTAR{2}{2}{5}
    \begin{pgfonlayer}{edges}
      \draw[synz] (N-1-2-1) -- (S-1-0) -- (S-2-0) -- (S-2-5) (S-2-4) -- (N-2-3-1);
    \end{pgfonlayer}
    \node[below=.25cm of Bx-2] {(c)};
  \end{tikzpicture}
  \hspace{1cm}
  \begin{tikzpicture}
    \DRAWTORIC{5}
    \DRAWERROR{0}{2}{0}{z}
    \DRAWERROR{4}{2}{0}{z}
    \DRAWERROR{4}{2}{1}{z}
    \DRAWERROR{4}{1}{1}{z}
    \DRAWERROR{3}{0}{0}{z}
    \DRAWERROR{3}{0}{1}{z}
    \DRAWERROR{3}{4}{1}{z}
    \DRAWERROR{3}{3}{1}{z}
    \DRAWERROR{2}{2}{0}{z}
    \DRAWSTAR{1}{2}{5}
    \DRAWSTAR{2}{2}{5}
    \begin{pgfonlayer}{edges}
      \draw[synz] (N-0-2-0) -- (S-0-2) (N-2-2-0) -- (S-3-2) -- (S-3-4);
      \draw[synz] (S-3-5) -- (S-3-0) -- (S-4-0) -- (S-4-2) -- (S-5-2);
    \end{pgfonlayer}
    \node[below=.25cm of Bx-2] {(d)};
  \end{tikzpicture}
  \caption{(a) Stabilizer generators that anticommute with the error will measure -1, which are (i) the neighboring star operators for a Z error, (ii) the neighboring plaquette operators for an X error, and (iii) both star and plaquette operators for a Y error. In the case of a string of errors (iv), only the stabilizer generators at the end of these strings will anticommute with the error. Due to code degeneracy, the single Z error in (a.i) $E$ has the syndrome as (b) $E\bar{Z}_1$, (c) $E\bar{Z}_2$ and (d) $E\bar{Z}_1\bar{Z}_2$. }\label{sf:fig_degenerate}
\end{figure}

\subsection{Error correction}

An error $E$ can be corrected by applying it again to the lattice. The error operator $E$ is however unknown. We must therefore try to identify the correct operator given the measured syndrome. As mentioned in the previous chapter, this relationship between error does not always map one-to-one, which it is not in the surface code. An error $E$ can be multiplied with some operator $L$ that commutes with the stabilizer and they will result in the same syndrome.

If $L$ is in the stabilizer $\m{S}$, the product of the identified correction operator $C=E'$ with the real error operator $E$ will leave the code invariant. The resulting operator $CE=L$ is a stabilizer operator. However, the encoded logical operators also commute with the stabilizer, which means that $E$, $E\bar{Z}_1$, $E\bar{Z}_2$, $E\bar{Z}_1\bar{Z}_2$ will all lead to the same syndrome (Fig \ref{sf:fig_degenerate}a-d). Any identified correction operator $C$ can therefore be categorised into four classes of operators, of which only one includes the correct logical operator. The task of choosing most appropriate correction chain is up to the decoders (section \ref{sec:surface_decoders}).

\subsection{Quasiparticle picture}

The processes of error detection and correction can alternatively be presented in the \emph{quasiparticle picture}, where the anticommuting stabilizer measurements act like excitations on the lattice, which behave like the quasiparticles \emph{anyons}. A single error creates a pair of anyons, and a chain of errors causes movement of the anyon on the lattice. A pair of anyons can also annihilate each other when two error chains merge. The correction of errors can thus be viewed of movement of the correction chains until all anyons are annihilated. The quasiparticle picture removes the distracting underlying lattice from the problem, and decoding becomes simply identifying the right pairing between anyons to minimize the chance of a logical error.

\begin{figure}[h]
  \centering
  \begin{tikzpicture}
    \draw (0,0) rectangle (4,4);
    \node[circle, fill=red!50, minimum size=4] (N1) at (0.5,0.4) {};
    \node[circle, fill=red!50, minimum size=4] (N2) at (2,0.7) {};
    \node[circle, fill=red!50, minimum size=4] (N3) at (2.5,1.2) {};
    \node[circle, fill=red!50, minimum size=4] (N4) at (3.6,1) {};
    \node[circle, fill=red!50, minimum size=4] (N5) at (0.75,2.1) {};
    \node[circle, fill=red!50, minimum size=4] (N6) at (1.95,1.8) {};
    \node[circle, fill=cyan!50, minimum size=4] (N7) at (1.6, 3.4) {};
    \node[circle, fill=cyan!50, minimum size=4] (N8) at (2.7, 3.5) {};
    \node[circle, fill=cyan!50, minimum size=4] (N9) at (3.2, 1.5) {};
    \node[circle, fill=cyan!50, minimum size=4] (N10) at (3.1, 0.6) {};
    \draw[cyan!50, line width = 2] (N1) to[in=170, out=20] (N2);
    \draw[cyan!50, line width = 2] (N3) to[in=180, out=-10] (N4);
    \draw[cyan!50, line width = 2] (N5) to[in=170, out=-20] (N6);
    \draw[red!50, line width = 2] (N7) to[in=160, out=15] (N8);
    \draw[red!50, line width = 2] (N9) to[in=90, out=250] (N10);
    \node at (2, -.5) {(a)};
  \end{tikzpicture}
  \hspace{1cm}
  \begin{tikzpicture}
    \draw (0,0) rectangle (4,4);
    \node[circle, fill=red!50, minimum size=4] (N1) at (0.5,0.4) {};
    \node[circle, fill=red!50, minimum size=4] (N2) at (2,0.7) {};
    \node[circle, fill=red!50, minimum size=4] (N3) at (2.5,1.2) {};
    \node[circle, fill=red!50, minimum size=4] (N4) at (3.6,1) {};
    \node[circle, fill=red!50, minimum size=4] (N5) at (0.75,2.1) {};
    \node[circle, fill=red!50, minimum size=4] (N6) at (1.95,1.8) {};
    \node[circle, fill=cyan!50, minimum size=4] (N7) at (1.6, 3.4) {};
    \node[circle, fill=cyan!50, minimum size=4] (N8) at (2.7, 3.5) {};
    \node[circle, fill=cyan!50, minimum size=4] (N9) at (3.2, 1.5) {};
    \node[circle, fill=cyan!50, minimum size=4] (N10) at (3.1, 0.6) {};
    \draw[cyan!50, line width=2] (N1) to[in=170, out=20] (N2);
    \draw[cyan!50, line width=2] (N3) to[in=180, out=-10] (N4);
    \draw[cyan!50, line width=2] (N5) to[in=170, out=-20] (N6);
    \draw[red!50, line width=2] (N7) to[in=160, out=15] (N8);
    \draw[red!50, line width=2] (N9) to[in=90, out=250] (N10);
    \draw[cyan!50, dashed, line width=2] (N1) to[out=90, in=270] (N5);
    \draw[cyan!50, dashed, line width=2] (N2) to[out=80, in=275] (N6);
    \draw[cyan!50, dashed, line width=2] (N3) to[out=10, in=160] (N4);
    \draw[red!50, dashed, line width=2] (N7) to[out=-20, in=195] (N8);
    \draw[red!50, dashed, line width=2] (N9) to[out=290, in=75] (N10);
    \node at (2, -.5) {(b)};
  \end{tikzpicture}
  \hspace{1cm}
  \begin{tikzpicture}
    \draw (0,0) rectangle (4,4);
    \node[circle, fill=red!50, minimum size=4] (N1) at (0.5,0.4) {};
    \node[circle, fill=red!50, minimum size=4] (N2) at (2,0.7) {};
    \node[circle, fill=red!50, minimum size=4] (N3) at (2.5,1.2) {};
    \node[circle, fill=red!50, minimum size=4] (N4) at (3.6,1) {};
    \node[circle, fill=red!50, minimum size=4] (N5) at (0.75,2.1) {};
    \node[circle, fill=red!50, minimum size=4] (N6) at (1.95,1.8) {};
    \node[circle, fill=cyan!50, minimum size=4] (N7) at (1.6, 3.4) {};
    \node[circle, fill=cyan!50, minimum size=4] (N8) at (2.7, 3.5) {};
    \node[circle, fill=cyan!50, minimum size=4] (N9) at (3.2, 1.5) {};
    \node[circle, fill=cyan!50, minimum size=4] (N10) at (3.1, 0.6) {};
    \draw[yellow, line width=6, opacity=.3] (N3) to[in=180, out=-10] (N4);
    \draw[yellow, line width=6, opacity=.3] (N5) to[in=170, out=-20] (N6);
    \draw[yellow, line width=6, opacity=.3] (N3) to[out=120, in=-30] (N6);
    \draw[yellow, line width=6, opacity=.3] (N5) to[out=200, in=45] (0, 1.75);
    \draw[yellow, line width=6, opacity=.3] (N4) to[out=80, in=225] (4, 1.75);
    \draw[cyan!50, line width=2] (N1) to[in=170, out=20] (N2);
    \draw[cyan!50, line width=2] (N3) to[in=180, out=-10] (N4);
    \draw[cyan!50, line width=2] (N5) to[in=170, out=-20] (N6);
    \draw[red!50, line width=2] (N7) to[in=160, out=15] (N8);
    \draw[red!50, line width=2] (N9) to[in=90, out=250] (N10);
    \draw[cyan!50, dashed, line width=2] (N1) to[out=0, in=200] (N2);
    \draw[cyan!50, dashed, line width=2] (N3) to[out=120, in=-30] (N6);
    \draw[cyan!50, dashed, line width=2] (N5) to[out=200, in=45] (0, 1.75);
    \draw[cyan!50, dashed, line width=2] (N4) to[out=80, in=225] (4, 1.75);
    \draw[red!50, dashed, line width=2] (N7) to[out=-20, in=195] (N8);
    \draw[red!50, dashed, line width=2] (N9) to[out=290, in=75] (N10);
    \node at (2, -.5) {(c)};
  \end{tikzpicture}
  \caption{The quasiparticle picture of stabilizer measurements. Anticommuting stabilizers behave as anyons (circles), where a chain of errors (lines) creates a pair of anyons. Figure (b) shows a successful decoding of (a). Figure (c) shows a pairing that resulted in a correction operator that is in a different class as the error operator, which acquires a logical error.}\label{fig:quasiparticle}
\end{figure}

Figure \ref{fig:quasiparticle}a shows the quasiparticle representation of the errors suffered in Figure \ref{sf:fig_degenerate}a, which has suffered Z (blue lines) and X errors (red lines). The corresponding anyons can either be of the star type (red circle) or plaquette type (blue circle). Figure \ref{fig:quasiparticle}b shows a successful decoding. Note that here not all pairs are correctly identified, but the resulting loop still is in the same class of operators. In figure \ref{fig:quasiparticle}c the correction has failed as the resulting loop in the correction is in a difference class compared to the error. As the loop still commutes with the stabilizer, no error can be detected, but the encoded qubit has acquired a logical error.

\subsection{Code threshold}
Since the distance $d$ of the toric code on a $L\times L$ is $L$, we would expect that we can improve the robustness of the code by increasing the lattice size $L$. However, this also increases the total number of errors in the lattice, that adds an increased level of complexity in choosing the correct correction operator.

In practice, there is a trade-off between the positive effect of a larger code distance and the negative effect of larger number of errors. When the error rate $p$ is low, the positive effect outweighs the negative and increasing the lattice $L$ will increase the probability of successful error correction $p_C$. When the error rate is large, the negative effect outweighs the positive and increasing $p$ will decrease $p_C$. The point of transition in the error rate is called the \emph{code threshold} $p_{th}$.

The code threshold is not the only parameter that determines the potential of a certain code for practical use. The behavior for error rates far below the threshold is also important, as is the number of physical qubits needed to achieve the sought after level of error suppression. Nevertheless, the code threshold provides us with a very easy and useful tool to benchmark different codes and different decoding algorithms, and to compare them with each other. Therefore, in this thesis we will heavily rely on the value of the code threshold. The value of the threshold is heavily dependent on the chosen error model and the physical conditions of the stabilizer measurements. To compare different decoding algorithms, we therefore will use independent and identically distributed errors (i.i.d. noise), which is the \emph{independent noise model} from section \ref{qec:sec_errormodels}.

For the toric code, when the only source of errors is i.i.d. noise under the independent noise model, and all measurements can be made perfectly, the \emph{optimal threshold} has been proven to be 10.9\% (see section \ref{sec:optimal_decoder}). However, to achieve this value, one needs to consider all possible error configurations on the lattice to identify the correction operator $C$ that is most likely to be equal to the error operator $E$. This is a computationally heavy task that scales exponentially with the lattice size. It is therefore an impractical approach in reality.

Luckily, there exists other decoding algorithms that can find a solution much faster, albeit at the cost of reducing the code threshold. Edmond's \emph{Minimum Weight Perfect Matching} (MWPM) decoder scales cubic with the system, which allows for faster decoding, and achieves a code threshold of 10.3\% (section \ref{sec:MWPMdecoder}). Including faulty measurements the threshold drops down to 2.9\%. The \emph{Union-Find} decoder is a relatively new addition to the set of decoders for the surface code. It scales \emph{almost} linearly with the system, and has a code threshold of 9.9\% (section \ref{sec:UFdecoder}). In this thesis, we will try to combine certain properties of different decoders. In particular, we have created a heuristic for minimum weight which can be applied to the Union-Find decoder.

\section{The planar code}\label{sec:surface_planar}

Another variant of the surface code is the \emph{planar code}, which disposes the periodic boundary conditions of the torus. This allows the qubits to be placed onto a flat 2D surface. For real systems in which the qubits physically interact with each other, this is a huge benefit. Therefore, in this thesis, we will consider both toric and planar variants of the surface code.



\def\QS{15}
\def\s{1.5}

\begin{figure}[h]
  \centering
  \begin{tikzpicture}
    \DRAWPLANAR{6}
    \DRAWPLAQ{1}{3}
    \DRAWEPLAQ{0}{5}
    \DRAWSTAR{3}{3}{4}
    \DRAWESTAR{2}{5}
    \DRAWERROR{1}{3}{0}{z}
    \DRAWERROR{1}{3}{1}{z}
    \DRAWERROR{1}{2}{0}{z}
    \DRAWERROR{2}{3}{1}{z}
    \DRAWERROR{0}{4}{0}{z}
    \DRAWERROR{0}{5}{0}{z}
    \DRAWERROR{1}{5}{1}{z}
    \DRAWERROR{1}{5}{0}{x}
    \DRAWERROR{2}{5}{0}{x}
    \DRAWERROR{2}{5}{1}{x}
    \DRAWERROR{2}{3}{0}{x}
    \DRAWERROR{3}{3}{0}{x}
    \DRAWERROR{3}{3}{1}{x}
    \DRAWERROR{3}{4}{1}{x}
    \DRAWERROR{4}{0}{0}{x}
    \DRAWERROR{4}{1}{0}{y}
    \DRAWERROR{4}{2}{0}{x}
    \DRAWERROR{4}{3}{0}{x}
    \DRAWERROR{4}{4}{0}{x}
    \DRAWERROR{4}{5}{0}{x}
    \DRAWERROR{0}{1}{0}{z}
    \DRAWERROR{1}{1}{0}{z}
    \DRAWERROR{2}{1}{0}{z}
    \DRAWERROR{3}{1}{0}{z}
    \DRAWERROR{5}{1}{0}{z}
    \begin{pgfonlayer}{edges}
      \draw[synz] (N-0-1-0) -- (N-5-1-0);
      \draw[synx] (N-4-5-0) -- (N-4-0-0);
    \end{pgfonlayer}

    \node[script] at (P-1-3) {(c)};
    \node[script] at (S-3-3) {(d)};
    \node[script] at (P-0-5) {(a)};
    \node[script] at (S-2-5) {(b)};
    \node[above=.25cm of N-4-5-0] {(f)};
    \node[left=.25cm of N-0-1-0] {(e)};
  \end{tikzpicture}


  \caption{The planar code with lattice size $L=4$, which includes $N = 2L^2-2L+1$ qubits and $N_S = 2L^2-2L$ independent stabilizers. The boundary is defined by the (a) EDGE-plaquette and (b) EDGE-star operators, which exist next to the known (c) plaquette and (d) star operators, similar to the toric code. The planar codes encodes 1 logical qubit, which is represented by the logical (e) $\bar{Z}$ and (f) $\bar{X}$ operators.}\label{sf:fig_planar}
\end{figure}

\paragraph{Stabilizer generators}
There are a few key differences between the planar and toric codes. First of all, a new type of stabilizer generators define the non-periodic boundary of the lattice, which are referred to as \emph{EDGE operators}. These EDGE operators have only 3 neighboring qubits and are therefore the tensor product of 3 Pauli operators. The EDGE-plaquette operators lie at the east and west boundaries of the lattice (Figure \ref{sf:fig_planar}a) and the EDGE-star $S_{vE}$ operators lie at the north and south boundaries of the lattice (Figure \ref{sf:fig_planar}b). In the middle of the lattice, the bulk of the stabilizer generators still consist of 4 Pauli operators, identical to the ones in the toric code (Figure \ref{sf:fig_planar}c-d). Note that the stabilizer generators are still defined by equation \ref{eq:sf_plaquette} and \ref{eq:sf_star}, but now the relevant faces and vertices contain three neighboring qubits.

\paragraph{Stabilizer violoations}
A second key difference is that now not all errors will cause two stabilizer violations. In the bulk of the qubits on the lattice, a single error will still cause two neighboring stabilizers to measure -1, or create two anyons. At the boundary however, it now may be the case that an error is only included in one plaquette or star operator. This will also mean the decoding in the quasiparticle picture requires a slightly different approach.

\paragraph{Logical qubits}
Furthermore, we can inspect that a planar surface of dimension $L$ has $N = 2L^2-2L+1$ physical qubits. We can also find that there are $2L^2-2L$ stabilizer generators. As the boundary is now non-periodic, all generators are now independent, and therefore the number independent generators is $N_S = 2L^2-2L$. This means that the planar code encodes $N_L = N-N_S = 1$ a single logical qubit. The logical $\bar{X}$ and $\bar{Z}$ operators are pictured in Figure \ref{sf:fig_planar}e-f.\\

Other properties of the planar code are very similar to the toric code. The \emph{dual lattice} also exists for the planar code, for example. But the dual lattice exists at a 90 angle compared to the primary lattice. Also, as the bulk of the lattice still consists of 4-Pauli operator stabilizers, the decoding algorithms for the planar code is very similar to the toric code. It is due to the boundary that some slight alterations are needed, as we will see in the next section.

\section{Decoders}\label{sec:surface_decoders}
\subsection{The optimal decoder}\label{sec:optimal_decoder}
\subsection{Minimum Weight Perfect Matching}\label{sec:MWPMdecoder}
\subsection{The Union-Find decoder}\label{sec:UFdecoder}

Even the fastest MWPM algorithms still have a quadratic time complexity of $\m{O}(n^2\sqrt{n})$, where $n$ is the number of qubits. In order to realistically utilize a decoder with increasing decoding success rates using increasing lattice size, we would need to have a better time complexity. Luckily, an alternative algorithm called the Peeling Decoder has been developed which can solve errors over the erasure channel with a linear time complexity $\m{O}(n)$ \cite{delfosse2017linear}. The Union-Find Decoder builds on top of the Peeling Decoder to solve for Pauli errors with a time complexity of $\m{O}(n\alpha(n))$, where $\alpha$ is an inverse Ackermann function, which is smaller than 3 for any practical input size \cite{delfosse2017almost}. However, these algorithms have a tradeoff in the form of a decrease in the error threshold, and has the reported value of $p_{UF} = 9.2\%$.

A topic of interest will be weighted growth function for the Union-find decoder. This function of the algorithm will increase the error threshold to $p_{UF} = 9.9\%$, but has not been fully described in its publication.  In this section, we will describe the original Peeling decoder and the Union-Find decoder.   \\

\subsubsection{The Peeling decoder}
Let $\varepsilon \subset E$ be an erasure, a set of qubits on which an erasure error occurs, and let $\sigma \subset S$ be the measured error syndrome, the subset of stabilizer generators which anticommute with the erasure errors. In the absence of Pauli errors, all errors $P$ must lie inside the erasure. Therefore, for any pair of stabilizer generators in $\sigma$, the path of errors must also be in the erasure, which can be denoted by $P\subset \varepsilon$. Furthermore, due to the fact that errors $P$ are randomly distributed, any coset of errors and stabilizers $P\cdot S$ that solves the error syndrome $\sigma$ is the most likely coset. These features of an erasures forms the basis of the Peeling decoder. In order to find a coset of $P \cdot S$, the decoder reduces the size of the erasure by peeling edges from the erasure, while keeping the syndromes at the new boundary of the erasure. Elements of the syndrome can be moved by applying an correction on the adjacent qubit. At the end, the entire erasure is peeled or removed, and all corrections will have removed the errors up to a stabilizer.

\tikzset{
  anyon/.style={circle, fill=OrangeRed, minimum size=.2cm, inner sep=0},
  erasure/.style={NavyBlue, very thick},
  correction/.style={Green, very thick},
  description/.style={align=#1, anchor=west, text width=4cm},
  description/.default={left},
  error/.style={text=black, pos=0.5}
  }
\tikzset{
  anyon/.style={circle, fill=OrangeRed, minimum size=.2cm, inner sep=0},
  erasure/.style={NavyBlue, very thick},
  correction/.style={Green, very thick},
  description/.style={align=#1, text width=4cm},
  description/.default={left},
  error/.style={text=black, pos=0.5}
  }

\begin{figure}
  \centering
  \begin{tikzpicture}[on grid, scale=0.8]
    \node at (0,4) {a)};
    \draw[step=1cm,gray,thin] (0.1,0.1) grid (3.9,3.9);
    \draw[erasure] (1,1) -- (2,1) node[error]{$X$} -- (3,1) -- (3,2) -- (2,2) -- (1,2) -- cycle node[error]{$X$};
    \draw[erasure] (1,2) -- (1,3) -- (2,3) node[error]{$X$} -- (2,2);
    \node[description={center}] at (2, -.5) {initial state};

    \begin{scope}[shift={(6,0)}]
      \node at (0,4) {b)};
      \draw[step=1cm,gray,thin] (0.1,0.1) grid (3.9,3.9);
      \draw[erasure] (1,1) -- (2,1) node[anyon]{} -- (3,1) -- (3,2) -- (2,2) -- (1,2) node[anyon] (a) {} -- cycle;
      \draw[erasure] (a) -- (1,3) node[anyon]{} -- (2,3) node[anyon]{}-- (2,2);
      \node[description={center}] at (2, -.5) {identify syndrome};
    \end{scope}

    \begin{scope}[shift={(12,-.5)}]
      \draw[thin] (0,4) -- ++(.5,0) ++(.5,0) node[anchor=west]{normal edge};
      \draw[thin] (0,3) -- ++(.5,0) node[error]{$X$} ++(.5,0) node[anchor=west]{Pauli error};
      \draw[erasure] (0,2) -- ++(.5,0) ++(.5,0) node[anchor=west, text=black]{erased edge};
      \draw[thin] (0,1) -- ++(.5,0) node[anyon,pos=.5]{} ++(.5,0) node[anchor=west]{syndrome};
      \draw[correction] (0,0)   -- ++(.5,0) ++(.5,0) node[anchor=west,text=black]{correction edge};
    \end{scope}

    \begin{scope}[shift={(0,-6)}]
      \node at (0,4) {c)};
      \draw[step=1cm,gray,thin] (0.1,0.1) grid (3.9,3.9);
      \draw[erasure] (1,3) node[anyon]{} -- (2,3) node[anyon]{} -- (2,2) -- (1,2) node[anyon]{} -- (1,1) -- (2,1) node[anyon]{} -- (3,1) -- (3,2);
      \node[description={center}] at (2, -.5) {construct $F_{\varepsilon}$};
    \end{scope}

    \begin{scope}[shift={(6,-6)}]
      \node at (0,4) {d)};
      \draw[step=1cm,gray,thin] (0.1,0.1) grid (3.9,3.9);
      \draw[erasure] (1,3) node[anyon]{} -- (2,3) node[anyon]{} -- (2,2) -- (1,2) node[anyon]{} -- (1,1) -- (2,1) node[anyon](a){};
      \draw[erasure, dashed] (a) -- (3,1) node[pos=0, below, text=black]{$v$} node[pos=0.5, above]{$e$} node[pos=1, below, text=black]{$u$};
      \node[description={center}] at (2, -.5) {peel $e=(u,v), u \notin \sigma$};
    \end{scope}

    \begin{scope}[shift={(12,-6)}]
      \node at (0,4) {e)};
      \draw[step=1cm,gray,thin] (0.1,0.1) grid (3.9,3.9);
      \draw[erasure] (1,3) node[anyon]{} -- (2,3) node[anyon]{} -- (2,2) -- (1,2) node[anyon]{} -- (1,1);
      \draw[erasure, dashed] (1,1) node[below, text=black]{$v$} -- (2,1) node[anyon]{} node[pos=0.5, above]{$e$} node[below, text=black]{$u$};
      \node[description={center}] at (2, -.5) {peel $e=(u,v), u \in \sigma$};
    \end{scope}

    \begin{scope}[shift={(0,-12)}]
      \node at (0,4) {f)};
      \draw[step=1cm,gray,thin] (0.1,0.1) grid (3.9,3.9);
      \draw[erasure] (1,3) node[anyon]{} -- (2,3) node[anyon]{} --  (2,2) -- (1,2) node[anyon]{} -- (1,1) node[anyon](a){};
      \draw[correction] (a) node[below, text=black]{$v$} -- (2,1) node[pos=0.5, above]{$e$} node[below, text=black]{$u$};
      \node[description={center}] at (2, -.5) {flip $u,v$, add $e$ to $C$};
    \end{scope}

    \begin{scope}[shift={(6,-12)}]
      \node at (0,4) {g)};
      \draw[step=1cm,gray,thin] (0.1,0.1) grid (3.9,3.9);
      \draw[correction] (1,3) -- (2,3) node[error]{$X$} (2,1) -- (1,1) node[error]{$X$} -- (1,2) node[error]{$X$};
      \node[description={center}] at (2, -.5) {apply correction set $C$};
    \end{scope}

    \begin{scope}[shift={(12,-12)}]
      \node at (0,4) {h)};
      \draw[step=1cm,gray,thin] (0.1,0.1) grid (3.9,3.9);
      \node[description={center}] at (2, -.5) {end state};
    \end{scope}
  \end{tikzpicture}
  \caption{Schematic representation of the Peeling decoder. On an erasure $\m{E}\subset E$ (a), there may be some Pauli errors $P\subset \m{E}$ that anticommutes with some stabilizer measurements (b) that is identified as the syndrome $\sigma$. The first step is to construct a spanning forest $F_\m{E}\subset \m{E}$, a fully connected acyclic graph. Next the decoder sequentially removes leaf edges $e=(u,v)$ from the forest that connect to the forest via only one vertex $v$. If $u\in\sigma$ (e), remove $u$ from $\sigma$, flip $v$ in $\sigma$ and the edge $e$ is added to the correction set $C$ (f). If $u \notin \sigma$, move on the the next leaf. After applying the correction set $C$ (g), all errors on the lattice commutes with the stabilizers, potentially solving the error (h).}
\end{figure}

\todo[inline]{use capitals for sets $\Sigma$, and lowercase $\sigma$ for elements in the set}

We will now describe the Peeling decoder as is presented in Algorithm \ref{algo:peel}. In step 1, we will remove all cycles present in $\varepsilon$. We construct a spanning forest $F_\varepsilon$ inside erasure $\varepsilon$, the maximal subsect of edges of $\varepsilon$ that contains no cycles and spans all vertices of $\varepsilon$. From here, we loop over all edges in $F_\varepsilon$ (step 3), starting at a leaf edge $e = \{u,v\}$, removing the leaf edge from $F_\varepsilon$ (step 4), and conditionally add the edge to the correction set $\kappa$ if the pendant vertex $u$ is in $\sigma$ (step 6). If the correction is applied immediately, we can see that the pendant vertex $u$ is removed from $\sigma$ and that the value of $v$ is flipped in $\sigma$. Edges on a branch in the forest will be added to $\kappa$ until $v \in \sigma$, or a generator will be continuously moved from $u$ to $v$ until it encounters another generator, creating a correction path between two syndrome pairs.

\begin{lemma}\label{lem:forestofgraph}
  A forest $F$ is an undirected graph in which any two vertices are connected by at most one path. A forest $F_G$ of some graph $G$ is one in which some edges is removed such that there are no cycles in the graph, which satisfies the forest requirement.
\end{lemma}

\begin{algo}[algotitle=Peeling decoder \cite{delfosse2017linear}, label=algo:peel]
  \begin{algorithm}[H]
    \KwData{A graph $G = (V,E)$, an erasure $\varepsilon \subset E$ and syndrome $\sigma \subset V$}
    \KwResult{Correction set $\kappa \subset E$}
    \BlankLine
    construct a spanning forest $F_\varepsilon$ of $\varepsilon$\;
    initialize $\kappa$ by $\kappa = {\emptyset}$\;
    \While{$F_\varepsilon \neq \emptyset$}{
    pick a leaf edge $e = {u,v}$ with pendant vertex $u$, remove $e$ from $F_\varepsilon$ \;
    \If{$u \in \sigma$}{
      add $e$ to $\kappa$, remove $u$ from $\sigma$ and flip $v$ in $\sigma$}
    \Else{do nothing}
    }
    \KwRet{$\kappa$}
  \end{algorithm}
\end{algo}

\noindent Note that due the flipping of both $u$ and $v$ in $\sigma$, the parity of the number of generators in $\sigma$ is always preserved. The peeling decoder can therefore always solve erasures with an even parity, as the size of $\sigma$ will drop until at the end the syndrome will be empty $\sigma = \emptyset$, and call errors are corrected up to a stabilizer. This is always the case in the absence of measurement and Pauli errors, as all errors within the erasure either add or remove an even number of generators to or from $\sigma$.

\paragraph{Time complexity of the Peeling decoder}
The spanning forest $F_\varepsilon$ can be constructed in linear time. Also, the loop over the forest can be operated in linear if the list of leaves is pre-computed and updated during the loop. Thus the Peeling decoder has a linear time complexity in the size of the erasure $\m{O}(\abs{\varepsilon})$ and therefore also in the number of qubits $\m{O}(n)$.

\subsubsection{Growing erasures}

Now in the presence of Pauli errors, errors can occur on edges that are now not part of the erasure, and odd parity clusters can occur. Clusters that consists from only a single generator also exist, which are just end-points of syndromes caused by Pauli errors. We must therefore make an erasure $\varepsilon$ from the syndrome $\sigma$ that is compatible with the peeling decoder, which contains only even parity clusters. To do this, we can attractively grow the clusters with an odd parity by an half-edge on the boundaries on the clusters. When two odd parity clusters meet, the merged cluster will have a even parity, and can now be solved by the peeling decoder.

\paragraph{Union-Find algorithm}

\begin{tikzpicture}
  \draw[step=1cm,gray,thin] (0.1,0.1) grid (4.9, 4.9);
  \node[fill=OrangeRed, circle, text=white, scale=0.8] (N0) at (1, 1) {1};

\end{tikzpicture}

\todo[inline]{add definition of vertex set}
To keep track of the vertices of a cluster, it will be represented as a \emph{cluster tree}, where an arbitrary vertex of the cluster will be the root, and any other vertex will be a child of the root. Whenever an edge $(u,v)$ is fully grown, we will need to traverse the trees of the two vertices $u$ and $v$, and check whether they have the same root; whether they belong to the same cluster. If not, a merge is initiated by making the root of smaller cluster a child of the bigger cluster. These functions, \codefunc{find} and \codefunc{union} respectively, are part of the Union-Find algorithm (not to be confused with the Union-Find decoder) \cite{tarjan1975efficiency}.

\begin{tikzpicture}
  \draw (2,2) circle (0.2);
  \draw (2,2) -- (0,0);)
\end{tikzpicture}

\todo[inline]{syndrome identification, syndrome validation, peeling decoder}

Within the Union-Find algorithm, two features ensure that the complexity of the algorithm is not quadratic. 1). With \textbf{path compression}, as we traverse a tree from child to parent until we reach the root, we make sure that each vertex encountered that we have encountered along the way is pointed directly to the root. This doubles the cost of the \codefunc{find}, but speeds up any future call to any vertex on the traversed path. 2). With \textbf{weighted union}, we make sure to always make the smaller tree a child of the bigger tree. This ensures that the overall length of the path to the root stays minimal. In order to make this happen, we just need to store the size of the tree at the root.

\paragraph{Data structure}
Now it is clear what information is exactly needed to grow the clusters using the Union-Find algorithm. We will need to store the cluster in a sort of cluster-tree. At the root of each tree we store the size and parity of that cluster in order to facilitate weighted union and to select the odd clusters. We will need to store the state of each edge (empty, half-grown, or fully grown) in a table called \codeword{support}. And we need to keep track of the boundary of each cluster in a \codeword{boundary} list.

\paragraph{The routine}
The full routine of the Union-Find decoder as originally described (\cite{delfosse2017almost}, Algorithm 2) is listed in Algorithm \ref{algo:uf}. In line 1-2, we initialize the data structures, and a list of odd cluster roots $\m{L}$. We will loop over this list until it is empty, or that there are no more odd clusters left.

In each growth iteration, we will need to keep track of which clusters have merged onto one, therefore the fusion list $\m{F}$ is initialized in line 4. We loop over all the edges from the \codeword{boundary} of the clusters from $\m{L}$ in line 5, and grow each edge by an half-edge in \codeword{support}. If an edge is fully grown, it is added to $\m{F}$.

For each edge $(u,v)$ in $\m{F}$, we need to check whether the neighboring vertices belong to different clusters, and merge these clusters if they do. This is done using the Union-Find algorithm in line 6. We call \codefunc{find(u)} and \codefunc{find(v)} to find the cluster roots of the vertices. If they do not have the same root, we make one cluster the child of another by \codefunc{union(u,v)}. Note that this does not only merge two existing clusters, also new vertices, which have themselves as their roots, are added to the cluster this way. We also need to combine the boundary lists of the two clusters.

Finally, we need to update the elements in the cluster list $\m{L}$. First, we replace each element $u$ with its potential new cluster root \codefunc{find(u)} in line 7. We can avoid creating duplicate elements by maintaining an extra look-up table that keeps track of the elements $\m{L}$ at the beginning of each round of growth. In line 8, we update the \codeword{boundary} lists of all the clusters in $\m{L}$, and in line 9, even clusters are removed from the list, preparing it for the next round of growth.

\begin{algo}[algotitle=Union-Find decoder \cite{delfosse2017almost}, label=algo:uf]
  \begin{algorithm}[H]
    \KwData{A graph $G = (V,E)$, an erasure $\varepsilon \subset E$ and syndrome $\sigma \subset V$}
    \KwResult{A grown erasure $\varepsilon'$ such that each cluster $\gamma \subset \varepsilon$ is even}
    \BlankLine
    initialize cluster-trees, support and boundary lists for all clusters \;
    initialize list of odd cluster roots $\m{L}$\;
    \While{$\m{L} \neq \emptyset$}{
    initialize fusion list $\m{F}$ \;
    for all $u \in \m{L}$, grow all edges in the boundary list of cluster $C_u$ by a half-edge in support. If the edge is fully grow, add to fusion list $\m{F}$ \;
    for all $e={u,v} \in \m{F}$, if \emph{find($u$)} $\neq$ \emph{find($v$)}, then apply \emph{union($u,v$)}, append boundary list\;
    for all $u \in \m{L}$, replace $u$ with \emph{find($u$)} without creating duplicate elements\;
    for all $u \in \m{L}$, update the boundary list\;
    remove even clusters from $\m{L}$\;
    }
    run peeling decoder with grown erasure $\varepsilon'$
  \end{algorithm}
\end{algo}

\subsubsection{Time complexity of the Union-Find decoder}


\chapter{Decoders}\label{sec:surface_decoders}
\section{The optimal decoder}\label{sec:optimal_decoder}
\section{Minimum Weight Perfect Matching}\label{sec:MWPMdecoder}

\subsection{Quasiparticle picture}
The processes of error detection and correction can alternatively be presented in the \emph{quasiparticle picture}, where the anticommuting stabilizer measurements act like excitations on the lattice, which behave like the quasiparticles \emph{anyons}. A single error creates a pair of anyons, and a chain of errors causes movement of the anyon on the lattice. A pair of anyons can also annihilate each other when two error chains merge. The correction of errors can thus be viewed of movement of the correction chains until all anyons are annihilated. The quasiparticle picture removes the distracting underlying lattice from the problem, and decoding becomes simply identifying the right pairing between anyons to minimize the chance of a logical error.
\tikzstyle{rednode}=[circle, fill=red!50, minimum size=4]
\tikzstyle{bluenode}=[circle, fill=cyan!50, minimum size=4]
\tikzstyle{redline}=[red!50, line width = 2]
\tikzstyle{blueline}=[cyan!50, line width = 2]

\newcommand{\drawquasigrid}{
  \draw[step=.4cm, opacity=.25] (0,0) grid (4,4);
  \draw (0,0) rectangle (4,4);
  \node[rednode] (N1) at (0.5,0.4) {};
  \node[rednode] (N2) at (2,0.7) {};
  \node[rednode] (N3) at (2.5,1.2) {};
  \node[rednode] (N4) at (3.6,1) {};
  \node[rednode] (N5) at (0.75,2.1) {};
  \node[rednode] (N6) at (1.95,1.8) {};
  \node[bluenode] (N7) at (1.6, 3.4) {};
  \node[bluenode] (N8) at (2.7, 3.5) {};
  \node[bluenode] (N9) at (3.2, 1.5) {};
  \node[bluenode] (N10) at (3.1, 0.6) {};
  \draw[blueline] (N1) to[in=170, out=20] (N2);
  \draw[blueline] (N3) to[in=180, out=-10] (N4);
  \draw[blueline] (N5) to[in=170, out=-20] (N6);
  \draw[redline] (N7) to[in=160, out=15] (N8);
  \draw[redline] (N9) to[in=90, out=250] (N10);
}
\begin{figure}
    \centering
    \begin{tikzpicture}
      \drawquasigrid
      \node at (2, -.5) {\emph{(a)}};
    \end{tikzpicture}
    \hspace{1cm}
    \begin{tikzpicture}
      \drawquasigrid
      \draw[dashed, blueline] (N1) to[out=90, in=270] (N5);
      \draw[dashed, blueline] (N2) to[out=80, in=275] (N6);
      \draw[dashed, blueline] (N3) to[out=10, in=160] (N4);
      \draw[dashed, redline] (N7) to[out=-20, in=195] (N8);
      \draw[dashed, redline] (N9) to[out=290, in=75] (N10);
      \node at (2, -.5) {\emph{(b)}};
    \end{tikzpicture}
    \hspace{1cm}
    \begin{tikzpicture}
      \drawquasigrid
      \draw[yellow, line width=6, opacity=.3] (N3) to[in=180, out=-10] (N4);
      \draw[yellow, line width=6, opacity=.3] (N5) to[in=170, out=-20] (N6);
      \draw[yellow, line width=6, opacity=.3] (N3) to[out=120, in=-30] (N6);
      \draw[yellow, line width=6, opacity=.3] (N5) to[out=200, in=45] (0, 1.75);
      \draw[yellow, line width=6, opacity=.3] (N4) to[out=80, in=225] (4, 1.75);
      \draw[dashed, blueline] (N1) to[out=0, in=200] (N2);
      \draw[dashed, blueline] (N3) to[out=120, in=-30] (N6);
      \draw[dashed, blueline] (N5) to[out=200, in=45] (0, 1.75);
      \draw[dashed, blueline] (N4) to[out=80, in=225] (4, 1.75);
      \draw[dashed, redline] (N7) to[out=-20, in=195] (N8);
      \draw[dashed, redline] (N9) to[out=290, in=75] (N10);
      \node at (2, -.5) {\emph{(c)}};
    \end{tikzpicture}
    \caption{The quasiparticle picture of stabilizer measurements. Anticommuting stabilizers behave as anyons (circles), where a chain of errors (lines) creates a pair of anyons. Figure (b) shows a successful decoding of (a). Figure (c) shows a pairing that resulted in a correction operator that is in a different class as the error operator, which acquires a logical error. (Figure inspired by \cite{naomi})}\label{fig:quasiparticle}
  \end{figure}
  
Figure \ref{fig:quasiparticle}a shows the quasiparticle representation of the errors suffered in Figure \ref{sf:fig_degenerate}a, which has suffered Z (blue lines) and X errors (red lines). The corresponding anyons can either be of the star type (red circle) or plaquette type (blue circle). Figure \ref{fig:quasiparticle}b shows a successful decoding. Note that here not all pairs are correctly identified, but the resulting loop still is in the same class of operators. In figure \ref{fig:quasiparticle}c the correction has failed as the resulting loop in the correction is in a difference class compared to the error. As the loop still commutes with the stabilizer, no error can be detected, but the encoded qubit has acquired a logical error.


\section{Union-Find}\label{sec:UFdecoder}

The Union-Find decoder is a new fast decoding algorithm for topological codes to correct for Pauli errors, erasure errors, and the combination of both errors. The worst-case complexity of the algorithm is $\m{O}(n\alpha(n))$, where $n$ is the number of physical qubits and $\alpha$ is the inverse of Ackermann's function, which is very slowly growing, and is proven that $a(n)\leq 3$ for any practical amount of qubits.

Many types of decoding algorithms have been developed for the surface code, including the optimal decoder and the MWPM-decoder. Most of these decoders run at best in polynomial time, which is often considered efficient, but in practice even quadratic or cubic complexity is likely too slow to correct errors faster than they accumulate in a quantum device. Furthermore, any speed-up of the decoder will indirectly lead to a reduction of the noise strength, as a shorter time between two rounds of correction allows for fewer errors to appear. To this end, a new decoding algorithm named the \emph{Peeling decoder} has been developed that can solve errors over the erasure channel with a linear time complexity. The \emph{Union-Find} decoder is an extensions that additionally solves for Pauli errors. We will explore both algorithms in the coming sections and perform analyses on their complexities. 

\subsection{The Peeling decoder}
he Peeling decoder acquired its name by the nature of its behavior of sequentially \emph{peeling} from some tree of qubit-edges until the correction operator is left \cite{delfosse2017linear}. The scope of this decoder is limited to \emph{erasure} errors, or errors suffered through the erasure channel. Recall from equation \ref{qec:eq:erasure} that in an erasure, each qubit is erased from the system independently with probability $p_E$. Such a loss can be detected and the missing qubit is replaced by a totally mixed state of equation \ref{qec:eq:mixstate}, which can be interpreted as the original state that suffers from a Pauli error $I$, $X$, $Y$ or $Z$ chosen uniformly at random. Due to this uniform distribution, the primal and dual lattices of the surface code can be decoded separately from each other. The erased edges of graph $G_v(V,E_v)$ and $G_f(V,E_f)$ (see section \ref{sec:toricgraph}) suffer uniformly distributed $\{I,X\}$ and $\{I,Z\}$, respectively. We will consider $G_v(V,E_v)$ and simplify its notation to $G(V,E)$. 

We describe the decoding process of the primal lattice $G(V,E)$ and denote the set of lost qubits by $\gls{erasure}\subseteq E$. The set of Pauli errors within the erasure is $P_\m{E} = \{X_1,...X_m\}$. Error detection is performed in the same way as Pauli errors; by measuring the set of star operators or vertices $V$, which returns a set of nontrivial syndrome measurements $\sigma \subseteq V$. The decoder of an erasure error is thus provided with the extra information $\m{E}$, the subset of erased qubit locations, next to $\sigma$. 
\begin{lemma}\label{lem:peelinguni}
  For an erasure $\m{E} \subseteq E$ with uniformly distributed Pauli errors $P_\m{E}$ on a surface code, and a measured syndrome $\sigma$, any error  $\tilde{P}_\m{E}$ that produces $\sigma$ in a measurement is the most likely set of errors. 
\end{lemma}
\begin{proof}
  For the coset $\tilde{P}_\m{E}\cdot S$, where $\tilde{P}_\m{E}$ is some Pauli error caused by an erasure and $S$ is a set of stabilizers that act trivially on the codespace, the most likely configuration is the one that maximizes probability $\mathbb{P}(\tilde{P}_\m{E}\cdot S|\m{E},\sigma)$, where $\m{E}$ and $\sigma$ are known. This probability is proportional to $|\tilde{P}_\m{E}\cdot S \cap\m{E}|$. But since all qubits in $\m{E}$ suffer a Pauli error and this error is uniformly distributed, all configurations of $\tilde{P}_\m{E}$ have equal probability. 
  % Let $\m{E}\subseteq E$ be an erasure, a set of qubits on which an erasure error occurs, and let $\sigma \subseteq \m{S}$ be the measured error syndrome, the subset of the stabilizer generator group $\m{S}$, that anticommute with the erasure errors. In the absence of Pauli errors, all errors $P$ must lie inside the erasure.Therefore, for any measured syndrome, the path of errors must also be in the erasure, which can be denoted by $P\subseteq \m{E}$. As a result of this, in the case that the Pauli errors within the erasure are uniformly distributed, any error $\tilde{P}\subseteq\m{E}$ that produces the syndrome $\sigma$ is the most likely coset. 
\end{proof}

Consequently, if the error set $\tilde{P}_\m{E}$ is applied as the correction operator $C=\tilde{P}_\m{E}$, the resulting decoder is a \emph{maximum likelihood decoder}. In order to find $C$, we now do not try to find paths within $\m{E}$ that pair the syndrome vertices of $\sigma$, but rather try to recursively shrink the set of edges on which a decision is to be made. 
\begin{definition}\label{def:boundaryofedges}
  The boundary or support of a set of edges $\gls{boundary}(\tilde{E})$ denotes the set of vertices $\tilde{V}$ that supports all edges $e\in \tilde{E}$.
\end{definition}
\begin{definition}\label{def:forest}
  A spanning forest $F_{\tilde{E}}$ is a maximal subset of edges of $\tilde{E}$ that contains no cycles and $\mathscr{B}(F_{\tilde{E}}) = \mathscr{B}(\tilde{E})$.
\end{definition}
The first step in this is to produce $\gls{forest}$ inside $\m{E}$, where all syndrome vertices $\sigma$ are included in the support. Hence if $\m{E}$ is a connected graph, then $F_\m{E}$ is connected \emph{acyclic} graph. Such a forest can be found in linear time by either a depth-first search or breadth-first search of the $\m{E}$. Next, the decoder further reduces the size of the spanning forest $F_\m{E}$ by sequentially peeling edges from the tree, while constructing the correction set $\gls{correctionset}$, initiated as an empty set. The decoder loops over all edges in $F_\m{E}$, each time picking a \emph{leaf} edge $e = \{u,v\}$, connected to the forest by only one vertex $v$, removing the leaf edge from $F_\m{E}$. If the so-called \emph{pendant} vertex $u\in\sigma$, remove $u$ from $\sigma$, and $e$ is added to $\m{C}$ and the vertex $v$ is \emph{flipped}, such that $v$ is added to $\sigma$ if $v \notin \sigma$, and removed from $\sigma$ if $v\in\sigma$.  If $u\notin\sigma$, the edge $e$ is simply removed from $F_\m{E}$ (see algorithm \ref{algo:peel}). On account of these rules, edges on a branch that had a syndrome vertex as a leaf will continuously be added to $\m{C}$ until it encounters another syndrome vertex, creating a correction path between a syndrome pair. The dual lattice can be decoded in a similar way.
\begin{definition}
  The \emph{Pauli product} of a set of edges $\tilde{E}$ is the defined as the product of Pauli operators on each of the edges in the set
  \begin{equation}
    \gls{pauliproduct}(\tilde{E}) = \prod_{e\in \tilde{E}} \hat{P}_e,
  \end{equation}
  where the Pauli operator $\hat{P}$ corresponds to $X$ if $\tilde{E}\subseteq E_v$ and $Z$ if $\tilde{E}\subseteq E_f$. 
\end{definition}
\begin{algo}[algotitle=Peeling decoder (adapted from \cite{delfosse2017linear}), label=algo:peel]
  \begin{algorithm}[H]
    \KwData{A graph $G = (V,E)$, an erasure $\m{E} \subseteq E$ and syndrome $\sigma \subseteq V$}
    \KwResult{Correction $C$}
    \BlankLine
    construct a spanning forest $F_\m{E} \subseteq\m{E}$\;
    initialize $\m{C}$ by $\m{C} = {\emptyset}$\;
    \While{$F_\m{E} \neq \emptyset$}{
    pick a leaf edge $e = {u,v}$ with pendant vertex $u$, remove $e$ from $F_\m{E}$ \;
    \If{$u \in \sigma$}{
      add $e$ to $\m{C}$, remove $u$ from $\sigma$ and flip $v$ in $\sigma$}
    \Else{do nothing}
    }
    \KwRet{$C = \prod_{e\in \m{C}} X_e$}
  \end{algorithm}
\end{algo}
The spanning forest $F_\m{E}$ can be constructed in linear time. Also, the loop over the forest can be operated in linear if the list of leaves is pre-computed and updated during the loop. Thus the Peeling decoder has a linear time complexity in the size of the erasure $\m{O}(\abs{\m{E}})$ and therefore also in the number of qubits $\m{O}(n)$. Now, the structure of the forest $F_\m{E}$ is very dependent on from which vertex the DFS is started, and proof is required that any forest of $\m{E}$ is valid. Also, we show that for each forest, the peeling process returns the same correction.

\begin{lemma}\label{lem:anyforest}
  For any choice of $F_\m{E}$, there exists a subset $\m{C}\subseteq F_\m{E}$ such that $\mathscr{P}(\m{C})$ corrects the syndrome set $\sigma$.
\end{lemma}
\begin{proof}
  There exists a subset of edges $\tilde{E} = \{e_1,...,e_m\} \subseteq F_\m{E}$ such that $\mathscr{P}(\tilde{E})$ has a syndrome $\sigma$. By the definition of the forest $F_\m{E}$, adding another edge $e' \in F_\m{E} \vartriangle \m{E}$ creates a cycle $\gamma' \subseteq F_\m{E} \cup \{e_i\}$, where $\vartriangle$ denotes the symmetric difference between two sets. Now $\tilde{E}$ can be replaced by $\tilde{E}'=\tilde{E}\vartriangle\gamma'$ whose Pauli product $\mathscr{P}(\tilde{E}')$ has the same syndrome $\sigma$, as $\vartriangle$ augments the matching path between syndromes within $\gamma'$. Now, any edge $e_r\in \gamma' \notin \tilde{E}'$ can be removed from to create a new forest $F_\m{E}'=F_\m{E} \cup \{e_i\}\setminus e_r|e_r \neq e'$. For any cycle that exists from larger than 3 elements, $e_r$ must exist. Thus the Pauli product of subset $\tilde{E}' \subseteq F_\m{E}'\subseteq \m{E}$ is also a valid error with syndrome $\sigma$, and $\mathscr{P}(\tilde{E}')$ corrects $\sigma$. This can be done any number of times, thus every $F_\m{E}$ is valid.   
\end{proof}
\begin{lemma}\label{lem:peelingfe}
  The output correction set $\m{C}$ from the Peeling decoder is only dependent on spanning tree $F_\m{E}\subseteq\m{E}$, and not on the peeling process. 
\end{lemma}
\begin{proof}
   For each $F_\m{E}$, the outcome $\m{C}$ after peeling is unique and independent from the order of peeling. If there exists two subsets $\m{C}$ and $\m{C}'$, applying both $\mathscr{P}(\m{C})\mathscr{P}(\m{C}')$ yields the empty syndrome set $\sigma=\emptyset$, which means that either $\mathscr{P}(\m{C})\mathscr{P}(\m{C}')$ is a cycle or $\m{C}=\m{C}'$. Since $F_\m{E}$ has no cycles it means that $m{C}$ must be unique within $F_\m{E}$.
\end{proof}

\tikzset{
  anyon/.style={circle, fill=OrangeRed, minimum size=.2cm, inner sep=0},
  erasure/.style={NavyBlue, very thick},
  correction/.style={Green, very thick},
  description/.style={align=#1, text width=4cm},
  description/.default={left},
  error/.style={text=black, pos=0.5}
  }

\begin{figure}
  \centering
  \begin{tikzpicture}[on grid, scale=0.8]
    \node at (0,4) {a)};
    \draw[step=1cm,gray,thin] (0.1,0.1) grid (3.9,3.9);
    \draw[erasure] (1,1) -- (2,1) node[error]{$X$} -- (3,1) -- (3,2) -- (2,2) -- (1,2) -- cycle node[error]{$X$};
    \draw[erasure] (1,2) -- (1,3) -- (2,3) node[error]{$X$} -- (2,2);
    \node[description={center}] at (2, -.5) {initial state};

    \begin{scope}[shift={(6,0)}]
      \node at (0,4) {b)};
      \draw[step=1cm,gray,thin] (0.1,0.1) grid (3.9,3.9);
      \draw[erasure] (1,1) -- (2,1) node[anyon]{} -- (3,1) -- (3,2) -- (2,2) -- (1,2) node[anyon] (a) {} -- cycle;
      \draw[erasure] (a) -- (1,3) node[anyon]{} -- (2,3) node[anyon]{}-- (2,2);
      \node[description={center}] at (2, -.5) {identify syndrome};
    \end{scope}

    \begin{scope}[shift={(12,-.5)}]
      \draw[thin] (0,4) -- ++(.5,0) ++(.5,0) node[anchor=west]{normal edge};
      \draw[thin] (0,3) -- ++(.5,0) node[error]{$X$} ++(.5,0) node[anchor=west]{Pauli error};
      \draw[erasure] (0,2) -- ++(.5,0) ++(.5,0) node[anchor=west, text=black]{erased edge};
      \draw[thin] (0,1) -- ++(.5,0) node[anyon,pos=.5]{} ++(.5,0) node[anchor=west]{syndrome};
      \draw[correction] (0,0)   -- ++(.5,0) ++(.5,0) node[anchor=west,text=black]{correction edge};
    \end{scope}

    \begin{scope}[shift={(0,-6)}]
      \node at (0,4) {c)};
      \draw[step=1cm,gray,thin] (0.1,0.1) grid (3.9,3.9);
      \draw[erasure] (1,3) node[anyon]{} -- (2,3) node[anyon]{} -- (2,2) -- (1,2) node[anyon]{} -- (1,1) -- (2,1) node[anyon]{} -- (3,1) -- (3,2);
      \node[description={center}] at (2, -.5) {construct $F_{\varepsilon}$};
    \end{scope}

    \begin{scope}[shift={(6,-6)}]
      \node at (0,4) {d)};
      \draw[step=1cm,gray,thin] (0.1,0.1) grid (3.9,3.9);
      \draw[erasure] (1,3) node[anyon]{} -- (2,3) node[anyon]{} -- (2,2) -- (1,2) node[anyon]{} -- (1,1) -- (2,1) node[anyon](a){};
      \draw[erasure, dashed] (a) -- (3,1) node[pos=0, below, text=black]{$v$} node[pos=0.5, above]{$e$} node[pos=1, below, text=black]{$u$};
      \node[description={center}] at (2, -.5) {peel $e=(u,v), u \notin \sigma$};
    \end{scope}

    \begin{scope}[shift={(12,-6)}]
      \node at (0,4) {e)};
      \draw[step=1cm,gray,thin] (0.1,0.1) grid (3.9,3.9);
      \draw[erasure] (1,3) node[anyon]{} -- (2,3) node[anyon]{} -- (2,2) -- (1,2) node[anyon]{} -- (1,1);
      \draw[erasure, dashed] (1,1) node[below, text=black]{$v$} -- (2,1) node[anyon]{} node[pos=0.5, above]{$e$} node[below, text=black]{$u$};
      \node[description={center}] at (2, -.5) {peel $e=(u,v), u \in \sigma$};
    \end{scope}

    \begin{scope}[shift={(0,-12)}]
      \node at (0,4) {f)};
      \draw[step=1cm,gray,thin] (0.1,0.1) grid (3.9,3.9);
      \draw[erasure] (1,3) node[anyon]{} -- (2,3) node[anyon]{} --  (2,2) -- (1,2) node[anyon]{} -- (1,1) node[anyon](a){};
      \draw[correction] (a) node[below, text=black]{$v$} -- (2,1) node[pos=0.5, above]{$e$} node[below, text=black]{$u$};
      \node[description={center}] at (2, -.5) {flip $u,v$, add $e$ to $C$};
    \end{scope}

    \begin{scope}[shift={(6,-12)}]
      \node at (0,4) {g)};
      \draw[step=1cm,gray,thin] (0.1,0.1) grid (3.9,3.9);
      \draw[correction] (1,3) -- (2,3) node[error]{$X$} (2,1) -- (1,1) node[error]{$X$} -- (1,2) node[error]{$X$};
      \node[description={center}] at (2, -.5) {apply correction set $C$};
    \end{scope}

    \begin{scope}[shift={(12,-12)}]
      \node at (0,4) {h)};
      \draw[step=1cm,gray,thin] (0.1,0.1) grid (3.9,3.9);
      \node[description={center}] at (2, -.5) {end state};
    \end{scope}
  \end{tikzpicture}
  \caption{Schematic representation of the Peeling decoder. On an erasure $\m{E}\subset E$ (a), there may be some Pauli errors $P\subset \m{E}$ that anticommutes with some stabilizer measurements (b) that is identified as the syndrome $\sigma$. The first step is to construct a spanning forest $F_\m{E}\subset \m{E}$, a fully connected acyclic graph. Next the decoder sequentially removes leaf edges $e=(u,v)$ from the forest that connect to the forest via only one vertex $v$. If $u\in\sigma$ (e), remove $u$ from $\sigma$, flip $v$ in $\sigma$ and the edge $e$ is added to the correction set $C$ (f). If $u \notin \sigma$, move on the the next leaf. After applying the correction set $C$ (g), all errors on the lattice commutes with the stabilizers, potentially solving the error (h).}
\end{figure}

The peeling decoder will always output some correction $C$ such that $CP_\m{E}$ commutes with the stabilizer, given a set of nontrivial stabilizer measurements $\sigma$. This is proven by the previous lemmas. Finally, we will prove that this is true for any erasure $\m{E}\subseteq E$. 
\begin{theorem}\label{eq:anyevenparity}
  For any connected graph that suffered some erasure $\m{E}\subseteq E$ with pauli error $P_\m{E}$, if the parity of the number syndrome vertices within the graph is even, applying the Peeling decoder (algorithm \ref{algo:peel}) will produce a correction $\m{C}\subseteq \m{E}$ such that $\mathscr{P}(\m{C})P_\m{E}=CP_\m{E}$ commutes with the stabilizer.
\end{theorem}
\begin{proof}
  Consider a spanning forest $F_\m{E}$ containing $n_\sigma$ syndrome vertices. The forest is being stripped by the Peeling decoder on the leaf edge $e = (u,v)$, where the vertex $v$ is the pendant vertex. If $u\notin\sigma$, $e$ is simply removed from $F_\m{E}$ and $n_\sigma$ is unaltered. If $u\in\sigma$, $u$ is removed from $\sigma$ such that $n_\sigma'= n_\sigma -1$. Vertex $v$ is now flipped in $\sigma$, meaning that if $v\in\sigma$, it is removed and $n_\sigma'= n_\sigma -2$, or if  $v\notin\sigma$, it is added and $n_\sigma'= n_\sigma$. After peeling it must be that $n_\sigma=0$, from which follows that all erasures with \emph{even} parity can be solved. 
\end{proof}

\begin{definition}\label{def:cluster}
  A cluster $C$ is a subset of an erasure $\m{E}$, such that the edges of $C$ form a connected graph. 
\end{definition}
\begin{lemma}\label{lem:singlecluster}
  An edge $e$ can only belong to a single cluster. A vertex $v$ can only be in the boundary of a single cluster $\mathscr{B}(C)$. 
\end{lemma}
\begin{proof}
  If there exists some edge $e$ that belongs to two clusters $C_i, C_j$, they are connected via $e$. Per definition \ref{def:cluster} clusters $C_i, C_j$ must be a single cluster. The same is true for some vertex $v$ that belongs to both $\mathscr{B}(C_i)$ and $\mathscr{B}(C_j)$. 
\end{proof}
For only erasure noise, all erasures must have even parity as all errors $P_\m{E}$ and thus all syndromes $\sigma \in \mathscr{B}(\m{E})$, which is why erasure noise is the scope of the Peeling decoder. As other types of noise are added, modifications to the Peeling decoder are needed, as we will see later. Finally, we note that an erasure $\m{E}$ may not be a single subset of connected edges, but rather many connected subsets, denoted by $C_i$. This does not change anything in the decoder, as the peeling algorithm will pass every edge once, which is ensured by lemma \ref{lem:singlecluster}. Each cluster $C_i$ returns its own connected acyclic $F_{C_i}$ that can be peeled separately from each other. 
\begin{theorem}
  The Peeling decoder (algorithm \ref{algo:peel}) is a linear-time maximum likelihood decoder for erasures up to $d-1$ qubits, where $d$ is the minimum distance of the code. 
\end{theorem}
\begin{proof}
  If the erasure $\m{C}$ does not support a subset of edges that is equivalent to some logical operator $L=\mathscr{P}(\tilde{E})|\tilde{E}\subseteq\m{E}$, the product of the error and correction $CP_\m{E}$ cannot lead to a logical error, as $\m{C}\subseteq \m{E}$. Furthermore, on account of lemmas \ref{lem:peelinguni}, \ref{lem:anyforest} and \ref{lem:peelingfe}, any correction set $\m{C}\subseteq F_\m{E}$ is the most likely correction. Thus for any erasure pattern up to $d-1$ qubits, the Peeling decoder is a linear-time maximum likelihood decoder. 
\end{proof}

\subsubsection{Bounded surfaces}
For bounded surfaces such as the planar code (sec \ref{sec:surface_planar}), the peeling decoder needs some small alterations. Let us again only consider the primal lattice which is now denoted by $G = (V_\iota\cup V_{\delta} \cup V_{\omega}, E_\iota \cup E_{\delta})$. Syndrome measurements on such a graph are only supported by $\sigma \subseteq V_\iota\cup V_\delta$, as $V_\delta$ are \emph{open} vertices that only exist to support boundary edges $E_\delta$, and do not refer to some stabilizer generator or physical measurement. The missing information on $V_\delta$ makes it impossible to apply the pendant vertex rule at these vertices. To ensure that the peeling algorithm does not become stuck, we add the restriction for the pendant vertex $u \notin V_\delta$. Furthermore, the construction of the forest $F_\m{E}$ requires an additional alteration.
\begin{lemma}
  Two vertices $u,v$ within a forest $F_\m{E}$ that satisfy $u\in V_\delta, v \in V_\delta$ is equivalent to a cycle in $F_\m{E}$. 
\end{lemma}
\begin{proof}
  If there are an even number of vertices in a forest $F_\m{E}$ that are supported by $V_\delta$, it means that there are a number of unique paths within $F_\m{E}$ that lead from a element of $V_\delta$ to another element of $V_\delta$. Such a path is equivalent to some $\delta$-operators and commutes with the stabilizer. Hence, it cannot be caused by some detected error which anticommutes with the stabilizer.
\end{proof}
Due to this, we ensure that each forest $F_\m{E}$ can only support a maximum of 1 element of $V_\delta$. The forests are grown starting from vertices of the set $V_\delta$, and the algorithm is completed by a DFS same as before with the additional requirement. Note that now for every cluster, more than one connected acyclic forests may be formed, dependent on the number edges connected to the boundary. But yet every forest can still be peeled independently, as every edge can only be peeled once. 

\begin{algo}[algotitle=Peeling decoder for bounded surfaces (adapted from \cite{delfosse2017linear}), label=algo:peelbound]
  \begin{algorithm}[H]
    \KwData{A graph $G = (V\cup V_\delta,E)$, an erasure $\m{E} \subseteq E$ and syndrome $\sigma \subseteq V$}
    \KwResult{Correction $C$}
    \BlankLine
    construct a spanning forest $F_\m{E}\subseteq\m{E}$ with seed $V_\delta$\;
    initialize $\m{C}$ by $\m{C} = {\emptyset}$\;
    \While{$F_\m{E} \neq \emptyset$}{
    pick a leaf edge $e = {u,v}$ with pendant vertex $u\notin V_\delta$, remove $e$ from $F_\m{E}$ \;
    \If{$u \in \sigma$}{
      add $e$ to $\m{C}$, remove $u$ from $\sigma$ and flip $v$ in $\sigma$}
    \Else{do nothing}
    }
    \KwRet{$C = \prod_{e\in \m{C}} X_e$}
  \end{algorithm}
\end{algo}

\begin{figure}
    \centering
    \begin{tikzpicture}[on grid, scale=0.8]
      \node at (-.5,4) {\emph{(a)}};
      \draw[step=1cm,gray,thin] (0.1,0) grid (3.9,4);
      \draw[erasure] (0.1,1) -- (1,1)  -- (2,1)  node[error]{$X$} -- (2,2) -- (2,3) -- (1,3)  -- (1,2) node[error]{$X$}  -- (0.1,2)  node[error]{$X$} (1,1) -- (1,2) -- (2,2);
      \draw[erasure] (3.9,4) -- (3,4)  node[error]{$X$} -- (3,3)  -- (3,2)  node[error]{$X$}  -- (3.9,2) (2,3) -- (3,3) -- (3.9,3);
      \node[description={center}] at (2, -.5) {initial state};
  
      \begin{scope}[shift={(6,0)}]
        \node at (-.5,4) {\emph{(b)}};
        \draw[step=1cm,gray,thin] (0.1,0) grid (3.9,4);
        \draw[erasure] (0.1,1) -- (1,1) node[anyon]{} -- (2,1) node[anyon]{} -- (2,2) -- (2,3) -- (1,3) node[anyon]{} -- (1,2)  -- (0.1,2) (1,1) -- (1,2) -- (2,2);
        \draw[erasure] (3.9,4) -- (3,4) node[anyon]{} -- (3,3) node[anyon]{} -- (3,2) node[anyon]{}  -- (3.9,2) (2,3) -- (3,3) -- (3.9,3);
        \node[description={center}] at (2, -.5) {identify syndrome};
      \end{scope}
  
      \begin{scope}[shift={(12,-.5)}]
        \draw[thin] (0,4) -- ++(.5,0) ++(.5,0) node[anchor=west]{normal edge};
        \draw[thin] (0,3) -- ++(.5,0) node[error]{$X$} ++(.5,0) node[anchor=west]{Pauli error};
        \draw[erasure] (0,2) -- ++(.5,0) ++(.5,0) node[anchor=west, text=black]{erased edge};
        \draw[thin] (0,1) -- ++(.5,0) node[anyon,pos=.5]{} ++(.5,0) node[anchor=west]{syndrome};
        \draw[correction] (0,0)   -- ++(.5,0) ++(.5,0) node[anchor=west,text=black]{correction edge};
      \end{scope}

      \begin{scope}[shift={(0,-6)}]
        \node at (-.5,4) {\emph{(c)}};
        \draw[step=1cm,gray,thin] (0.1,0) grid (3.9,4);
        \draw[erasure] (0.1, 2) -- (1,2) -- (1,1) node[anyon]{} -- (2,1) node[anyon]{} -- (2,2) -- (2,3) -- (1,3) node[anyon]{};
        \draw[erasure] (3.9,4) -- (3,4) node[anyon]{} -- (3,3) node[anyon]{} -- (3,2) node[anyon]{};
        \node[description={center}] at (2, -.5) {construct $\m{T}_\m{R}$};
      \end{scope}

      \begin{scope}[shift={(6,-6)}]
        \node at (-.5,4) {\emph{(d)}};
        \draw[step=1cm,gray,thin] (0.1,0) grid (3.9,4);
        \draw[correction] (0.1, 2) -- (1,2) node[error]{$X$} -- (1,1) node[error]{$X$} (2,1)  -- (2,2) node[error]{$X$} -- (2,3) node[error]{$X$} -- (1,3) node[error]{$X$};
        \draw[correction] (3.9,4) -- (3,4) node[error]{$X$} (3,3) -- (3,2) node[error]{$X$};
        \node[description={center}] at (2, -.5) {apply correction $\n{P}(\m{C})$};
      \end{scope}

      \begin{scope}[shift={(12,-6)}]
        \node at (-.5,4) {\emph{(e)}};
        \draw[step=1cm,gray,thin] (0.1,0) grid (3.9,4);
        \path (1,1) -- (2,1) node[error]{$X$} -- (2,2) node[error]{$X$} -- (2,3) node[error]{$X$} -- (1,3) node[error]{$X$} -- (1,2) node[error]{$X$} -- (1,1) node[error]{$X$};
        \node[description={center}] at (2, -.5) {end state};
      \end{scope}

    \end{tikzpicture}
    \caption{Schematic visualization of the Peeling decoder on a surface with boundaries. On an erasure $\m{R}\subset \m{E}$ (a), there may be some Pauli errors $\m{E}_\m{R} \subset \m{R}$ that anticommutes with some stabilizer measurements (b) that is identified as the syndrome $\sigma$. (c) The forest $\m{T}_\m{R}$ now has the extra constriction that it can only support single element of $\m{V}_\delta$, the open vertices, and peeling is only allowed on pendant vertices $v\notin \m{V}_\delta$. (d) After peeling, the correction $\n{P}(\m{C})$ is outputted and can be applied to correct error. (e) The end state is now a cycle of errors, which commutes with the stabilizer. This is not a feature of the Peeling decoder, but is just an example.}
  \end{figure}




\subsection{Union-Find data structure}
The Union-Find data structure, also known as the \emph{disjoint-set} data structure \cite{tarjan1975efficiency}, is the data structure that is crucial for the near linear complexity of the Union-Find decoder. In this section, we describe the data structure and analyse its complexity. 

The disjoint-set data structure consists of two types of instructions for manipulating a set of elements partitioned into a number of disjoint subsets, which are non-overlapping. These instructions provide near-constant-time operations to determine whether elements are in the same set, and to merge sets if that is not the case. 
\begin{itemize}
  \item $\codefunc{Find}(x)$ computes the unique set containing the element $x$
  \item $\codefunc{Union}(A,B)$ combines sets $A$ and $B$ into a new combined set
\end{itemize}
We first show that a very simple implementation runs in $\m{O}(n^2)$ time. Applying the \emph{union by weight} rule, this is reduced to $\m{O}(n \log n)$. Combined with the \emph{Path compression} rule, the overall complexity is $\m{O}(n \alpha (n))$, where $\alpha$ is the inverse of Ackermann's function. The original proof \cite{tarjan1975efficiency} was further reinforced by subsequent publications \cite{tarjan1979class,tarjan1984worst} and was proven to be optimal \cite{fredman1989cell}. In this thesis, we will follow a simplified approach, where the \emph{union by weight} rule is replaced by a different by similar \emph{union by rank} rule \cite{kozen1992design}. Later Tarjan and also other sources used this to cast this analysis in terms of potential functions \cite{tarjan1999handout, harfst2000potential, cormen2009introduction}. Another approach is provided by \cite{seidel2005top}, where the upper bound is computed via a top-down method. 


\subsubsection{Simple implementation}
A simple implementation of the instructions would be to store for each set $X$ a list $l_X$ of all the elements it contains, and a pointer to $l_X$ at all elements $x\in l_X$, which is called the \emph{linked-list} representation. The function $\codefunc{find}(x)$ simply returns the list $l_X$, which represents the set. After a call to $\codefunc{Union}(X, Y)$, all elements of $l_Y$ are appended to $l_X$, and the pointers of all elements $y\in Y$ are updated. This takes $\m{O}(m)$ operations, where $m=|l_Y|$ the number of elements. In a system of $n$ elements, there may be a maximum of $n-1$ calls to \codefunc{Union}. As the $i$\textsuperscript{th} call involves sets $|l_X|=1$ and $|l_Y|=i$, the number of updates is 
\begin{equation}
  \sum_{i=1}^{n-1} i = \m{O}(n^2),
\end{equation}
where each \codefunc{Union} operation has on average $\m{O}(n)$ updates. 

\subsubsection{Union by weight}
In the above case, we assume in each \codefunc{Union} the worst case where a larger list is appended to a smaller list, and the pointers of the largest list must be updated. Suppose that each list $l_X$ additionally stores the length of the list itself, which can be easily maintained, it can be set as a rule to always append the smaller list to the larger list. With this simple rule, the complexity of the Union-Find algorithm can be reduced to $\m{O}(n \log n)$. 

\begin{theorem}
  A sequence of \codefunc{Union} and \codefunc{Find} operations on $n$ initial single element disjoint sets using the weighted union rule takes $\m{O}(n \log n)$ time. 
\end{theorem}
\begin{proof}
  Consider a set $X$ with list $l_X$ that undergoes a sequence of \codefunc{Union} and \codefunc{Find} operations. The first time all pointers in $l_X$ are updated, the resulting set must have at least 2 elements. Similarly, the next time all pointers are updated, the resulting set must have at least 4 elements due to the weighted union rule. With system size $n$, any initial set $X$ can be updated at most $\log n$ times, where the update size is proportional to $n$. Hence the sequence takes $\m{O}(n \log n)$ time. 
\end{proof}

\subsubsection{Disjoint-set trees}
The pointer update in each set during a \codefunc{Union} can be minimized by a different representation of the Union-Find data structure, where each set is represented by a \emph{rooted tree}. Each member has a pointer to its parent, and the root of the tree points to itself, and is the representative element of the cluster. Function \codefunc{Find} is now performed by following the parent pointers until the root of the tree, and $\codefunc{Union}$ points one root towards another, such that there is only one single root that is the representative element. Each call to $\codefunc{Union}(r_x,r_y)$ is preceded by calls to $r_x=\codefunc{Find}(x)$ and $r_y=\codefunc{Find}(y)$ to find their respective roots and to link their trees if $r_x \neq r_y$. 

\tikzstyle{lw1}=[line width=1]
\tikzstyle{fnode}=[draw, circle, minimum size=0.65cm, inner sep=0, lw1]

\begin{figure}[]
  \centering
  \begin{tikzpicture}[on grid]
    \node[fnode] at (0,0)(a){a};
    \node[fnode] at (0,1)(b){b};
    \node[fnode] at (0.75,2)(c){c};
    \node[fnode] at (1.5,1)(d){d};
    \node[fnode] at (3,0)(e){e};
    \node[fnode] at (3,1)(f){f};
    \node[fnode] at (3,2)(g){g};
    \draw[lw1, ->] (a) -- (b);
    \draw[lw1, ->] (b) -- (c);
    \draw[lw1, ->] (d) -- (c);
    \draw[lw1, ->] (e) -- (f);
    \draw[lw1, ->] (f) -- (g);
    \draw[lw1, ->] (c) .. controls +(0.75,1) and +(-0.75,1) .. (c);
    \node at (1.5,-1) {(a)};

    \begin{scope}[shift={(6,-1)}]
      \node at (1.5,0) {(b)};
      \node[fnode] at (0,0)(a2){a};
      \node[fnode] at (0,1)(b2){b};
      \node[fnode] at (0.75,2)(c2){c};
      \node[fnode] at (1.5,1)(d2){d};
      \node[fnode] at (2.75,1)(e2){e};
      \node[fnode] at (2.75,2)(f2){f};
      \node[fnode] at (1.75,3)(g2){g};
      \draw[lw1, ->] (a2) -- (b2);
      \draw[lw1, ->] (b2) -- (c2);
      \draw[lw1, ->] (d2) -- (c2);
      \draw[lw1, ->] (e2) -- (f2);
      \draw[lw1, ->] (f2) -- (g2);
      \draw[lw1, ->] (c2) -- (g2);
      \draw[lw1, ->] (g2) .. controls +(0.75,1) and +(-0.75,1) .. (g2);
    \end{scope}
  \end{tikzpicture}
  \caption{<caption>}
  \label{<label>}
\end{figure}

\subsubsection{Union by rank}

The \emph{union by weight} rule from the previous section can be implemented without any alteration. However, in the disjoint-set tree structure, a slightly different rule, \emph{union by rank}, can be implemented behaves similarly, but allows for a easier analysis of the complexity when combined with the \emph{path compression} rule, which is introduced in the next section. 

\begin{definition}
  The rank of a node $x$ is the maximum height of a subtree rooted in $x$. Let $R_t(x)$ denote the rank of node $x$ at time $t$. 
\end{definition}

The union by rank rule is then as follows: to apply $\codefunc{Union}(r_x,r_y)$, make the smaller-ranked root $R(r_x) < R(r_y)$ point to the larger; in case of a tie, also increase the rank of the new root by one. As long as a node $x$ has no parent, its rank can increase as other trees as pointed to itself. But as $x$ becomes a child of node $x$, the subtree rooted at $x$ become fixed and so does its rank. Any descendent $x$ of ancestor $y$ must thus always comply to $R(x)<R(y)$. 

\begin{lemma}\label{lem:rank1}
  Let $T_t(x)$ be the subtree rooted in node $x$ at time $t$, and let $|T_t(x)|$ denote the number of elements in the subtree, then if the weight by rank rule is implemented, it must be true that 
  \begin{equation}
    \abs{T_t(x)} \geq 2^{R_t(x)}
  \end{equation}
\end{lemma}
\begin{proof}
    \begin{equation*}
        R_t(x) = R_{t+1}(x) = R_{t+1}(y) - 1
    \end{equation*}
    \begin{equation*}
        \abs{T_{t+1}(y)} \geq abs{T_{t+1}(x)}
    \end{equation*}
    \begin{align*}
        \abs{T_{t+1}(y)} &= \abs{T_t(y)} + \abs{T_t(x)} \\
         &\geq 2^{R_t(x)} + 2^{R_t(x)} \\
         &= 2^{R_t(x)+1} \\
         &= 2^{R_{t+1}(y)}
    \end{align*}
\end{proof}


\begin{lemma}
    The maximum rank in all disjoint-set trees after a sequence calls to \codefunc{Find} and \codefunc{Union} in a system of $n$ elements is at most $\lfloor \log n \rfloor$. 
\end{lemma}
\begin{proof}
    By the previous lemma \ref{lem:rank1}, 
    \begin{equation*}
        n \geq \abs{T(x)} \geq 2^{R(x)}, 
    \end{equation*}
    Which leads to 
    \begin{equation*}
        \lfloor \log n \rfloor \geq R(x).
    \end{equation*}
\end{proof}

\begin{lemma}
    \begin{equation}
        \abs{\{x | R(u)=r \}} \leq \frac{n}{2^r}
    \end{equation}
\end{lemma}
\begin{proof}
    \begin{align*}
        n &\geq \abs{\underset{R(x)=r}{\bigcup} T(x)} \\
        &= \sum_{R(x)=r} \abs{T(x)} \\
        &\geq \sum_{R(x)=r} 2^r \\
        &= \abs{\{x | R(u)=r\}}\cdot 2^r
    \end{align*}
\end{proof}
% To keep track of the vertices of a cluster, it will be represented as a \emph{cluster tree}, where an arbitrary vertex of the cluster will be the root, and any other vertex will be a child of the root. Whenever an edge $(u,v)$ is fully grown, we will need to traverse the trees of the two vertices $u$ and $v$, and check whether they have the same root; whether they belong to the same cluster. If not, a merge is initiated by making the root of smaller cluster a child of the bigger cluster. These functions, \codefunc{find} and \codefunc{union} respectively, are part of the Union-Find algorithm (not to be confused with the Union-Find decoder) \cite{tarjan1975efficiency}.

% Within the Union-Find algorithm, two features ensure that the complexity of the algorithm is not quadratic. 1). With \textbf{path compression}, as we traverse a tree from child to parent until we reach the root, we make sure that each vertex encountered that we have encountered along the way is pointed directly to the root. This doubles the cost of the \codefunc{find}, but speeds up any future call to any vertex on the traversed path. 2). With \textbf{weighted union}, we make sure to always make the smaller tree a child of the bigger tree. This ensures that the overall length of the path to the root stays minimal. In order to make this happen, we just need to store the size of the tree at the root.



\subsection{Union-Find decoder}

In the context of the surface code, the vertices $v\in V$ are the elements and each disjoint set is the set over vertices that support a cluster $\mathscr{B}(C)$ (definitions \ref{def:boundaryofedges}, \ref{def:cluster}). It is redundant to additionally store the sets $C$ for edges, as it lemma \ref{lem:singlecluster} implies that if a vertex $v\in \mathscr{B}(C)$, that all edges support by $v$ satisfy $e \in C$. 

Note that while the nodes in the tree are equivalent to vertices $v \in V$, parent pointers in the disjoint-set tree structure are \emph{not} equivalent to edges $e\in E$. The edge set $E$ with its erasure subset $\m{E}\subseteq E$ and subsequently cluster $C\subseteq \m{E}$ and forest $F_C\subseteq C$ are related to physical qubits and the lattice structure of the surface code, whereas edges of the tree $\bound(C)$ exists to point towards the representative element at the root.

The Peeling decoder solves exclusively for erasure errors. To be able to compare with the MWPM decoder, or other types of decoders, Pauli noise must be included. To this end, we use the independent noise model of equations \ref{qec:eq:bitflip} and \ref{qec:eq:phaseflip}, which means that again we can consider the primal and dual lattices separately. These noise channels introduce extra Pauli errors $P_p$ such that not all Pauli errors are in the erasures $P\not\subseteq \m{E}$, where $P = P_p\triangle P_\m{E}$. This means also that not all syndromes are in the boundary of the erasure $\sigma \not\subseteq \mathscr{B}(\m{E})$ (definition \ref{def:boundaryofedges}), and odd parity clusters can occur. Per theorem \ref{eq:anyevenparity}, the Peeling decoder cannot solve for these errors, and needs some alteration.

To this end, an altered erasure $\bar{\m{E}}$ that contains only even-parity clusters can be constructed from the syndrome $\sigma$ in a pre-processing step that is dubbed \emph{syndrome validation}. The validated erasure $\bar{\m{E}}$ is compatible with the peeling decoder. To do this, we sequentially grow the clusters with an odd parity by an half-edge on the boundaries on the clusters. When two odd parity clusters meet, the merged cluster will have a even parity, and can now be solved by the peeling decoder. 
\begin{proposition}
  The Peeling decoder can be altered to additionally solve for Pauli errors by a pre-processing step that initializes some altered erasure $\bar{E}$, such that theorem \ref{eq:anyevenparity} is satisfied, after which the Peeling decoder can proceed as before. 
\end{proposition}
\begin{figure}[]
  \centering
  \begin{tikzpicture}
    \node[draw, circle, OrangeRed, fill=OrangeRed!50!white, line width=1, text=black] (s1) at (0,1.5) {$\sigma$};
    \node[draw, circle, NavyBlue, fill=NavyBlue!50!white, line width =1, text=black] (e1) at (0,.5) {$\m{E}$};
    \node[draw, circle, OrangeRed, fill=OrangeRed!50!white, line width=1, text=black] (s2) at (5,1.5) {$\sigma$};
    \node[draw, circle, NavyBlue, fill=NavyBlue!50!white, line width =1, text=black] (e2) at (5,.5) {$\bar{\m{E}}$};
    \draw[OrangeRed, line width = 1] (s2) -- +(-1,0);
    \draw[NavyBlue, line width = 1] (e2) -- +(-1,0);
    \draw[OrangeRed, line width = 1, ->]  (s1) -- +(1,0) (s2) -- +(1,0);
    \draw[NavyBlue, line width = 1, ->] (e1) -- +(1,0) (e2) -- +(1,0);
    \node[draw, circle, Green, fill=Green!50!white, line width=1, text=black] (c) at (10,1) {$C$};
    \draw[Green, line width = 1, <-] (c) -- +(-1,0);
    \node[left=0 of s1, align=right] {syndrome};
    \node[left=0 of e1, align=right] {erasure};
    \node[right=0 of c, align=left] {correction};
    \draw[line width=1] (1,0) rectangle +(3,2) (6,0) rectangle ++(3,2);
    \node[text width = 2cm, align=center] at (2.5,1) {\emph{Syndrome validation}};
    \node[text width = 2cm, align=center] at (7.5,1) {\emph{Peeling decoder}};
  \end{tikzpicture}
  \caption{Stages of decoding of the Union-Find decoder. A pre-processing step that is called \emph{syndrome validation} is added to the Peeling decoder such that an altered erasure $\bar{\m{E}}$ is constructed that satisfies theorem \ref{eq:anyevenparity}, where all erasures have an even number of syndromes. (Figure inspired by \cite{delfosse2017almost})} 
  \label{fig:ufstages}
\end{figure}
Per lemma \ref{lem:singlecluster}, an edge can only be in a single cluster $C$ and a vertex in a single cluster boundary $\mathscr{B}(C)$. The merge of two clusters thus requires the update of the parent cluster of at least one set of vertices and edges. The challenge is to efficiently store this cluster index value such that the update complexity after each merge is minimized. This is done via the Union-Find data structure, and the altered decoder is therefore dubbed the Union-Find decoder \cite{delfosse2017almost}.

\paragraph{Data structure}
Now it is clear what information is exactly needed to grow the clusters using the Union-Find algorithm. We will need to store the cluster in a sort of cluster-tree. At the root of each tree we store the size and parity of that cluster in order to facilitate weighted union and to select the odd clusters. We will need to store the state of each edge (empty, half-grown, or fully grown) in a table called \codeword{support}. And we need to keep track of the boundary of each cluster in a \codeword{boundary} list.

\paragraph{The routine}
The full routine of the Union-Find decoder as originally described (\cite{delfosse2017almost}, Algorithm 2) is listed in Algorithm \ref{algo:uf}. In line 1-2, we initialize the data structures, and a list of odd cluster roots $\m{L}$. We will loop over this list until it is empty, or that there are no more odd clusters left.

In each growth iteration, we will need to keep track of which clusters have merged onto one, therefore the fusion list $\m{F}$ is initialized in line 4. We loop over all the edges from the \codeword{boundary} of the clusters from $\m{L}$ in line 5, and grow each edge by an half-edge in \codeword{support}. If an edge is fully grown, it is added to $\m{F}$.

For each edge $(u,v)$ in $\m{F}$, we need to check whether the neighboring vertices belong to different clusters, and merge these clusters if they do. This is done using the Union-Find algorithm in line 6. We call \codefunc{find(u)} and \codefunc{find(v)} to find the cluster roots of the vertices. If they do not have the same root, we make one cluster the child of another by \codefunc{union(u,v)}. Note that this does not only merge two existing clusters, also new vertices, which have themselves as their roots, are added to the cluster this way. We also need to combine the boundary lists of the two clusters.

Finally, we need to update the elements in the cluster list $\m{L}$. First, we replace each element $u$ with its potential new cluster root \codefunc{find(u)} in line 7. We can avoid creating duplicate elements by maintaining an extra look-up table that keeps track of the elements $\m{L}$ at the beginning of each round of growth. In line 8, we update the \codeword{boundary} lists of all the clusters in $\m{L}$, and in line 9, even clusters are removed from the list, preparing it for the next round of growth.

\begin{algo}[algotitle=Union-Find decoder \cite{delfosse2017almost}, label=algo:uf]
  \begin{algorithm}[H]
    \KwData{A graph $G = (V,E)$, an erasure $\m{E} \subseteq E$ and syndrome $\sigma \subseteq V$}
    \KwResult{A grown erasure $\m{E}'$ such that each cluster $\gamma \subseteq \m{E}$ is even}
    \BlankLine
    initialize cluster-trees, support and boundary lists for all clusters \;
    initialize list of odd cluster roots $\m{L}$\;
    \While{$\m{L} \neq \emptyset$}{
    initialize fusion list $\m{F}$ \;
    for all $u \in \m{L}$, grow all edges in the boundary list of cluster $C_u$ by a half-edge in support. If the edge is fully grow, add to fusion list $\m{F}$ \;
    for all $e={u,v} \in \m{F}$, if \emph{find($u$)} $\neq$ \emph{find($v$)}, then apply \emph{union($u,v$)}, append boundary list\;
    for all $u \in \m{L}$, replace $u$ with \emph{find($u$)} without creating duplicate elements\;
    for all $u \in \m{L}$, update the boundary list\;
    remove even clusters from $\m{L}$\;
    }
    run peeling decoder with grown erasure $\m{E}'$
  \end{algorithm}
\end{algo}

\subsubsection{Time complexity of the Union-Find decoder}


\chapter{Modifications to the Union-Find decoder}


\section{Object oriented approach}

Others who have implemented weighted growth (wrongly) use an algorithm that has a time complexity of $\m{O}(n\log n)$, which is worse than the main algorithm \cite{nando}. We will introduce a weighted growth algorithm that has a linear time complexity, and therefore preserving the inverse Ackermann time complexity of the Union-Find decoder.

\subsection{A new data structure}

\subsection{Finding clusters}

\section{Bucket Cluster Sort (BCS)}
To further increase the error threshold for the Union-Find decoder from $9.2\%$ to $9.9\%$, Nickerson implements weighted growth, where clusters are grown in increasing order based on their sizes \cite{delfosse2017}. However, the main problem with weighted growth is that the clusters now need to be sorted, and that after each growth iteration another round of sorting is necessary, due to the fact that the clusters have changed sizes due to growth and merges, and the order of clusters may have been changed. Nickerson has not given a description of how weighted growth in implemented. As the complexity of the algorithm is now dominated by the Union-Find algorithm, we need to make sure that weighted growth does not add to this complexity. To avoid this iterative sorting, we need to make sure that the insertion of a new element in our sorted list of clusters does not depend on the values in that list.

The Bucket Cluster sorting algorithm as described in this section is evolved from a more complicated version that is described in appendix \ref{ap.bucketsort}, which has a sub-linear complexity of $\m{O}(\sqrt{n})$.

\subsection{How to sort for weighted growth using BCS}

Let us now first look at what weighted growth for the Union-Find decoder exactly does. When a cluster is odd, there exists at least one path of errors connecting this cluster to a generator outside of this cluster. When the cluster grows, a number of edges $k$ that is proportional to the size $S$ of the cluster is added to the cluster. If $k \propto S$ new edges are added, only $1/k$ of these edges will correctly connect the cluster with the generator. Therefore, more "incorrect" edges will be added during growth of a larger cluster.

Note however, that the benefit of growing a smaller cluster is not substantial if the clusters are of similar size. Take two clusters $C_\alpha, C_\beta$ with size $S_\alpha <<S_\beta$, growth of cluster $C_\beta$ will add $\sim k_{\beta}/2$ "incorrect" edges on average, whereas growth of cluster $C_\alpha$ will add $\sim k_{\alpha}/2 << k_{\beta}/2$ edges as $k_{\alpha} \propto S_\alpha$ and $k_{\beta} \propto S_\beta$. However, if $S_\alpha \simeq S_\beta$, the number of added "incorrect" edges for both clusters will also be similar, and it is the same when $S_\alpha = S_\beta$.

\begin{lemma}\label{lem:incorrectedges}
  For two clusters $C_\alpha, C_\beta$ with size $S_\alpha << S_\beta$ the number of vertices in the clusters, $Grow(S_\beta)$ will add a smaller amount of \emph{incorrect} edges to the cluster, which are edges that are not part of the matching.
\end{lemma}

The sorting method that is suited for our case is \emph{Bucket sort}. In this algorithm, the elements are distributed into $k$ buckets, after which each bucket is sorted individually and the buckets are concatenated to return the sorted elements. Applied to the clusters, we sort the odd-parity clusters into $k$ buckets, which replaces the odd cluster list $\m{L}$. As the sizes of the clusters can only take on integer values, each bucket can be assigned a clusters size, and sorting of each individual bucket is not necessary. Furthermore, as we are not interested in the overall order of clusters, concatenating of the buckets is not necessary.

\subsubsection{Growing a bucket}
The procedure for the Union-Find decoder using the bucket sort algorithm is now to sequentially grow the clusters from a bucket starting from bucket 0, which contain the smallest single-generator clusters of size 1. After a round of growth, in the case of no merge event, these clusters are grown half edges, but are still size 1. We would therefore need twice as many buckets to differentiate between clusters without and with half-edges. Let us call them full-edged and half-edged clusters, respectively. Starting from bucket 0, even buckets contain full-edged clusters and odd buckets contain half-edged clusters of the same size. To grow a bucket, clusters are popped from the bucket, grown on the boundary, after which the clusters is to be distributed in a bucket again in a subroutine named \codefunc{Place}.

\begin{equation}\label{eq:bucket_place}
  \codefunc{Place}(C) = \begin{cases}
               C\rightarrow b_{2(S_C-1)}, & \mbox{if $S_C$ even} \\
               C\rightarrow b_{2(S_C-1)+1}, & \mbox{otherwise}
             \end{cases}
\end{equation}

In the case of no merge event, clusters grown from even bucket $b_i$ must be placed in odd bucket $b_{i + 1}$, as it does not increase in size, and clusters grown from odd bucket $j$ must be placed in even bucket $b_{j + 2k + 1}$ with $k \in \mathbb{N}_0$ the number of added vertices. Also in the case of a union event of clusters $C_\alpha$ and $C_\beta$, the new cluster $\codefunc{union}(C_\alpha, C_\beta) = C_{\alpha\beta}$ must be placed in a bucket $b_{\alpha\beta} > b_{\alpha}, b_{\alpha\beta} > b_{\beta}$. Thus we can grow the buckets sequentially, and need not to worry about bucket that have been already "emptied". This ensures that for two clusters $C_\alpha$ and $C_\beta$ with $S_\alpha < S_\beta$, cluster A will be grown first, adding a fewer amount of "incorrect" edges as per lemma \ref{lem:incorrectedges}. Clusters of the same size $S_\alpha=S_\beta$ are placed in the same bucket and their order of growth is dependent on their order of placements.

\begin{theorem}\label{the:bucket_order}
  Weighted growth is achieved by growing the odd clusters sequentially starting from bucket $b_0$. Grown odd clusters from bucket $b_c$ are added back to the bucket list using the \codefunc{place} subroutine, in a bucket $b_{g}$ where $g > c$.
\end{theorem}

\begin{lemma}\label{lem:bucket_suborder}
  Clusters $C_\alpha$ and $C_\beta$ with $S_\alpha = S_\beta$ are placed int the same bucket $b_{S_\alpha}$, and their growing order is dependent on the order of placement within the bucket.
\end{lemma}

\subsubsection{Faulty entries}

\begin{figure}
  \centering
  \includegraphics[width=\linewidth]{cluster_merge_A.pdf}
  \caption{Faulty entries of clusters can occur in the buckets, a) cluster that should not be there due to a merge event. Situation a can be solved by checking the parity of the cluster. Checking the parity of the root cluster solves a) and b). Checking the bucket\_number of the root cluster solves all.}\label{3.fig.clustermergeB}
\end{figure}

Now let us be clear: \emph{only odd parity clusters will be placed in buckets, but each bucket does not only contain odd parity clusters}. As a merge happens between two odd parity clusters $C_\alpha$ and $C_\beta$ during growth of $C_\beta$, cluster $C_\alpha$ has already been placed in a bucket, as it was still odd after its growth. But cluster $C_\alpha$ is now part of cluster $AB$ and has even parity, and the entry of cluster $C_\alpha$ is faulty. To prevent growth of the \emph{faulty entry}, we can check for the parity of the root cluster.

Furthermore, it is possible that another cluster $C_\gamma$ merges onto $C_{\alpha\beta}$, such that the cluster $C_{\alpha\beta\gamma}$ is odd again. Now, the faulty entry of cluster A passes the previous test. To solve this issue, we store an extra bucket number $C_b$ at the root of a cluster. Whenever a cluster increases in size or merges to an odd parity cluster, we first update the $C_b$ to the appropriate value and place it in its bucket. If the cluster merges to an even parity cluster, we update the $C_b$ to $Null$. Now, every time a cluster is popped from bucket $i$, we can just check weather the current bucket corresponds to the $C_b$ of the root cluster.

\begin{lemma}\label{lem:bucket_faulty}
  Each bucket $b_i$ does not necessary contain clusters that still belong to $b_i$. Growth of these faulty entries are prevented by storing the bucket number $j$ at the cluster $C_b = j$ during \codefunc{Place} and checking for $i=j$ and odd cluster parity add the beginning of \codefunc{Grow}.
\end{lemma}

\subsubsection{Number of buckets}
How many buckets do we exactly need? On a lattice there can be $n$ vertices, and a clusters can therefore grow to size $n$, spanning the entire lattice. Naturally, if a cluster spans the entire lattice, the solution given by the peeling decoder is now trivial. But we need to make sure that the decoder \emph{can} give a solution. Consider an odd cluster $C_\mu$ of size $S_\alpha~n/2$ which covers half the lattice. There must exists another odd cluster $C_\beta$ for matchings to exists, which has size $S_\beta\leq n/2$.
As per lemma \ref{the:bucket_order}, $C_\beta$ will grow before $C_\alpha$. As the remaining number of vertices is $n-S_\alpha-S_\beta$, $C_\beta$ can never grow larger than $C_\alpha$ and will merge into $C_\alpha$ if no other odd cluster exists. There exists a maximum cluster size $S_\mu$ for which after $\codefunc{Grow}(C_\mu)$ this is true. This cluster size $S_\mu$ is dependent on the code and the parity of lattice size $L$. We illustrate in figure \ref{fig:bucket_cmsizes} the clusters $C_\mu$ for the toric and planar code. Their maximum odd cluster size $S_\mu$ is listed in table \ref{tab_smax}, where $L'=L-1$ for the planar code.

\begin{lemma}
  Once an odd cluster $C_\alpha$ has reached a size $S_\alpha > S_\mu$, it is certain that a smaller cluster $C_\beta$ will grow in size before the bucket of $C_\alpha$ is reached, and it will merge into an even cluster $\codefunc{Union}(C_\alpha, C_\beta) = C_{\alpha\beta}$.
\end{lemma}

\begin{table}[h]
  \centering
  \begin{tabular}{|l|c|c|}
    \hline
    % after \\: \hline or \cline{col1-col2} \cline{col3-col4} ...
     & $L$ even & $L$ odd \\
     \hline
    Toric & $S_\mu = L\times (\frac{L}{2}-1) -1$ & $S_\mu = L\times ( \frac{L'}{2} -2) + (\frac{L'}{2}-1)$ \\
    \hline
    Planar & $S_\mu = L \times (\frac{L}{2} -1) $  & $S_\mu = L'\times \frac{L'}{2} -1$ \\
    \hline
  \end{tabular}
  \caption{The maximum cluster size $S_\mu$ for which it is not certain that another cluster will merge onto the current cluster, or the maximum cluster size for which a cluster is allowed to grow.  }\label{tab_smax}
\end{table}


This maximum cluster size $S_\mu$ for growth determines the number of buckets $k + 1$ we will need.
\begin{equation}\label{eq:bucket_numbuckets}
  k = 2(S_\mu-1)
\end{equation}
Any cluster with size $S\leq S_\mu$ will be placed into a bucket according to equation \ref{eq:bucket_place}. If $S>S_\mu$, the cluster will not be placed into a bucket, and shall be assigned bucket number $C_b=Null$, as there is no bucket available.


\def\QS{10}
\def\s{1}
\begin{figure}[h]
  \centering

  \begin{subfigure}{0.45\linewidth}
    \centering
        \begin{tikzpicture}
        \DRAWTORIC{5}
        \DRAWPLAQ{0}{0}
        \DRAWPLAQ{0}{1}
        \DRAWPLAQ{0}{2}
        \DRAWPLAQ{0}{3}
        \DRAWPLAQ{0}{4}
        \DRAWPLAQ{1}{0}
        \DRAWPLAQ{1}{1}
        \DRAWPLAQ{2}{3}
        \DRAWPLAQ{2}{4}
        \DRAWPLAQ{3}{0}
        \DRAWPLAQ{3}{1}
        \DRAWPLAQ{3}{2}
        \DRAWPLAQ{3}{3}
        \DRAWPLAQ{3}{4}
        \end{tikzpicture}
        \caption{Toric odd $L=5$}
  \end{subfigure}
  \hspace{1cm}
  \begin{subfigure}{0.45\linewidth}
    \centering
      \begin{tikzpicture}
        \DRAWPLANAR{6}
        \DRAWPLAQ{1}{1}
        \DRAWPLAQ{2}{1}
        \DRAWPLAQ{4}{1}
        \DRAWPLAQ{1}{2}
        \DRAWPLAQ{2}{2}
        \DRAWPLAQ{4}{2}
        \DRAWPLAQ{1}{3}
        \DRAWPLAQ{4}{3}
        \DRAWPLAQ{1}{4}
        \DRAWPLAQ{4}{4}
        \DRAWPLAQ{3}{4}
        \DRAWPLAQ{1}{5}
        \DRAWPLAQ{4}{5}
        \DRAWPLAQ{3}{5}
        \DRAWPLAQ{3}{3}
        \DRAWEPLAQ{0}{1}
        \DRAWEPLAQ{5}{1}
        \DRAWEPLAQ{0}{2}
        \DRAWEPLAQ{5}{2}
        \DRAWEPLAQ{0}{3}
        \DRAWEPLAQ{5}{3}
        \DRAWEPLAQ{0}{4}
        \DRAWEPLAQ{5}{4}
        \DRAWEPLAQ{0}{5}
        \DRAWEPLAQ{5}{5}
      \end{tikzpicture}
    \caption{Planar even $L=6$}
  \end{subfigure}
  \begin{subfigure}{0.45\linewidth}
    \centering
      \begin{tikzpicture}
        \DRAWTORIC{6}
        \DRAWPLAQ{0}{0}
        \DRAWPLAQ{1}{0}
        \DRAWPLAQ{3}{0}
        \DRAWPLAQ{4}{0}
        \DRAWPLAQ{0}{1}
        \DRAWPLAQ{1}{1}
        \DRAWPLAQ{3}{1}
        \DRAWPLAQ{4}{1}
        \DRAWPLAQ{0}{2}
        \DRAWPLAQ{1}{2}
        \DRAWPLAQ{3}{2}
        \DRAWPLAQ{4}{2}
        \DRAWPLAQ{0}{3}
        \DRAWPLAQ{1}{3}
        \DRAWPLAQ{3}{3}
        \DRAWPLAQ{4}{3}
        \DRAWPLAQ{0}{4}
        \DRAWPLAQ{1}{4}
        \DRAWPLAQ{3}{4}
        \DRAWPLAQ{4}{4}
        \DRAWPLAQ{0}{5}
        \DRAWPLAQ{2}{5}
        \DRAWPLAQ{3}{5}
        \DRAWPLAQ{4}{5}
    \end{tikzpicture}
    \caption{Toric even $L=6$}
  \end{subfigure}
  \begin{subfigure}{0.45\linewidth}
    \centering
      \begin{tikzpicture}
        \DRAWPLANAR{7}
        \DRAWPLAQ{1}{1}
        \DRAWPLAQ{2}{1}
        \DRAWPLAQ{4}{1}
        \DRAWPLAQ{5}{1}
        \DRAWPLAQ{1}{2}
        \DRAWPLAQ{2}{2}
        \DRAWPLAQ{4}{2}
        \DRAWPLAQ{5}{2}
        \DRAWPLAQ{1}{3}
        \DRAWPLAQ{2}{3}
        \DRAWPLAQ{4}{3}
        \DRAWPLAQ{5}{3}
        \DRAWPLAQ{1}{4}
        \DRAWPLAQ{2}{4}
        \DRAWPLAQ{4}{4}
        \DRAWPLAQ{5}{4}
        \DRAWPLAQ{1}{5}
        \DRAWPLAQ{2}{5}
        \DRAWPLAQ{4}{5}
        \DRAWPLAQ{5}{5}
        \DRAWPLAQ{1}{6}
        \DRAWPLAQ{4}{6}
        \DRAWPLAQ{5}{6}
        \DRAWEPLAQ{0}{1}
        \DRAWEPLAQ{6}{1}
        \DRAWEPLAQ{0}{2}
        \DRAWEPLAQ{6}{2}
        \DRAWEPLAQ{0}{3}
        \DRAWEPLAQ{6}{3}
        \DRAWEPLAQ{0}{4}
        \DRAWEPLAQ{6}{4}
        \DRAWEPLAQ{0}{5}
        \DRAWEPLAQ{6}{5}
        \DRAWEPLAQ{0}{6}
        \DRAWEPLAQ{6}{6}
      \end{tikzpicture}
    \caption{Planar odd $L=7$}
  \end{subfigure}
  \caption{The clusters $C_\mu$ with maximum cluster size $S_\mu$ that is allowed to grow is pictured for each case on the left. On the right, another cluster $C_\beta$ is pictured that has a maximum size while still separated from $C_\mu$.}\label{fig:bucket_cmsizes}
\end{figure}


\subsubsection{Largest bucket occurrence}
Not all buckets will be filled depending on the configuration of the lattice. It would therefore be redundant to go through all buckets just to find out that the majority of them is empty. To combat this, we can keep track of the largest filled bucket $b_M$. Whenever a bucket $b_i$ has been emptied and $i = M$, we can break out of the bucket loop to skip the remainder of the buckets.

\subsection{Complexity of BCS}
Let us focus on the operations on a single cluster before it is grown an half-edge. A cluster is placed in a bucket, popped from that bucket some time after, checked for faulty entry, and if passed grown. All these operations are done linear time $\m{O}(1)$. There are a maximum of $\m{O}(L^2) = \m{O}(n)$ buckets to go through. Thus the overall complexity of $\m{O}(n\alpha(N))$ is preserved.

\subsection{The BCS Union-Find decoder}




\section{Delayed Merge of boundary lists (DM)}

\begin{figure}
  \centering
  \includegraphics[width=\linewidth]{parent_child_A.pdf}
  \caption{The parent-child method for merging boundary lists. By storing a list of pointers of child clusters at the parent cluster, we needn't append the full boundary list from the child to the parent cluster. The tree representation (TR) is shown on the top right. } \label{3.fig.parentchildA}
\end{figure}

When two clusters merge, one needs to check for the larger cluster between the two, and make the smaller cluster the child of the bigger cluster, which lowers the depth of the tree and is called the \emph{weighted union rule}. Applied to the toric lattice, the Union-Find decoder also needs to append the boundary list (which contains all the boundary edges of a cluster) of the smaller cluster onto the list of the larger cluster. This method, as explained before, requires that the new boundary list needs to be checked again.

In our application, instead of appending the entire boundary list, we just add a pointer stored at the parent cluster to the child cluster. As a parent can have many children, the pointers are appended to a list \codeword{children}. When growing a cluster, we first check if this cluster has any child clusters. If yes, these child clusters will be grown first by popping them from the list, but any new vertices will always be added to the parent cluster. Also during and after a merge, we make sure that any new vertices are always added to the parent cluster. Any child will exist in the list of a parent for one round of growth, after which its boundaries will be grown, and the child is absorbed into the parent. This method also works recursively by keeping track of the root cluster instead of just the parent cluster, and many levels of parent-child relationships can exists, but again, only for one round of growth.

\begin{figure}
  \centering
  \includegraphics[width=\linewidth]{parent_child_B.pdf}
  \caption{Growing a merged boundary using the parent-child method. The tree representation (TR) is shown on the top right. }\label{3.fig.parentchildB}
\end{figure}


\section{Growing Edge Priority based on path degeneracy (GEP)}

\subsection{Degeneracy on connecting edges between Clusters (GEP-C)}
\subsection{Degeneracy on Vertices with connecting edges (GEP-V)}

\section{Growth Delay based on Matching Potential (EVENGROW)}

For the UF-decoder, each cluster $C^\alpha$ is represented by a set of vertices $\m{V}^\alpha = \{v_1, v_2, v_3 ... v_{C^\alpha_s}\}$, where $C^\alpha_s$ is the size of the cluster. Here, the $\m{V}^\alpha$ is stored in a tree, and each tree root is a unique identifier of the cluster. When new vertices $v_n$ are added during \codefunc{Grow}$(C^\alpha)$, they are added to the tree as a child of the root. When an edge is fully grown, we traverse the tree from the two neighboring vertices $v_x$, $v_y$ to their roots using $\codefunc{Find}(v_x)$ and $\codefunc{Find}(v_y)$ respectively. If $\codefunc{Find}(v_x) \neq \codefunc{Find}(v_y)$ the cluster are merged using $\codefunc{Union}(v_x, v_y)$ by making one vertex a child of another's root. The depth of the tree $\m{V}^\alpha$ is kept low due to \emph{path compression} and \emph{weighted union} of clusters.


[uneven growth potential in a cluster]

\subsection{Node representation of cluster}

\todo[inline]{What are nodes}

\todo[inline]{what can we achieve with delays}

In order to delay the growth of certain nodes in the cluster, we need to additionally store the set of nodes $\m{N}^\alpha = \{n^1, n^2, .... n^N\}$, which may contain both syndrome-nodes $a^i$ and junction-nodes $j^i$, and is also stored in a tree. Note that superscripts does not stand for "to the power of", but rather a indexer. We reserver the subscript for node variables. As we will see in the next section, the calculation of node-parities and node-delays is dependant on the direction in which $\m{N}^\alpha$ is traversed, we store the root at $C^\alpha$ such that growth occurs in the same direction as the delay calculation.

\begin{theorem}
  The set of nodes $\m{N}^\alpha = \{n^1, n^2, .... n^N\}$ of cluster $C^\alpha$ is stored as a tree with root $n^{r, \alpha}$, and exists next to the exists set of vertices $\m{V}^\alpha$. The function of $\m{N}^\alpha$ is to store the list of boundary edges at the nodes and growing each node according to the calculated node delay.
\end{theorem}


\subsection{Node delay calculation}

\todo[inline]{explain node parities}

\begin{lemma}
  Any node $n_i \in \m{N}^\alpha$ is a valid root.
\end{lemma}


The \emph{ancestry} is the path along which the parent-child relationships are defined between the nodes of a given set.

\begin{lemma}\label{lem:nodecalc_parity}
  The node parity $n^i_p$ is defined as the parity of the number of children nodes of node $n^i$, and is thus dependant on which node is set as root. If the ancestry in a set changes, node parities within the set become "undefined" and need to be recalculated. If some odd number of nodes is attached to $n^j$, node parities for nodes $\{n_i \in \m{N} | n_i \mbox{ ancestor of } n_j\}$ are flipped.   
\end{lemma}

\todo[inline]{delay calculation}

\begin{lemma}\label{lem:nodecalc_ancestrypath}
 The calculation of node delays is only valid while node parities within the set are defined along the same ancestry as the node delay calculation. 
\end{lemma}

\begin{lemma}\label{lem:nodecalc_undefineddelay}
  As the delay of a certain node $n^i$ becomes undefined, the delays of all children nodes of $n^i$ also becomes undefined. 
\end{lemma}

\begin{lemma}\label{lem:nodecalc_junction}
  Children junction-nodes do not add to the count of the number of children of its parent node, and therefore affect the parent node differently as per lemma \ref{lem:nodecalc_parity}. 
\end{lemma}

\todo[inline]{parity calculation}

To calculate the parities and delays in a given node set $\m{N}$ with "undefined" node parities and delays, we have to traverse the entire set. We denote the node set size as $S_\m{N}$ as the total number of nodes $n^i \in \m{N}$. Note that node set size is different from cluster size, which is the size of Vertex set $\m{V}$, and is referred to as $S_\m{V}$ from now on.

\begin{theorem}
  To prepare a cluster with node set $\m{N}$ and node root $n^r$ with undefined node parities and delays, we calculate node parities in $\m{N}$ by calling the head recursive function $\codefunc{calc_parity}(n^r)$, and sequentially calculate node delays in $\m{N}$ by calling the tail recursive function $\codefunc{calc_delay}(n^r)$.
\end{theorem}

\subsection{Growing a cluster}

The boundary list for each cluster is not stored at $C^\alpha$, but separately stored at each of the nodes $n^i$ in $\m{N}^\alpha$. To grow a cluster $\codefunc{Grow}(C^\alpha)$, we now traverse all $n^i \in \m{N}^\alpha$ from the root $n^{r, \alpha}$ and apply $\codefunc{GrowNode}(n^i)$, which increases the support of all boundary edges at node $n^i$ by 1. If this node hasn't waited enough $n^i_w - n^i_d - C^\alpha_{md}> 0$, we skip this node, add to the wait $n^i_w = n^i_w +1$ and apply \codefunc{GrowNode} on its children. New vertices grown from node $n^i$ are added to $\m{V}^\alpha$, while storing the node at each new vertex $v^j_n = n^i$. New boundary edges are appended to the boundary lists stored each node $n_j$. The number of nodes in $\m{N}^\alpha$ and the shape of the tree therefore does not change while no merge between clusters has happened.

\begin{theorem}\label{the:grownode}
  A cluster is grown by calling $\codefunc{GrowNode}(n^{r,\alpha})$, which first checks for the wait of the current node $n^i_w - n^i_d - C^\alpha_{md}> 0$ to grow its boundary edges, and then recursively applies \codefunc{GrowNode} to its children.
\end{theorem}

\subsection{Joint of node sets}
With the addition of the node set $\m{N}$, during a union of clusters $C^\alpha$ and $C^\beta$, we have to additionally combine the node sets $\m{N}^{\alpha}$ and $\m{N}^\beta$. Let us first make a clear distinction between the various routines. On the vertex set $\m{V}$ we $\codefunc{Union}(v^\alpha, v^\beta)$, the two vertices spanning the edge connecting two clusters. On node set $\m{N}$, we introduce here $\codefunc{Joint}(n^\alpha, n^\beta)$, which is called on the two nodes $n^\alpha=v_n^\alpha, n^\beta=v_n^\beta$ that connects to vertices $v^\alpha, v^\beta$, respectively. From now on, when we talk about the "merge clusters $C^\alpha$ and $C^\beta$", "the union of vertex sets $\m{V}^\alpha$ and $\m{V}^\beta$" or the "joint of node sets $\m{N}^{\alpha}$ and $\m{N}^\beta$", we always refer to the combination of these two routines.

Within the vertex set $\m{V}$, we apply \emph{path compression} and \emph{weighted union} to minimize the depth of the tree and therefore minimizing the calls to the \codefunc{Find} function. Similarly, in the node set $\m{N}$, we would also like to apply a set of rules to minimize the calls to \codefunc{calc_parity} and \codefunc{calc_delay}. As the structure of the tree is crucial in computing the parities and relative delays between the nodes, these rules will be quite different than in vertex set $\m{V}$. Our rules will be dependant on the parities of the joining node sets, which is the parity of the number of syndrome-node in the set. This is due to that junction-nodes do not add to the count of the number of children nodes per lemma \ref{lem:nodecalc_junction}. Note that the parity of a node set $\m{N}_p$ is therefore exactly the same as the parity of a cluster $C_p$, which also refer to the number of syndromes in the cluster.

\begin{lemma}
  The parity of node set $\m{N}_p$ is the parity in the number of syndrome-nodes $\sigma^i \in \m{N}$. The parity of node set $\m{N}_p$ is analogous to cluster parity $C_p$. 
\end{lemma}

\subsubsection{Joint to even node set}

Let us first consider the joint operation of two or more node sets, where the resulting node set $\m{N}^{e}$ is even. As this also means that the cluster is even, this cluster will not be grown, and naively we could say the we need not to worry about the parties and delays within $\m{N}^{e}$. If we do calculate the parities of a node set with even parity $\m{N}_p$, we will end up with an odd node $n^r_{p=o}$ as root of node set $\m{N}$. It therefore does not make sense to talk about node parities within an even node set. Luckily, but not coincidentally, if a node set is even, the cluster is even and therefore will not grow.

\begin{lemma}\label{lem:nodecalc_even}
  Node parities become undefined if multiple node sets joins into a new set $\m{N}$ with even parity $\m{N}_p$.
\end{lemma}

However, it is entirely possible that another cluster grows, and merges onto the cluster of $\m{N}^{e}$. In that case, we might think about recovering some of the node parities and delays that were calculated in the subsets of $\m{N}^e$, such that we don't have to traverse $\m{N}^{e}$ entirely for its parities and delays.

\subsubsection{Joint to odd node set}

The joint operation of an even $\m{N}^e$ and an odd node set $\m{N}^o$ in nodes $n^e, n^o$ respectively, and assume that this joining is due to the growth of odd cluster $\m{N}^o$ onto an "idle" $\m{N}^e$. The joint of these two sets leaves a new odd node set $\m{N}_{new}^o$ with subsets $\m{N}'^e$ and $\m{N}'^o$, referring to the original node sets. We are provided with two choices, a) make $n^e$ child of $n^o$, or b) make $n^o$ child of $n^e$. Note that the child node $n^c$ will become the \emph{sub-root} if subset $\m{N}'^c$.

If the subset $\m{N}'^e$ consists of only two odd node sets $\m{N}^o_0, \m{N}^o_1$, where $n_0, n_1$ are the joining nodes, the ancestry in $\m{N}''^o_0$ is preserved and $n_1$ is the sub-root of $\m{N}''^o_1$. We see that the parities in all ancestors of $n^0$ are flipped. Let's consider the cases and find whether we can minimize the parity and delay calculation in $\m{N}'^{e}$. 

For case a), an even number of nodes of $\m{N}'^e$ is attached to $n^o$, and the ancestry in $\m{N}'^o$ hasn't changed. This means that the parities in $\m{N}'^o$ do not change per lemma \ref{lem:nodecalc_parity}, and the delays in $\m{N}'^o$ are still valid as per lemma \ref{lem:nodecalc_ancestrypath}. In $\m{N}'^e$, as the ancestry path has changed, we are certain to traverse $\m{N}'^e$ from the sub-root $n^e$ to calculate the delays in this subset which is in the order of $S_{\m{N}^e}$.

In case b), as an odd number of nodes of $\m{N}'^o$ is attached to $n^e$, it means that parities of all ancestor of $n^e$ are flipped. As the ancestry in $\m{N}'^{o}$ has changed, we are certain to traverse $\m{N}'^o$ from the sub-root $n^o$ to calculate the delays which is in the order of $S_{\m{N}^o}$. The node parity changes in $\m{N}'^e$ will be dependant on the location of $n^e$ in the ancestry compared to $n^1$ and $n^2$, and all children nodes of these parity changes will have to recalculate their delays. Let's call the number of nodes needs to calculate parity and delays in $\m{N}'^e$ a value $S_e \leq S_{\m{N}^e}$, leaving the total number of operations in the order of $S_e + S_{\m{N}^o}$.

For $\m{N}'^e$ consisting of two subsets, keeping track of the parity changes between $n^e$, $n^0$ and $n^1$ is still a trivial task, and we might gain in minimization in operations in case b) compared to case a) for some value $S_e$ such that $S_e + S_{\m{N}^o} < S_{\m{N}^e}$. But as the number of subsets in $\m{N}'^e$ increases, the task of finding the ancestry paths of parity changes becomes analogous to traversing $\m{N}'^e$ entirely $S_e \rightarrow S_{\m{N}^e}$. To simplify things, we always choose case a. 

\begin{theorem}\label{the:nodejoint}
  The union of node sets $\m{N}^\alpha, \m{N}^\beta$ on nodes $n^\alpha, n^\beta$ respectively is performed with $\codefunc{Joint}(n^\alpha, n^\beta)$. If the resulting node set $\m{N}$ is odd, one of $\m{N}^\alpha$ and $ \m{N}^\beta$ is odd while the other is even, and $\codefunc{Joint}(n^\alpha, n^\beta)$ makes the node of the even set $n^e$ a child of the node of the odd set $n^o$. If the resulting node set $\m{N}$ is even, the choice is arbitrary. 
\end{theorem}



\subsection{Multiple joints per growth iteration}

[store sub-roots of even subsets at the root of the set, perform parity and delay calculation not directly after joint, but rather before growth]


\begin{lemma}\label{lem:oddisevenodd}
  An odd node set $\nset$ that is the result of some joint operations must consist of an odd subset $\nset'^o$ and an even subset $\nset'^e$, where the even subset $\nset'^e$ may consist of smaller sub-subsets $\nset''$. 
\end{lemma}

\begin{theorem}\label{the:calclist}
  For each joint operation between odd node set $\m{N}^o$ and even node set $\m{N}^e$ on nodes $n^o, n^e$ per theorem \ref{the:nodejoint}, we store the sub-root $n^e$ of subset $\m{N}'^e$ to a list $\m{C}$ at root $r^o$ of the resulting set $\m{N}^{res}$ of cluster $C^{res}$ 
  \begin{equation}\label{eq:the_nodejoint}
    n^e \rightarrow \m{C}_{r^o}
  \end{equation}
  If cluster $C$ is selected for growth as per theorem \ref{the:bucket_order}, we first check for nodes in $\m{C}_{n^r}$ at root and apply $\codefunc{calc_parity}(n^i)$ and $\codefunc{calc_delay}(n^i)$ for all nodes $n^i \in \m{C}_{n^r}$ to calculate parities and delays in undefined parts of the set. We then call $\codefunc{GrowNode}(n^r)$ per theorem \ref{the:grownode}. 
\end{theorem}

\subsection{Complexity of EVENGROW}

The contribution to the time complexity of the UF-EG decoder compared to the UF-decoder can be divided into two parts. First is the contribution by \codefunc{calc_parity} and \codefunc{calc_delay}. As these two functions are always called together per theorem \ref{the:calclist}, we can just introspect the number of calls to one of them, and call this contribution the \emph{delay} complexity. The second contribution will be caused by \codefunc{GrowNode}, as now we have to additionally traverse the node set tree's of each cluster to access its boundary edges as compared to a single boundary list per cluster. We call this second contribution the \emph{node} complexity.

\subsubsection{Delay complexity}

Consider an odd cluster represented by node set size $S_{\m{N}}$ that consists of some number of subclusters $'C^i$ with node subsets $'\nset^i$. As this cluster is odd, it will be selected for growth. And because it consists of a number of subsets, $\nset$ is bound to consist of an odd subset $'\nset^o$ and an even subset $'\nset^e$ (lemma \ref{lem:oddisevenodd}) on which we are to calculate the parities and delays (theorem \ref{the:calclist}).

\paragraph{Fragmentation of a node set}

Let us call this division of odd set into smaller odd and even subsets the \emph{partial fragmentation} of $\nset^o$. We can continue to partially fragment $'\nset^o$ into $''\nset^{o,o}$ and $''\nset^{o,e}$ the same way. We can apply an \emph{intermediate fragmentation} of $'\nset^e$ into 2 odd subsets $\{'\nset^{o_1}, '\nset^{o_2}\}$, and call the fragmentation of $\nset^o$ into a set of node sets $\m{F} = \{'\nset^o, '\nset^{o_1}, '\nset^{o_2}\}$ a \emph{fragmentation step}. Note that a note set $\nset^o$ can only be fragmented if $S_{\nset^o} \geq 3$, in which case the resulting subsets have size 1. 

\begin{lemma}\label{lem:partialfrag}
  Let the division of an odd node set $\nset^o$ into subsets $\m{F}_0'=\{'\nset^o, '\nset^e\}$ and subsequently into $\m{F}_0 = \{'\nset^o, '\nset^{o_1}, '\nset^{o_2}\}$ be the a fragmentation step of $\nset$.
  \begin{equation}\label{eq:partialfrag}
    \m{F}_0 = f(\nset^o) = f'(\{'\nset^o, '\nset^e\}) = \{'\nset^o, '\nset^{o_1}, '\nset^{o_2}\} \hspace{.3cm} | \hspace{.3cm} \bigcup \m{F} =\nset^o, \bigcap \m{F} = \emptyset, S_{\nset^o} \geq 3
  \end{equation}
\end{lemma}

Each odd node set of $\m{F}_0$ can undergo the same fragmentation step into odd subsets, leaving us again with a set of node subsets $\m{F}_1$. We can do this some $p$ times until our resulting set of node sets $\m{F}_{p-1}$ consists only of smallest possible node subsets $'^*\nset^o_m$ where $S_\nset=1$. To find the worst case complexity, we want to maximize the delay complexity within $\nset$, we are to find the sequence of joint operations that maximizes the sum of even node sets sizes $S_{\nset^e}$ in all partial fragmentations in the \emph{full fragmentation} of $\nset$.

Looking at this fragmentation from the other way, we have a set of size 1 node sets that undergo joint operations in each intermediate and partial fragmentations. In the intermediate fragmentation, two odd node sets join, and we do not add to the count of $N_delay$. In the partial fragmentation, an odd and an even note sets join, and we have to calculate the delays in the even node set before moving on to the next joint operation. 

\begin{lemma}
  Let the full fragmentation of $\nset$ be
  \begin{equation}\label{eq:fullfrag}
    F(\nset^o) = \underbrace{f(f(...f(\nset)))}_\text{p times} = \{'^*\nset^{o_0}_m, '^*\nset^{o_1}_m, '^*\nset^{o_2}_m, ... ,'^*\nset^{o_{N_\sigma}}_m \} \hspace{.3cm} | \hspace{.3cm} S_{'^*\nset^{o}_m} = 1, 
  \end{equation}
  where along each fragmentation step $k$ a partial fragmentation $\m{F}'_k$ is produced, the number of delay calculations is
  \begin{equation}\label{eq:maxdelay}
    N_{delay} = \sum_{k=0}^{p-1} \sum \{ S_{'^*\nset^e} | \forall \mbox{ even } '^*\nset^e \in \m{F}'_k \}. 
  \end{equation}
\end{lemma}

Note that here we ignore the fact that the intermediate fragmentation of an even node set may not result in two but many odd subsets. Let us call the number of odd subsets the \emph{fragmentation number} $N_f$. If an even node set $\nset^e$ is fragmented with $N_f=2$, a fragmentation steps will be
\begin{eqnarray*}
% \nonumber % Remove numbering (before each equation)
  \m{F}^e_1 &=& \{ \pre{1} \nset^{o_1}, \pre{1} \nset^{o_2}\},  \\
  \m{F}'^e_2 &=& \{\pre{2}\nset^{o_1,o}, \pre{2}\nset^{o_1,e}, \pre{2}\nset^{o_2,o}, \pre{2}\nset^{o_2,e} \}. 
\end{eqnarray*} 

\todo[noline]{change prior subsets from ' to pre-superscript notation}

For $N_f = 4$, a fragmentation step will be

\begin{eqnarray*}
% \nonumber % Remove numbering (before each equation)
  \m{F}^e_1 &=& \{ \pre{1} \nset^{o_1}, \pre{1} \nset^{o_2}\, , \pre{1} \nset^{o_3}\, , \pre{1} \nset^{o_4}\},  \\
  \m{F}'^e_2 &=& \{\pre{2}\nset^{o_1,o}, \pre{2}\nset^{o_1,e},  \pre{2}\nset^{o_2,o}, \pre{2}\nset^{o_2,e},  \pre{2}\nset^{o_3,o}, \pre{2}\nset^{o_3,e}, \pre{2}\nset^{o_4,o}, \pre{2}\nset^{o_4,e} \}. 
\end{eqnarray*}

If the size of $S_{\nset^o}$ is large enough, the sum of even node set sizes in these two kinds of fragmentations will be the same. However, the number of subsets in each fragmentation step has increased by a factor of 2, which means that the average size of subsets have decreased by 2. This means that the node set size decreases faster towards the minimum size of 3, as more fragmentation steps are applied. As the sum of even node set sizes in each fragmentation step is the same, increasing $N_f$ will decrease the number of fragmentation steps and thus the number of delay calculations $N_{delay}$. Thus our decision of $N_f=2$ in lemma \ref{lem:partialfrag} is correct.

\paragraph{Partial fragmentation ratio}
\todo[inline]{show ratio of 2/3}

\paragraph{Time complexity}
% \chapter{Threshold simulations}

To test for the threshold Pauli error value, we simulate for a large number of samples at various lattice sizes for a range of Pauli error rates around $p = 0.1$. For the threshold, only Pauli X errors are considered, as Pauli Z errors will give the same result. For each lattice size and Pauli error rate, the samples will return a probability rate of successful decodings $P_{succes}$. We will then fit the data to the function (\cite{chengyang}, equation 43):
\begin{eqnarray}
% \nonumber % Remove numbering (before each equation)
P_{succes} &=& A + Bx + Cx^2 + \begin{cases}
                                D_{even}\cdot L^{-1/\mu_{even}} &\mbox{L even}\\
                                D_{odd}\cdot L^{-1/\mu_{odd}} &\mbox{L odd}
                              \end{cases}\\
\mbox{with } x &=& (p - p_{thres})L^{1/\nu}
\end{eqnarray}\label{eq.4.fit}
where all but $P_{succes}$, $p$ and $L$ are fitting parameters. Note that there are distinct values for $D$ and $\mu$ for even and odd lattices. This is due to a discrepancy in the decoder threshold caused by a nonnegligible finite-size effect for even and odd lattices. Therefore, for each fit done on any dataset, only data from even or odd lattices will be selected. The fitting of the data will be done in Python using a least squared method. 

\begin{tabular}{|r|c|c|}
  \hline
  % after \\: \hline or \cline{col1-col2} \cline{col3-col4} ...
  & scales with $n$ & scales with (dependent on) $p$ \\
  \hline
  \hline
  number of clusters & \checkmark & increases for small $p$ \\ \hline
  size of clusters & only for large $p$ & \checkmark\\ \hline
  number of cluster growths & \checkmark & \checkmark\\ \hline
  number of merging events & \checkmark & increases for small $p$ \\ \hline
  number of child clusters & only for large $p$ & increases for small $p$ \\ \hline
\end{tabular}\\

\subsection*{Peeling algorithm: main components}

\begin{algorithm}[htpb]
\SetAlgoNoEnd
\SetKwInOut{comp}{Complexity}

\SetKwData{vertex}{Vertex}\SetKwData{cluster}{Cluster}
\SetKwData{erloc}{ErLoc}\SetKwData{anloc}{AnLoc}
\SetKwData{clusters}{Clusters}
\SetKwData{graph}{Graph}
\SetKwData{neigh}{Neighbors}

\SetKwFunction{FVN}{FindVertexNeighbors}

\KwData{\anloc $=$ anyon locations, \erloc $=$ erasure locations}
\KwResult{\graph object containing all the Clusters}
\BlankLine

\nlset{n} \For{\vertex in \anloc}{
    \If{\vertex not $\in$ \graph}{
    add \vertex to \graph}
    \If{\vertex $\notin$ \clusters}{
        add new \cluster\;
        \nlset{1} \neigh $=$ \FVN{\vertex, \cluster} (Algorithm \ref{al:fvn})\;
        \nlset{+n} \While(\tcc*[h]{a maximum of n extra vertices are added on initial loop}){\neigh $\neq$ None}{
            \For{\vertex in \neigh}{
                \nlset{1} new \neigh $=$ \FVN{\vertex, \cluster}
            }
        }
    }
}
\BlankLine
\comp{O(n)}
\caption{FindClusters}\label{al:fc}
\end{algorithm}


\begin{algorithm}[htpb]
\SetAlgoNoEnd
\SetKwInOut{comp}{Complexity}

\SetKwData{vertex}{Vertex}\SetKwData{cluster}{Cluster}
\SetKwData{graph}{Graph}\SetKwData{bound}{Boundary}
\SetKwData{edge}{Edge}\SetKwData{hedge}{HalfEdge}
\SetKwData{neigh}{Neighbors}\SetKwData{pcluster}{ParentCluster}
\SetKwData{remlist}{RemList}\SetKwData{mergelist}{MergeList}
\SetKwData{nblist}{NewBoundaryList}
\SetKwData{clusters}{Clusters}

\SetKwFunction{sgc}{SelectGrowClusters}
\SetKwFunction{gpc}{GetParentCluster}


\KwData{\graph containing even and uneven clusters}
\KwResult{\graph containing only even clusters}
\BlankLine

\nlset{n} select \cluster with \sgc (Algorithm \ref{al:sgc}) \;
\nlset{$\downarrow$}\While{uneven \clusters $\in$ \graph}{
    \nlset{n} \For(\tcc*[h]{total cluster growths scales with n}){\cluster in selection}{
        \For{\bound (\edge, near \vertex, far \vertex) in \cluster}{
            \BlankLine

            \eIf{near \hedge $\notin$ \clusters}{
                \tcc{edge not grown, first growth-step}
                \cluster\ $\leftarrow$ near \hedge \;
                \eIf{far \hedge $\in$ \clusters}{
                    \cluster\ $\leftarrow$ \edge \;
                    \nlset{n} \pcluster $=$ \gpc{far \edge}\ (Algorithm \ref{al:gpc})\;
                    \eIf{\pcluster $=$ \cluster}{
                        \remlist $\leftarrow$ \bound
                    }{
                        \mergelist $\leftarrow$ (\cluster, \pcluster)
                    }
                }{
                    no action, edge is grown a half-step
                }
                \BlankLine

            }{
                \tcc{edge is already half-grown, second growth-step}
                \cluster $\leftarrow$ \hedge\;
                \eIf{far \vertex $\notin$ \clusters}{
                    \cluster $\leftarrow$ \vertex \;
                    \nblist $\leftarrow$ far \vertex
                }{
                    \pcluster $=$ \gpc{far \vertex} \;
                    \If{\pcluster $\neq$ \cluster}{
                        \mergelist $\leftarrow$ (\cluster, \pcluster)
                    }
                }
            }
        }
        \BlankLine
        \nlset{n} \For(\tcc*[h]{dependent on number of merges}){\bound in \remlist}{
            \tcc{these boundaries are excluded from second growth-step}
            remove \bound from \cluster
        }
        \BlankLine
        \nlset{1} \For(\tcc*[h]{dependent on cluster size}){\vertex in \nblist}{
            \tcc{vertices lie on far end of completely grown edge, this is the new boundary}
            find new \bound with \fvn (Algorithm \ref{al:fvn})
        }
    }
    \BlankLine
    \nlset{n} \For{\cluster 1, \cluster 2) in \mergelist}{
        \nlset{n} \pcluster 1 $=$ \gpc{\cluster 1}\;
        \nlset{n} \pcluster 2 $=$ \gpc{\cluster 2}\;
        \If{not already merged}{
            \cluster 1 $\cup$ \cluster 2
        }
    }
    select grow clusters with \sgc \;
}

\BlankLine
\comp{$O(n^2)$}
\caption{GrowClusters}\label{al:gc}
\end{algorithm}

\begin{algorithm}[htpb]
\SetAlgoNoEnd
\SetKwInOut{comp}{Complexity}

\SetKwData{vertex}{Vertex}\SetKwData{cluster}{Cluster}
\SetKwData{graph}{Graph}\SetKwData{bound}{Boundary}
\SetKwData{edge}{Edge}\SetKwData{hedge}{HalfEdge}
\SetKwData{neigh}{Neighbors}\SetKwData{pcluster}{ParentCluster}
\SetKwData{newn}{NewNodes}\SetKwData{nvertex}{NeighborVertex}
\SetKwData{clusters}{Clusters}

\SetKwFunction{sgc}{SelectGrowClusters}
\SetKwFunction{gpc}{GetParentCluster}
\SetKwFunction{fvn}{FindVertexNeighbors}

\KwData{\graph containing even clusters}
\KwResult{\graph containing clusters of trees}
\BlankLine

\nlset{n} \For{\cluster in \graph}{
    get random \vertex from \cluster\;
    mark \vertex as traversed\;
    \newn $\leftarrow$ \vertex\;
    \nlset{$\downarrow$} \While(){\newn $\neq$ None}{
        \nlset{1} \For(\tcc*[h]{dependent on cluster size}){\vertex in \newn}{
            get \neigh (edges, vertices) of \vertex $\in$ \cluster\;
            \For{neighbor \vertex, \edge) in \neigh}{
                \eIf{neighbor \vertex is traversed}{
                    remove neighbor edge from \cluster
                }{
                    mark neighbor \vertex as traversed\;
                    \newn $\leftarrow$ neighbor \vertex
                }
            }
        }
    }
}


\BlankLine
\comp{O(n)}
\caption{TraverseTrees}\label{al:tt}
\end{algorithm}


\begin{algorithm}[htpb]
\SetAlgoNoEnd
\SetKwInOut{comp}{Complexity}

\SetKwData{vertex}{Vertex}\SetKwData{cluster}{Cluster}
\SetKwData{graph}{Graph}\SetKwData{bound}{Boundary}
\SetKwData{edge}{Edge}\SetKwData{hedge}{HalfEdge}
\SetKwData{neigh}{Neighbors}\SetKwData{pcluster}{ParentCluster}
\SetKwData{newn}{NewNodes}\SetKwData{nvertex}{NeighborVertex}
\SetKwData{clusters}{Clusters}

\SetKwFunction{sgc}{SelectGrowClusters}
\SetKwFunction{gpc}{GetParentCluster}
\SetKwFunction{fvn}{FindVertexNeighbors}

\KwData{\graph containing clusters of trees}
\KwResult{\graph containing the matching, syndrome edges}
\BlankLine

\nlset{n} \For{\cluster in \graph}{
    \nlset{$\downarrow$} \For{\vertex in \cluster}{
        \nlset{1} \While(\tcc*[h]{dependent on cluster size}){$\#$ connected \edge $=$ 1}{
            remove \edge from \cluster\;
            \If{\vertex $=$ anyon}{
                matching $\leftarrow$ \edge\;
                remove anyon status of \vertex\;
                flip anyon status of opposite \vertex
            }
            opposite \vertex is new \vertex
        }
    }
}


\BlankLine
\comp{O(n)}
\caption{PeelTrees}\label{al:pt}
\end{algorithm}

\FloatBarrier
\subsection*{Peeling algorithm: functions}

\begin{algorithm}[htpb]
\SetAlgoNoEnd
\SetKwInOut{comp}{Complexity}

\SetKwData{vertex}{Vertex}\SetKwData{edge}{Edge}
\SetKwData{cluster}{Cluster}\SetKwData{bound}{Boundary}
\SetKwData{erloc}{ErLoc}\SetKwData{anloc}{AnLoc}
\SetKwData{graph}{Graph}
\SetKwData{neigh}{Neighbors}

\SetKwFunction{FVN}{}

\KwData{\vertex, \cluster}
\KwResult{\neigh of input \vertex within the \cluster}
\BlankLine

\nlset{1} \For{neighboring \vertex, \edge of input \vertex}{
    \eIf(\tcc*[h]{newly found vertices and edges}){\vertex $\notin$ \graph}{
        \graph $\leftarrow$ \vertex, \edge\;
        \If{\vertex $\in$ \anloc}{
            \vertex is an anyon
        }
    }{
        \If(\tcc*[h]{vertex already added but edge not}){\edge $\notin$ \graph}{
            \graph $\leftarrow$ \edge
        }
    }
    \eIf{\edge $\in$ \erloc}{
        add \vertex, \edge to \cluster\;
        \If{\vertex $\notin$ a \cluster}{
            \neigh $\leftarrow$ \vertex
        }
    }{
        \bound of \cluster $\leftarrow$ (\vertex, \edge)
    }
}
\BlankLine
\comp{$O(1)$}
\caption{FindVertexNeighbors}\label{al:fvn}
\end{algorithm}




\begin{algorithm}[htpb]
\SetAlgoNoEnd
\SetKwInOut{comp}{Complexity}

\SetKwData{vertex}{Vertex}\SetKwData{edge}{Edge}
\SetKwData{cluster}{Cluster}\SetKwData{pcluster}{ParentCluster}
\SetKwData{bound}{Boundary}
\SetKwData{erloc}{ErLoc}\SetKwData{anloc}{AnLoc}
\SetKwData{graph}{Graph}
\SetKwData{neigh}{Neighbors}

\SetKwFunction{FVN}{}

\KwData{\graph}
\KwResult{list of \cluster objects selected for growth}
\BlankLine

\nlset{n} \For{\cluster, \pcluster in \graph}{
    \If{\cluster is \pcluster}{
        select \cluster for growth
    }
}
\If(\tcc*[h]{only smallers Clusters are selected}){Weighted growth}{
    \nlset{n} find smallest cluster size\;
    \nlset{n} find clusters with smallers size
}

\BlankLine
\comp{$O(n)$}
\caption{SelectGrowthClusters}\label{al:sgc}
\end{algorithm}


\begin{algorithm}[htb]

\SetAlgoNoEnd
\SetKwInOut{comp}{Complexity}

\SetKwData{vertex}{vertex}\SetKwData{edge}{edge}
\SetKwData{cluster}{Cluster}\SetKwData{pcluster}{ParentCluster}
\SetKwData{bound}{boundary}
\SetKwData{erloc}{ErLoc}\SetKwData{anloc}{AnLoc}
\SetKwData{graph}{Graph}
\SetKwData{clusterindex}{ClusterIndex}

\SetKwFunction{FVN}{}

\KwData{\cluster}
\KwResult{\pcluster}
\BlankLine

\pcluster $=$ \clusterindex(\cluster)\;
\nlset{n} \While{\pcluster $\
neq$ \cluster}{
    \cluster $=$ \pcluster\;
    \pcluster $=$ \clusterindex(\cluster)
}


\BlankLine
\comp{$O(n)$}
\caption{GetParentCluster}\label{al:gpc}
\end{algorithm}


\begin{appendices}
  \chapter{Bucket sort}\label{ap.bucketsort}

\paragraph{Maximum number of growth iterations}
To categorize the clusters into buckets, we need to know how many buckets we would need. To find out, let us look at two generators that are placed maximally far from each other. On a even toric lattice with length $l$, they are placed $l/2$ horizontally and $l/2$ vertically from each other. If clusters from each bucket grow simultaneously by a half-edge, these two single-generator clusters would meet each other after $l$ iterations, or including odd lattices, after $2\lfloor l/2 \rfloor$ iterations. As illustrated in Figure \ref{3.fig.maxgrowthit}, we see that now the entire lattice is covered, thus this is exactly the number of iterations needed.

\begin{lemma}
  Given a toric lattice with size $\{l,l\}$, a maximum number of $N_B = 2\lfloor l/2 \rfloor$ buckets is required for all generators to be connected, where clusters from each bucket is grown simultaneously.
\end{lemma}

\begin{figure}[htpb]
  \centering
  \includegraphics[width=\linewidth]{max_growth_iterations.pdf}
  \caption{If two single-generator clusters are placed a) maximally far from each other, b) a maximal number of growth iterations is needed. c) After $2\lfloor l/2 \rfloor$ iterations, the entire lattice is covered.}\label{3.fig.maxgrowthit}
\end{figure}

\paragraph{Analytical cluster categorization}
To sort or place the clusters into the buckets, we can again look at the case of the two generators. We presume that each growth iteration of the cluster is started from a different bucket. Knowing this, we can use the sequence of the size of the single-generator cluster as it grows.

\begin{equation}\label{3.eq.sequence}
  S_{cluster, i} = 1, 1, 5, 5, 13, 13, 25, 25, 41, 41, 61, 61, ...
\end{equation}

Note that a cluster does not increase in size in alternating iterations, as the growth of half-edges does not add new vertices to the cluster. We ignore these duplicate numbers for now. We find that this series of numbers can be written as a finite sum.

\begin{equation}\label{3.eq.series}
% \nonumber % Remove numbering (before each equation)
  S'_{cluster}(i) = 1 + \sum_{0}^{x=i} 4x = 2i(i+1) + 1
\end{equation}

We now have a very simple analytical function to place clusters into buckets, where the number of elements per bucket increases linearly with $4i + 8$. For a cluster of size $S$, its bucket number $i$ can be calculated by rewriting and flooring equation \ref{3.eq.series} $i=\lfloor (\sqrt{2S-1} - 1)/2 \rfloor$. Now earlier we disregarded the duplicate entries is $S_{cluster, i}$, so we need to take an extra step to check whether a cluster is in its half-grown state. These buckets are illustrated in Figure \ref{3.fig.bucketsranges}a.

\begin{definition}[Bucket place method 1]\label{3.def.placebucket1}
  A cluster with size $S$ is placed in bucket number $i$ of $l$ buckets with $i=2\lfloor (\sqrt{2S-1} - 1)/2 \rfloor$ if cluster is fully grown, or $i=2\lfloor (\sqrt{2S-1} - 1)/2 \rfloor + 1$ otherwise.
\end{definition}

\begin{figure}[htpb]
  \centering
  \includegraphics[width=\linewidth]{buckets_ranges.pdf}
  \caption{The first six buckets for a) method A from definition \ref{3.def.placebucket1} and b) method B from definition \ref{3.def.placebucket2}. Even buckets contains fully grown clusters, odd buckets contain half-grown clusters of the same size.}\label{3.fig.bucketsranges}
\end{figure}

Single-generator clusters, or clusters with size 1, is never caused by an only an erasure error. It might therefore by useful to give these clusters their own bucket, such that these single-generator clusters are always grown first, such as in Figure \ref{3.fig.bucketsranges}b.

\begin{definition}[Bucket place method 2]\label{3.def.placebucket2}
  For separate single-generator buckets, check whether the cluster has size 1, in which case the clusters belongs in bucket 0 or 1 depending on its growth state. Otherwise, a cluster with size $S$ belongs in bucket number $i$ of $l$ buckets with $i=2\lfloor (\sqrt{2S-3} - 1)/2 \rfloor + 1$ if cluster is fully grown, or $i=2\lfloor (\sqrt{2S-3} - 1)/2 \rfloor + 2$ otherwise.
\end{definition}

To compare method 1 of definition \ref{3.def.placebucket1} and method 2 of definition \ref{3.def.placebucket2}, let us find the size of the largest odd possible on a lattice. Given that any two odd parity clusters grow simultaneously, this size is $S_{max} = (\lfloor l/2\rfloor-1)l$. We calculate the bucket number (for the half-grown state) for both methods and plot them versus the lattice size in Figure \ref{3.fig.maxbucket}. The bucket number of the largest odd parity cluster for method 2 is exactly the largest available bucket. Therefore method 2 is most probably superior.

\begin{figure}[htpb]
  \centering
  \includegraphics[width=0.7\linewidth]{max_bucket.pdf}
  \caption{The largest bucket that will be utilized for methods A and B (see fig. \ref{3.fig.bucketsranges}). Method A does not utilize the largest two buckets. }\label{3.fig.maxbucket}
\end{figure}

The big benefit of this type of bucket sort is that we do not need to look at existing entries in the buckets. Only thing that is necessary is to initialize the buckets, which can be done in linear time.

\paragraph{Growing a bucket}

The procedure for the Union-Find decoder using the bucket sort algorithm is now to sequentially grow the buckets starting from bucket 0, which contains all the single-generator clusters. From each bucket, which can be a simple list, clusters are popped from the bucket, grown, and placed into a new bucket, a process which we will call \emph{growing a bucket}. In the case that no merge happens, clusters from even bucket $i$ are placed in odd bucket $i+1$. However, what happens to clusters grown from odd bucket $j$? It would be problematic if a cluster can be placed in even bucket $j-1$, which has the same size range, as that bucket comes first in the sequence, and thus will not be grown again. To proof that this will never happen, let's state the following:

\begin{lemma}
  A cluster with size $S_i = 2(i+1)(i+2) + 1$ grown from a single-generator cluster has the least number of neighbor vertices of all clusters of size $S_i$.
\end{lemma}

This is equivalent to saying that single-generator seeded cluster with size $S_i$ will have the least amount of new vertices added to itself for all clusters with the same size. This is fairly easy to proof, as there can be no more compactly organized cluster than a cluster grown from a single generator. We know odd parity clusters from this most compactly organized set must go into the next bucket, as the size of this set if clusters is what defines the sequence (eq. \ref{3.eq.sequence}). Thus less compactly organized odd parity clusters must at least grow into the next even bucket, and potentially grow into a higher even bucket.

\begin{lemma}
  In the case of no merging event, clusters from even bucket $i$ are grown into bucket $i+1$, and cluster from odd bucket $j$ are grown into bucket $j+2k+1$ with $k \in \mathbb{N}_0$.
\end{lemma}

We now know that our buckets can be grown sequentially, where we need not to worry about buckets that already been emptied.

\paragraph{Merging}
Up until now we have avoided talking about the merging of clusters, so let us finally address this. In the original paper of the Union-Find algorithm, when two clusters merge, one needs to check for the larger cluster between the two, and make the smaller cluster the child of the bigger cluster, which lowers the depth of the tree and is called the \emph{weighted union rule}. Applied to the toric lattice, the Union-Find decoder also needs to append the boundary list (which contains all the boundary edges of a cluster) of the smaller cluster onto the list of the larger cluster.

In our application, instead of appending the entire boundary list, we just add a pointer at the parent cluster to the child cluster. As a parent can have many children, the pointers are appended to a \emph{children list}. When growing a cluster, we first check if this cluster has any child clusters. If yes, these child clusters will be grown first by popping them from the list, but any new vertices will always be added to the parent cluster. Also during and after a merge, we make sure that any new vertices are always added to the parent cluster. Any child will exist in the list of a parent for one iteration, after which its boundaries will be grown, and the child is absorbed into the parent. This method also works recursively by keeping track of the root cluster instead of the parent cluster, and many levels of parent-child relationships can exists.

\paragraph{Faulty entries}
Now let us first be clear: \emph{only uneven parity clusters will be placed in buckets, but each bucket does not only contain uneven parity clusters}. As a merge happens between two uneven parity clusters A and B during growth of B, cluster A has already been placed in a bucket. But clusters A is now part of cluster AB and is even. So, to prevent growth of the \emph{faulty entry}, we just need to have an extra check of the parity of the root cluster.

\begin{figure}
  \centering
  \includegraphics[width=0.8\linewidth]{cluster_merge_A.pdf}
  \caption{bla}\label{3.fig.clustermergeA}
\end{figure}

\paragraph{Wastebasket}
If the lattice consists of more than just the two single-generator clusters, it is entirely possible for odd parity clusters to grow larger than $S_{max} = (l/2-1)l$. The bucketing formulas also does not prevent itself from outputting a larger bucket number than the available number of buckets. The elegance of this method is that whenever a cluster has a bucket number $i_1>l$, there will always be another \emph{smaller} cluster with bucket number $i_2\leq l$ that will merge into the larger cluster, which evens the parity. In another words, if a clusters bucket number $i$ is computed to be larger than $l$, we can figuratively put the cluster into the \emph{wastebasket}.


\end{appendices}

\printbibliography
\end{document}
