\chapter{Ackermann's function}\label{ap:ackermann}
When computing the complexity of the Union-Find data structure \cite{tarjan1975efficiency} with the \emph{union by rank} and \emph{path compression} rules, the Ackermann's function was used. In this chapter, we will show that this function is very fast-growing, and its inverse thus very slow-growing. After the initial publication by Ackermann \cite{ackermann1928hilbertschen}, many authors have modified the function to suit their purposes. For example, the Ackermann's function used for the Union-Find data structure by Tarjan \cite{tarjan1975efficiency} is defined by the following relations:
\begin{align*}
    A'(0,x) &= 2x, \\
    A'(k,0) &= 0 \text{ for } k \geq 1, \\
    A'(k,1) &= 2 \text{ for } k \geq 1, \\
    A'(k,x) &= A'(k-1, A'(k, x-1)) \text{ for } k \geq 1, x \geq 2,
\end{align*}
with the inverse defined as
\begin{equation*}
    \alpha'(n) = \min \{k | A'(k,4) \geq \log_2(n)\}, 
\end{equation*}
while the original function had three non-negative integer arguments. Tarjan himself used a different definition in a later publication \cite{tarjan1984worst}. In this thesis, we will take the definition by Kozen \cite{kozen1992design} in his simplified proof of the Union-Find complexity, with $k\geq 0$ and $x\geq 0$:
\begin{align}
    \nonumber A(0,x) &= x + 1\\
    A(k+1,x) &= A^x(k,x)
\end{align}
where $A^i(k,x)$ is the $i$-fold composition of $A(k,x)$ with itself on the second argument:
\begin{align}
    \nonumber  A^0(k,x) &= 1 \\
    A^x(k,x) &= \underbrace{A(k, A(k, ..., A(k,x)))}_{x \text{ times}}.
\end{align}
In other words, to compute $A(k+1,x)$, we start with $x$ and apply $x=A(k,x)$ recursively for $x$ times, where $A(k,x)$ itself may be the composition of many functions of $A(k-1,x)$. It is evident that the function is strictly rising, and that for any $x$, $x\leq A(k,x)$. 

For low values of $x$, we find that $A(k,x)$ stays constant:
\begin{align*}
    A(k,0) &= 1 \\
    A(k,1) &= 2
\end{align*}
But for $x\geq 2$, as the value for $k$ increases, the Ackermann's function grows extremely fast,
\begin{align*}
    A(0,x) &= x+1 \\
    A(1,x) &= A^x(0,x) = 2x  \\
    A(2,x) &= A^x(1,x) = x2^x \geq 2 \uparrow (x+1)\\
    A(3,x) &= A^x(2,x) \geq  \underbrace{2^{2^{2^{.^{.^2}}}}}_{x+1} = 2\uparrow ^2 (x+1) \\
    A(4,x) &= A^x(3,x) \geq 2\uparrow^2(2\uparrow^2(x+1)) = 2\uparrow ^3 (x+1)\\
    &\vdots
\end{align*}
where $\uparrow^i$ is the Knuth up-arrow notation for very large integers \cite{knuth1976mathematics}, denoted by the recursive relations
\begin{equation}
    a\uparrow^i b = \begin{cases}
        a^b &\text{if }i=1,\\
        1   &\text{if }i>1 \text{ and } b=0,\\
        a\uparrow^{i-1}(a \uparrow^i(b-1)) &\text{otherwise.}
    \end{cases}
\end{equation}

For $x=2$, Ackermann's function as a function of $k$ already grows at such speed, such that for $k=4$, the value $A(4,2)$ is already so incomprehensible big such it exceeds the number of atoms in the known, observable universe. 
\begin{align*}
    A(0,2) &= 3 \\
    A(1,2) &= 4  \\
    A(2,2) &= 8\\
    A(3,2) &= 2\cdot 2^{2(1+2^2)} = 2^11 = 2048\\
    A(4,2) &\geq 2\uparrow^2 2048 = \underbrace{2^{2^{2^{.^{.^2}}}}}_{2048}
\end{align*}
We define t inverse Ackermann's function $a(n)$ that focuses on this $k=2$:
\begin{equation}
    \alpha(n) = \min\{k | A(k,2) \geq n \}.
\end{equation}

For any system of physical size $n$, the inverse of Ackermann's function satisfies $\alpha(n) \leq 4$. This Ackermann's function grows at roughly the same rate as in the proof of the complexity of the Union-Find data structure by Tarjan \cite{tarjan1975efficiency}, and thus its inverse is also similar.
